\section{Tasks \& schedule}

The general outline of the project is to first get better acquainted with the 
limitations imposed by the halting problem in general, followed by an analysis 
of one or more ideas which have been successfully applied to perform automated 
termination analysis in variously constrained computing environments, with the 
preliminary focus on portraying these ideas in a higher-order programming 
context.

\begin{enumerate}

\item {\bf Read and analyze classical literature on the general undecidability 
of the halting problem and the consequences thereof.}

\begin{description}

\item [Period:] week 40/2011

\end{description}

\item {\bf While week number $<50$ repeat steps \ref{start-loop} to 
\ref{end-loop}.}

Steps \ref{start-loop} to \ref{end-loop} present an expected work-load in terms 
of weeks. This work-load can vary depending on the complexity of the articles 
chosen in step \ref{start-loop}. As already mentioned, the preliminary focus 
will be on the size-change termination principle, and this will hence be the 
first area of research, if time allows for it, other methods will be explored 
as well.

\begin{description}

\item [Period:] week 41-50/2011

\end{description}

\item {\bf Find a suitable amount of relevant scientific material and perform a 
preliminary analysis thereof.}\label{start-loop}\\

This task embodies finding a set of relevant articles, of both classical and 
derived nature.  To find a \emph{suitable} amount of scientific material it is 
both important to estimate the relevance of the considered articles as well as 
their complexity given the time constraints.  Such estimations cannot be 
derived from reading the titles alone, nor are they seldom clear from the 
articles' respective abstracts.  It is therefore expected that the preliminary 
analysis goes in lock-step with finding the material.

\begin{description}

\item [Expected work-load:] 2 weeks

\end{description}

\item {\bf Pick out the key concepts from the articles above and devise a 
simple language to which these concepts can be applied without loss of 
generality.}\ \\

This step involves reading through the previously chosen set of articles and 
picking out the key ideas. Some articles may prove to be of a lesser 
importance, or of greater complexity than originally estimated, and a brief 
detour to task \ref{start-loop} may be necessary.

The purpose of the language was already described in the learning objectives.  
In addition to that, it should be as simple and as small as possible.  There is 
seemingly no interest in devising any sort of interpreter for the language, 
however this should be trivial if the language indeed is as simple and small as 
intended. If it should prove too complicated to devise a single language for 
all the key concepts, a set of simple dialects is preferred over one more 
complicated language.

\begin{description}

\item [Expected work-load:] 2 weeks

\end{description}

\item {\bf Describe each key concept in terms of the 
language.}\label{end-loop}\ \\

This is expected to be the most time-consuming and challenging part of the 
project as it involves rereading the articles and analysing them in enough 
detail to be able to reconvey the ideas in own terms. The intent here is 
furthermore not to recap the articles iteratively and in full detail, but 
rather to recap the key ideas by combining both classical and derived articles 
to describe the chosen key concepts.

\begin{description}

\item [Expected work-load:] 2-4 weeks

\end{description}

\item {\bf Proof reading, overall project revision and conclusion.}

At this stage it should be possible to reason about the methods covered in the 
previous sections and conclude of what use those methods are in their 
respective computing environments, i.e. are they sound, are they complete, are 
they efficient, etc.

\begin{description}

\item [Period:] week 52/2011 - 01/2012

\end{description}

\end{enumerate}
