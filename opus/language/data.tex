\section{Data}

\D{} is a simple language where the emphasis is on the sizes of data. Hence,
the way that data values are constructed does not have to be particularly
practical, but all values have to be well founded and easily comparable.

The language \D{} is untyped, and represents all data in terms of
\emph{unlabeled ordered binary trees}, henceforth referred to as simply,
\emph{binary trees}. Such a tree is recursively defined as follows:

\begin{definition}

A binary tree is a set that is either empty, henceforth referred to as a leaf
or simply $0$, or contains a single unlabeled node with two binary trees as
it's left and right child, henceforth simply referred to as a node. We'll refer
to the set of all possible values in \D{} as $\mathbb{B}$.

\end{definition}

To operate on such trees we'll require a few primitives. Namely, a
representation of leafs, recursive construction and destruction of nodes, as
well as a way to tell nodes and leafs apart. Most of these will be derived in
the operational semantics of \D{}\footnote{See \referToSection{d-sos}.},
however, they will make use of the following primitive function:

\begin{definition}

The function $\cdot :\mathbb{B}\times\mathbb{B}\rightarrow\mathbb{B}$
constructs a node with the two arguments as it's left and right child,
respectively. We'll refer to this function, as well as the operator $\cdot$ in
general, as ``cons''. 

\end{definition}

Sometimes we'll refer to the \emph{shape} of a data value. A shape
specification starts at the root of a value, and specifies a few some immediate
nodes or leafs, leaving some sub-values unspecified. This will often be expressed
either in graphical binary tree notation, or using the syntax of \D{} for
constructing binary trees. In either case there will be sub-trees who's actual
structure is irrelevant. When using \D{}'s notation we'll make use of auxiliary
variables for such sub-trees, and when using graphical binary tree notation,
we'll make use of the conventional triangle. As an example, consider
\referToFigure{first-shape}.

\includeFigure{first-shape}{Representation of a value consisting of a node with
a node as its left child. The triangles represent sub-trees who's actual
structure is irrelevant to the shape specification.}
 
