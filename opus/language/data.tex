\section{Data}

The automated termination analysis techniques discussed in latter chapters will
rely heavily on the well-foundedness of the language's data types.

\begin{definition}\label{definition:well-foundedness} Let $\mathbb{B}$ denote
all the values of a data type $T$, and let $f:\mathbb{B}\rightarrow \mathbb{Z}^*$
be a surjective function, where $\mathbb{Z}^*$ is the set of natural numbers. $T$
is well-founded if $$\forall B\subseteq \mathbb{B} \left( B \neq \emptyset
\rightarrow \exists\ b' \in B \mid \forall\ b\in B\backslash\{b'\}\ f(b') <
f(b)\right).$$\end{definition}

We choose to keep \D{} untyped, and hence, there is only one data type to
ensure well-foundedness for. We chose to represent all data in terms of
\emph{unlabeled ordered binary trees}, henceforth referred to as simply
\emph{binary trees}.

\begin{definition}\label{definition:values} \D{} represents all data in terms
of binary trees. We refer to the set of all values representable in \D{} as
$\mathbb{B}$. A binary tree is a set that is either empty, referred to as a
leaf, or contains a single unlabeled node with two binary trees as its left and
right child, respectively.\end{definition}

To operate on such trees we'll require a few primitives. We'll need, a
representation of leafs, recursive construction and destruction of nodes, as
well as a way to tell nodes and leafs apart. Most of these primitives will be
defined in \referToSection{d-syntax} and \referToSection{d-sos}, however, we
are ready to define the construction function.

\begin{definition} The function (dot) $\cdot
:\mathbb{B}\times\mathbb{B}\rightarrow\mathbb{B}$ constructs a node with the
two arguments as it's left and right child, respectively. We'll refer to this
function, as well as the operator $\cdot$, as ``cons''.\end{definition}

\subsection{Size}

\referToDefinition{well-foundedness} made use of a surjective function $f :
\mathbb{B}\rightarrow\mathbb{Z}^*$. To ensure well-foundedness of \D{}'s data
values we need to define such a function or equivalently, define a notion of
the size of a data value in \D{}.

\begin{definition}\label{definition:size} The size of a value in \D{} is the
number of nodes in the tree representing that value.\end{definition}

\begin{theorem}\label{theorem:type-is-well-founded} The type of all values in
\D{} is well-founded.\end{theorem}

\begin{proof} The proof is two-fold, proving the existence of the surjective
function $f:\mathbb{B}\rightarrow\mathbb{Z}^*$. First, we prove by induction that
any natural number can be represented in \D{}:

\begin{description}[\setleftmargin{70pt}\setlabelstyle{\bf}]

\item [Base case] A leaf has no nodes, and hence represents the value $0$.

\item [Assumption] If we can represent $n\in\mathbb{Z}^*$ in \D{}, then we can
also represent $n+1\in\mathbb{Z}^*$ in \D{}. 

\item [Induction] Let $n$ be represented by some $b\in\mathbb{B}$, then $n+1$
can be represented by $0\cdot b$. 

\end{description}

Second, by \referToDefinition{values}, any $b\in\mathbb{B}$ has one and only
one number of nodes, hence it has one and only one representation in
$\mathbb{Z}^*$, indeed the number of nodes.\end{proof}

\begin{definition} \referToDefinition{size} allows us to WLOG overload the
binary operators $=$, $<$, $>$, $\geq$ and $\leq$, defined over
$\mathbb{Z}^*\times\mathbb{Z}^*$, for
$\mathbb{B}\times\mathbb{B}$.\end{definition}

\subsection{Visual representation}

To provide for a more comprehensible discussion, we'll sometimes visualize the
values we're dealing with. \referToFigure{visualization-example} shows a few
sample value visualizations.

\begin{definition}\label{definition:value-visualization} We visually represent
values in $\mathbb{B}$ using a common convention of drawing binary trees, with
the root at the top and the subtrees drawn in an ordered manner in a downward
direction. Nodes are represented by filled dots while leafs are represented by
hollow ones. We use the Reingold-Tilford algorithm\cite{reingold-tilford} for
laying out the trees.\end{definition}

\includeFigure{visualization-example}{A few sample value visualizations having
the sizes $1$, $2$, $2$, and $3$ respectively.}

Although visually, a strict increase is usually associated with an upwards
direction, and a strict decrease is usually associated with a downwards
direction, this definition implies the exact opposite. A strict increase in
value would imply more nodes and hence a downward extension of the binary tree
along one of the branches, while a strict decrease in value would imply fewer
nodes and hence and upward contraction of the binary tree in one of the
branches.

\subsection{Shapes}

\begin{definition}\label{definition:shape} A shape is an abstract description
of a value in \D{}. Shapes describe values, and values match shapes. A shape
$s$ describes a value $b$ if $b$ matches the shape $s$. We refer to the set of
all possible shapes as $\mathbb{S}$, and define the binary relation $\succ$ to
be the set $\left\{(b,s)\mid b\in\mathbb{B} \wedge s\in\mathbb{S} \wedge b\
\text{matches}\ s\right\}$.\end{definition}

\begin{definition}\label{definition:value-matches-shape} A shape is a binary
tree that is either a leaf, a triangle, or a node having two shapes as its left
and right child, respectively. Any value matches the triangle shape. Only the
leaf value matches the leaf shape. A node value matches a node shape if the
children of the node value match the respective children of the node
shape.\end{definition}

\begin{lemma} Any shape that contains at least one triangle describes
infinitely many values.\end{lemma}

\begin{proof} Follows directly from \referToDefinition{value-matches-shape}.
\end{proof} 

As with values, it might prove beneficial to the discussion to visualize the
shapes. \referToFigure{shape-visualization-example} shows a few visualizations
of shapes. 

\begin{definition} Generally we'll represent shapes as we represent values.
Triangle shapes will be represented with hollow triangles. \end{definition}

\includeFigure{shape-visualization-example}{A few shape visualization examples.
The leftmost shape describes values that are nodes with leafs as right children
and any trees as left children. The two leftmost values in
\referToFigure{visualization-example} match this shape.}

\begin{definition}\label{definition:shape-matches-shape} A shape
$s_1\in\mathbb{S}$ matches a shape $s_2\in\mathbb{S}$ if $v\succ s_2$, where
$v$ is a value constructed from $s_1$ by replacing all the triangles in the
binary tree with leafs. We hence overload the binary relation $\succ$ with the
set $\left\{(s_1,s_2)\mid s_1\in\mathbb{S} \wedge s_2\in\mathbb{S} \wedge s_1\
\text{matches}\ s_2\right\}$.\end{definition}

\begin{lemma} $\forall\ s_1, s_2 \in \mathbb{S} \left(\left(s_1\neq s_2 \wedge
s_1\succ s_2\right) \rightarrow s_2\nsucc s_1\right)$\end{lemma}

\begin{proof}If $s_1\neq s_2$ and $s_1\succ s_2$, then by
\referToDefinition{value-matches-shape}, $s_1$ must have more nodes than $s_2$,
and by \referToDefinition{value-matches-shape} and
\referToDefinition{shape-matches-shape}, a shape with more nodes cannot match a
shape with fewer nodes.\end{proof}


