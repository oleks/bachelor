\section{Data}

\D{} is a simple language where the emphasis is on the sizes of data. Hence,
the way that data values are constructed does not have to be particularly
practical, but all values have to be well founded and easily comparable.

The language \D{} is untyped, and represents all data in terms of
\emph{unlabeled ordered binary trees}, henceforth referred to as simply,
\emph{binary trees}. Such a tree is recursively defined as follows:

\begin{definition}

A binary tree is a set that is either empty or contains a single unlabeled node
with two binary trees as it's left and right child, respectively.

\end{definition}

For simplicity, we'll sometimes refer to the empty set as a \emph{leaf}, and
the unlabeled node simply as \emph{node}.

\begin{definition}

The set of all possible binary trees is denoted by $\mathbb{B}$.

\end{definition}

To operate on such trees we'll require a few basic things, namely a
representation of leafs, recursive construction and destruction of nodes, as
well as a way to tell nodes and leafs apart. Most of these we'll be derived in
the operational semantics \referToSection{d-sos}, however, we do require the
following basic definitions:

\begin{definition}

The leaf with the atom $0$. Clearly, $0\in\mathbb{B}$.

\end{definition}

\begin{definition}

The function $\cdot
:\mathbb{B}\times\mathbb{B}\rightarrow\mathbb{B}$ constructs a node with the
two arguments as it's left and right child, respectively. 

\end{definition}

