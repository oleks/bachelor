\section{Data}

We chose to keep \D{} untyped.

\begin{definition} Let $\mathbb{B}$ denote the infinite set of values
representable in \D{}.\end{definition}

\begin{definition} Let the data type $T$ range over the set
$\mathbb{B}$.\end{definition}

The automated termination analysis techniques discussed in latter chapters will
rely heavily on the well-foundedness of the language's data types.

\begin{definition}\label{definition:well-foundedness} Let
$f:\mathbb{B}\rightarrow \mathbb{N}^0$ be a surjective function. $T$ is
well-founded iff $$\forall B\subseteq \mathbb{B} \left( B \neq \emptyset
\rightarrow \exists\ b' \in B \forall\ b\in B\backslash\{b'\}\ f(b') < f(b)
\right).$$\end{definition}

\begin{definition}\label{definition:values} A value $b\in\mathbb{B}$ is a
either $\emptyset$ or $\left\langle b_{left},b_{right} \right\rangle :
\mathbb{B} \times \mathbb{B}$.\end{definition}

\referToDefinition{values} implies that we chose to represent all values in
\D{} in terms of \emph{unlabeled ordered binary trees}, henceforth referred to
as simply \emph{binary trees}.  We refer to empty binary trees, i.e.
$\emptyset$, as \emph{leafs}. For any nonempty binary tree $b= \left\langle
b_{left},b_{right} \right\rangle : \mathbb{B} \times \mathbb{B}$, we refer to
$b$ as a \emph{node}, $b_{left}$ as the \emph{left child} of $b$, and
$b_{right}$ as the \emph{right child} of $b$.

\begin{definition}\label{definition:value-leaf} We represent a leaf with the
atom $0$, and define the binary relation $\cdot$ to be the set\\ $\left\{
\left\langle b_{left},b_{right} \right\rangle \mid b_{left}, b_{right}
\in\mathbb{B} \right\}$.\end{definition}

We'll sometimes refer to the $\cdot$ operator as ``cons''.

\begin{theorem}\label{theorem:infinite-values} The set $\mathbb{B}$ is
infinite.\end{theorem}

\begin{proof} Follows from \referToDefinition{values}.\end{proof}

\subsection{Size}

\referToDefinition{well-foundedness} made use of a surjective function $f :
\mathbb{B}\rightarrow\mathbb{N}^0$. To ensure well-foundedness of \D{}'s data
values we need to define such a function or equivalently, define a notion of
the size of a data value in \D{}.

\begin{definition}\label{definition:size} The size of a value $b\in\mathbb{B}$
is the number of nodes in the value.\end{definition}

\begin{theorem}\label{theorem:type-is-well-founded} There exists a surjective
function $f:\mathbb{B}\rightarrow\mathbb{N}^0$.\end{theorem}

\begin{proof} The proof is two-fold. First, we prove by induction that any
$n\in\mathbb{N}^0$ can be represented in \D{}:

\begin{description}[\setleftmargin{70pt}\setlabelstyle{\bf}]

\item [Base case] A leaf has no nodes, and hence represents the value $0$.

\item [Assumption] If we can represent $n\in\mathbb{N}^0$ in \D{}, then we can
also represent $n+1\in\mathbb{N}^0$ in \D{}. 

\item [Induction] Let $n$ be represented by some $b\in\mathbb{B}$, then $n+1$
can be represented by $0\cdot b$. 

\end{description}

Second, by \referToDefinition{values}, any $b\in\mathbb{B}$ has one and only
one number of nodes, hence it has one and only one representation
$n\in\mathbb{N}^0$, indeed the number of nodes.\end{proof}

\begin{corollary} $T$ is well-founded.\end{corollary}

\begin{proof} Follows from \referToDefinition{well-foundedness} and
\referToTheorem{type-is-well-founded}.\end{proof} 

\begin{definition} \referToDefinition{size} allows us to WLOG overload the
binary operators $=$, $<$, $>$, $\geq$ and $\leq$, defined over
$\mathbb{N}^0\times\mathbb{N}^0$, for
$\mathbb{B}\times\mathbb{B}$.\end{definition}

\subsection{Visual representation}

To provide for a more comprehensible discussion, we'll sometimes visualize the
values we're dealing with. \referToFigure{visualization-example} shows a few
sample value visualizations.

\begin{definition}\label{definition:value-visualization} We visually represent
values in $\mathbb{B}$ using a common convention of drawing binary trees, with
the root at the top and the subtrees drawn in an ordered manner in a downward
direction. Nodes are represented by filled dots while leafs are represented by
hollow ones. We use the Reingold-Tilford algorithm\cite{reingold-tilford} for
laying out the trees.\end{definition}

\includeFigure{visualization-example}{A few sample value visualizations having
the sizes $1$, $2$, $2$, and $3$ respectively.}

Although visually, a strict increase is usually associated with an upwards
direction, and a strict decrease is usually associated with a downwards
direction, this definition implies the exact opposite. A strict increase in
value would imply more nodes and hence a downward extension of the binary tree
along one of the branches, while a strict decrease in value would imply fewer
nodes and hence and upward contraction of the binary tree in one of the
branches.

\subsection{Shapes}

A shape is an abstract description of a value in \D{}. Shapes \emph{describe}
values, and values \emph{match} shapes, in particular, a shape describes a
value iff the value matches the shape.

\begin{definition} We refer to the set of all possible shapes as
$\mathbb{S}$.\end{definition}

\begin{definition} A shape $s\in\mathbb{S}$ is either $\emptyset$,
$\left\langle s_{left},s_{right} \right\rangle : \mathbb{S} \times
\mathbb{S}$, or $\any$.\end{definition}

We refer to $\emptyset\in\mathbb{S}$ as the leaf shape, $\any\in\mathbb{S}$ as
the triangle shape, and any $\left\langle s_{left},s_{right} \right\rangle
\in\mathbb{S}$ as a node shape.

\begin{definition} We refer to the leaf shape as $0$ and the triangle shape
merely as some $s\in\mathbb{S}$. We also overload the binary relation $\cdot$
with the set $\left\{ \left\langle s_{left}, s_{right} \right\rangle \mid
s_{left}, s_{right} \in\mathbb{S} \right\}$.\end{definition}

\begin{definition}\label{definition:succ} We define the binary relation
$\succ$, read ``matches'', ranging over $\mathbb{B}\times\mathbb{S}$ using the
following subdefinitions:

\begin{enumerate}

\item \label{definition:any-shape} $\forall\ b\in\mathbb{B}\ b\succ\any$.

\item \label{definition:leaf-shape} $ \left( 0\succ \emptyset \right) \wedge
\left( \forall\ b\in\mathbb{B}\backslash\{0\}\ b\not{\succ} \emptyset \right)$.

\item \label{definition:node-shape} $\forall\ s=\left\langle s_{left},s_{right}
\right\rangle \in\mathbb{S}\ \forall\ b= \left\langle b_{left},b_{right}
\right\rangle \in\mathbb{B}\ \left( \left( b_{left} \succ s_{left} \wedge
b_{right} \succ s_{right} \right) \rightarrow b \succ s \right)$.

\end{enumerate}

\end{definition}

\begin{lemma} Any shape that contains a triangle shape in its binary tree,
describes infinitely many values.\end{lemma}

\begin{proof} Follows directly from \referToDefinition{succ} and
\referToTheorem{infinite-values}.\end{proof} 

As with values, it might prove beneficial to the discussion to visualize the
shapes. \referToFigure{shape-visualization-example} shows a few visualizations
of shapes. 

\begin{definition} Generally we'll visually represent shapes as we represent
values. Triangle shapes will be represented with hollow triangles.
\end{definition}

\includeFigure{shape-visualization-example}{A few shape visualization examples.
The leftmost shape describes values that are nodes with leafs as right children
and any trees as left children. The two leftmost values in
\referToFigure{visualization-example} match this shape.}

\begin{definition}\label{definition:shape-matches-shape} Let $f : \mathbb{S}
\rightarrow \mathbb{B}$ be a surjective function that given a shape
$s\in\mathbb{S}$ transforms it into a $b\in\mathbb{B}$ by replicating the
binary tree, except that any $\any$ is replaced by $0$. We overload the binary
relation $\succ$ with the set $\left\{ \left\langle s_1, s_2 \right\rangle \mid
s_1,s_2 \in\mathbb{S}, f(s_1)\succ s_2 \right\}$.\end{definition}

\begin{lemma} $\forall\ s_1, s_2 \in \mathbb{S} \left(\left(s_1\neq s_2 \wedge
s_1\succ s_2\right) \rightarrow s_2\nsucc s_1\right)$\end{lemma}

\begin{proof}If $s_1\neq s_2$ and $s_1\succ s_2$, then by
\referToDefinition{succ}, $s_1$ must have more nodes than $s_2$, and by
\referToDefinition{succ}, a shape with more nodes cannot match a shape with
fewer nodes.\end{proof}
