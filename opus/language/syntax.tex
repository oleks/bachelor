\section{Syntax}\label{section:d-syntax}

We describe the syntax of \D{} in terms of an extended Backus-Naur
form\footnote{The extension lends some constructs from regular expressions to
achieve a more concise dialect. The extension is described in detail in
\referToAppendix{ebnf}.}. This is a core syntax definition, and other, more
practical, syntactical features may be defined later on as needed. The initial
non-terminal is $\nonterm{program}$.

\begin{align}
\nonterm{program}\ ::=&\ \nonterm{clause}^*\ \nonterm{expression}\\
\nonterm{expression}\ ::=&\ \nonterm{element}\ (\ \term{.}\ \nonterm{expression}
\ )\ ?\\
\nonterm{element}\ ::=&\ \term{0}\ |\ \term{(}\ \nonterm{element}\ \term{)}\ |
\ \nonterm{name}\ |\ \nonterm{application}\\
\nonterm{application}\ ::=&\ \nonterm{name}
\ \nonterm{expression}^+\\
\nonterm{clause}\ ::=&\ \nonterm{name}\ \nonterm{pattern}^+
\ \term{:=}\ \nonterm{expression}\\
\nonterm{pattern}\ ::=&\ \nonterm{pattern-element}\ (\ \term{.}
\ \nonterm{pattern}\ )\ ?\\
\nonterm{pattern-element}\ ::=&\ \term{0}\ |\ \term{\_}\ |\ \term{(}
\ \nonterm{pattern}\ \term{)}\ |\ \nonterm{name}\\
\nonterm{name}\ ::=&\ [\term{a}\mathmono{-}\term{z}]
\ \left (\ [\term{-}\ \term{a}\mathmono{-}\term{z}]^*
\ [\term{a}\mathmono{-}\term{z}]\ \right )?
\end{align}

\begin{definition} \referToTable{sos-definitions} defines shorthands for
various language constructs. We'll often refer to these in further discussions.
Additionally, we'll let the atoms $0$ and $\_$ represent
themselves.\end{definition} 

\makeTable[h!]
{sos-definitions}
{Shorthands for various language constructs for use in latter discussions. We
provide shorthands for an instance, a list, and the space of a construct. For
instance, $x$ is some particular expression, $X$ is some particular list of
expressions, and $\mathbb{X}$ is the set of all possible expressions.}
{|l|c|c|c|}
{\textbf{Description}&\textbf{Instance}&\textbf{List}&\textbf{Space}}
{
Expression & $x$ & $X$ & $\mathbb{X}$\\
Element (of an expression) & $e$ & $E$ & $\mathbb{E}$\\
Function & $f$ & $F$ & $\mathbb{F}$\\
Clause & $c$ & $C$ & $\mathbb{C}$\\
Pattern & $p$ & $P$ & $\mathbb{P}$\\
Value & $b$ & $B$ & $\mathbb{B}$\\
Name & $v$ & $V$ & $\mathbb{V}$
}

\begin{definition} For any given $n\in\mathbb{N}$ and $P\subseteq\mathbb{P}$,
we say that $n\in P$ if $n$ occurs in some $p\in P$.\end{definition}

\begin{definition} A clause $c$ is a tuple $(n_c,P_c,x_c)$, where $n_c$ is the
name of the clause, $P_c$ is the list of patterns of the clause, and $x_c$ is
the expression of the clause.\end{definition}

\begin{definition} A function $f$ is a non-empty set of clauses $C$, s.t.
$\forall\ c_1,c_2\in C \left(|P_{c_1}|=|P_{c_2}|\wedge n_{c_1}=n_{c_2}\right)$,
where $|P_{c_1}|$ and $|P_{c_2}|$ are the respective sizes of the pattern lists
of the clauses $c_1$ and $c_2$, and $n_{c_1}$ and $n_{c_2}$ are their
respective names. A function signature for the set of clauses $C$ is hence the
tuple $(n,|P|)$, s.t. $\forall\ c\in C (|P_c|=|P|\wedge n_c=n)$.  We'll adopt
the Erlang notation when talking about function signatures, i.e. if we have a
function \mono{less} that takes in two parameters, we'll refer to it as
\mono{less/2}.\end{definition}

0-ary clauses are disallowed to avoid having to deal with constants in general.
The term $\term{\_}$ in $\nonterm{pattern-element}$ is the conventional
wildcard operator; it indicates a value that won't used in the clause
expression, but some value has to be there for an argument to match the
pattern. Furthermore, as will be clear from the semantics, multiple occurrences
of \term{\_} in a clause pattern list does not indicate that the same value has
to be in place for each \term{\_}. 

% TODO this should be clear from the semantics.

% Multiple wildcards in the parameter list indicate possibly different value
% arguments, while multiple occurances of the same variable name in the parameter
% list are disallowed.
