\section{Syntax}\label{section:d-syntax}

We describe the syntax of \D{} in terms of an extended Backus-Naur
form\footnote{The extension lends some constructs from regular expressions to
achieve a more concise dialect. The extension is described in detail in
\referToAppendix{ebnf}.}. This is a core syntax definition, and other, more
practical, syntactical features may be defined later on as needed. The initial
non-terminal is $\nonterm{program}$.

\begin{align}
\nonterm{program}\ ::=&\ \nonterm{clause}^*\ \nonterm{expression}\\
\nonterm{expression}\ ::=&\ \nonterm{element}\ (\ \term{.}\ \nonterm{expression}
\ )\ ?\\
\nonterm{element}\ ::=&\ \term{0}\ |\ \term{(}\ \nonterm{element}\ \term{)}\ |
\ \nonterm{name}\ |\ \nonterm{application}\\
\nonterm{application}\ ::=&\ \nonterm{name}
\ \nonterm{expression}^+\\
\nonterm{clause}\ ::=&\ \nonterm{name}\ \nonterm{pattern}^+
\ \term{:=}\ \nonterm{expression}\\
\label{nonterm-pattern}
\nonterm{pattern}\ ::=&\ \nonterm{pattern-element}\ (\ \term{.}
\ \nonterm{pattern}\ )\ ?\\
\label{nonterm-pattern-element}
\nonterm{pattern-element}\ ::=&\ \term{0}\ |\ \term{\_}\ |\ \term{(}
\ \nonterm{pattern}\ \term{)}\ |\ \nonterm{name}\\
\nonterm{name}\ ::=&\ [\term{a}\mathmono{-}\term{z}]
\ \left (\ [\term{-}\ \term{a}\mathmono{-}\term{z}]^*
\ [\term{a}\mathmono{-}\term{z}]\ \right )?
\end{align}

\begin{definition} \referToTable{sos-definitions} defines shorthands for
various language constructs. We'll often refer to these in further discussions.
Additionally, we'll let the atoms $0$ and $\_$ represent
themselves.\end{definition} 

\makeTable[h!]
{sos-definitions}
{Shorthands for various language constructs for use in latter discussions. We
provide shorthands for an instance, a list, and the space of a construct. For
instance, $x$ is some particular expression, $X$ is some particular list of
expressions, and $\mathbb{X}$ is the set of all possible expressions.}
{|l|c|c|c|}
{\textbf{Description}&\textbf{Instance}&\textbf{Finite list}&\textbf{Space}}
{
Expression & $x$ & $X$ & $\mathbb{X}$\\
Element (of an expression) & $e$ & $E$ & $\mathbb{E}$\\
Function & $f$ & $F$ & $\mathbb{F}$\\
Clause & $c$ & $C$ & $\mathbb{C}$\\
Pattern & $p$ & $P$ & $\mathbb{P}$\\
Value & $b$ & $B$ & $\mathbb{B}$\\
Name & $v$ & $V$ & $\mathbb{V}$\\
Program & $r$ & $R$ & $\mathbb{R}$
}

%\begin{definition} The \nonterm{expression} at the end of \nonterm{program} can
%be considered as the main clause of a program, which we'll refer to as
%$c_{main}$.\end{definition}

\begin{definition} For any given $v\in\mathbb{V}$ and $P\subset\mathbb{P}$,
we say that $v\in P$ if $v$ occurs in some $p\in P$.\end{definition}

\begin{definition}\label{definition:clause-tuple} A clause $c\in\mathbb{C}$ is
a tuple $\left\langle v,P,x \right\rangle$, where $v\in\mathbb{V}$ is the name
of the clause, $P\subset\mathbb{P}$ is a non-empty list of patterns of the
clause, and $x\in\mathbb{X}$ is the expression of the clause. $P$ is ordered by
occurrence of the patterns in the program text.\end{definition}

\begin{definition} We say that a clause $c= \left\langle v,P,x \right\rangle$
``accepts'' an argument list $B$ iff $|P|=|B|$ and $\forall\ \{i\mid 0\geq i <
|P|\}\ b_i\in B \wedge p_i\in P \wedge b_i\succ p_i$.\end{definition}

\begin{definition}\label{definition:function-tuple} A function $f\in\mathbb{F}$
is a tuple $\left\langle v,C \right\rangle$, where $v \in \mathbb{V}$ is the
name of the function, and $C\subset\mathbb{C}$ is the non-empty list of clauses
of the function. It must hold for $C$ that $\forall\ c\in C\
\left(c=\left\langle v_c, P_c, x_c \right\rangle \wedge v_c=v\right)$ and
$\forall\ c_1,c_2\in C\ \left(c_1=\left\langle v_1, P_1, x_1 \right\rangle
\wedge c_2=\left\langle v_2, P_2, x_2 \right\rangle \wedge |P_1|=|P_2|\right)$.
$C$ is ordered by occurrence of the clauses in the program
text.\end{definition}

\begin{definition} A signature of some function $f=\left\langle v, C
\right\rangle$ is the tuple $\left\langle v,|P| \right\rangle$, s.t.
$\forall\ c\in C_f\ |P_c|=|P|$. We'll adopt the Erlang notation when talking
about function signatures, i.e. if we have a function \mono{less} that takes in
two parameters, we'll refer to it as \mono{less/2}.\end{definition}

We assume for it to be fairly simple to construct the set $F$ of a given
program $r$ given the set of clauses $C$ derived during syntactic analysis of
the program text.

\begin{definition} A program $r$ is a tuple $\left\langle F,x \right\rangle$,
where $F\subset\mathbb{F}$ is the list of functions defined in program $r$, and
$x$ is the expression of program $r$.\end{definition}

\begin{definition}\label{definition:function-call} A function call is a tuple
$\left\langle v, X \right\rangle$, where $v\in\mathbb{V}$ is the name of the
callee, and $X\subset\mathbb{X}$ is a non-empty list of arguments for the
function call, ordered by occurrence of the expressions in the program
text.\end{definition}

0-ary clauses are disallowed to avoid having to deal with constants in general.
The term $\term{\_}$ in $\nonterm{pattern-element}$ is the conventional
wildcard operator; it indicates a value that won't used in the clause
expression, but some value has to be there for an argument to match the
pattern. Furthermore, as will be clear from the semantics, multiple occurrences
of \term{\_} in a clause pattern list does not indicate that the same value has
to be in place for each \term{\_}. 

\begin{definition} When describing various values and patterns in definitions,
theorems, proofs, etc. we'll sometimes make use of $\_$ to denote parts of the
value or pattern that are irrelevant to the said definition, theorem, proof,
etc.\end{definition}

% TODO this should be clear from the semantics.

% Multiple wildcards in the parameter list indicate possibly different value
% arguments, while multiple occurances of the same variable name in the parameter
% list are disallowed.

\subsection{Patterns constitute shapes}

The nonterminal declarations for patterns, in particular \ref{nonterm-pattern}
and \ref{nonterm-pattern-element}, indicate that a pattern are equatable to
shapes.

\begin{definition}\label{definition:pattern-corresponds-shape} The pattern
\term{0} corresponds to the leaf shape. The patterns \term{\_} and
\nonterm{name} correspond to triangle shapes. Any pattern \mono{a.b}
corresponds to the shape $a\cdot b$ iff the pattern \mono{a} corresponds to the
shape $a$ and the pattern \mono{b} corresponds to the shape
$b$.\end{definition}

\begin{definition}\label{definition:pattern-is-shape} $\forall\ p\in\mathbb{P}\
\forall\ s\in\mathbb{S}\ p=s$ iff $p$ corresponds to $s$ as by
\referToDefinition{pattern-corresponds-shape}.\end{definition}

\begin{definition} We overload the binary relation $\succ$ with the set\\
$\left\{ \left\langle p_1,p_2 \right\rangle\mid p_1,p_2\in\mathbb{P},
s_1,s_2\in\mathbb{S} \wedge p_1=s_1 \wedge p_2=s_2
\right\}$.\end{definition}

\subsection{Unary functions from multivariate functions}

The patterns of a clause as well as the arguments of a function call get
special treatment in \D{} in that they according to
\referToDefinition{clause-tuple} and \referToDefinition{function-call} are
ordered by their occurrence in the program text. This order is important to
make sure that the appropriate argument is matched against the appropriate
pattern. 

While this is setup is practical for the programmer, it is of no use to us due
to \referToTheorem{multivariate-to-unary}. In latter discussions, this
particular theorem allows us to keep to unary functions, and regard the
extension to multivariate functions as a fairly simple matter.

\begin{theorem}\label{theorem:multivariate-to-unary} Any multivariate function
in \D{} can be represented with a unary function.\end{theorem}

\begin{proof}

Given a multivariate function $f= \left\langle v,C \right\rangle$:

\begin{enumerate}

\item For each clause $c\in C$, where $c=\left\langle v,P,x \right\rangle$,
replace the pattern list $P$ with $P'=\{p\}$. Construct $p$ by initially
letting $p=0$, and folding left-wise over $P$, performing $p=p\cdot p'$ for
each $p'\in P$. 

\item For each call $\left\langle v, X\right\rangle$ to function $f$, replace
$X$ with the set $X'=\{x\}$, where $x$ has been constructed in a manner
equivalent to the pattern $p$ above.

\end{enumerate}

It is easy to see that both the constructed patterns and expressions are indeed
valid patterns and expressions, and that $f$ hence becomes a unary
function.\end{proof}

As this transformation is relatively simple to perform, we redefine the generic
clause tuple to have but one pattern in place of a list. 

\begin{definition}\label{definition:unary-clause} We redefine the clause $c$ to
be the tuple $\left\langle v,p,x\right\rangle$, where $v\in\mathbb{V}$ is the
name of the clause, $p\in\mathbb{P}$ is the pattern of the clause, and
$x\in\mathbb{X}$ is the expression of the clause.\end{definition}

\begin{definition}\label{definition:unary-function-call} We redefine a function
call to be the tuple $\left\langle v,x\right\rangle$, where $v\in\mathbb{V}$ is
the name of the callee, and $x\in\mathbb{X}$ is the argument to the (always
unary) callee.\end{definition}
