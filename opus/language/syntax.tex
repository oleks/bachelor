\section{Syntax}\label{section:d-syntax}

We describe the syntax of \D{} in terms of an extended Backus-Naur
form\footnote{The extension lends some constructs from regular expressions to
achieve a more concise dialect. The extension is described in detail in
\referToAppendix{ebnf}.}. This is a core syntax definition, and other, more
practical, syntactical features may be defined later on as needed. The initial
non-terminal is $\nonterm{program}$.

\begin{align}
\nonterm{program}\ ::=&\ \nonterm{declaration}^*\ \nonterm{expression}\\
\nonterm{expression}\ ::=&\ \nonterm{element}\ (\ \term{.}\ \nonterm{expression}
\ )\ ?\\
\nonterm{element}\ ::=&\ \term{0}\ |\ \term{(}\ \nonterm{element}\ \term{)}\ |
\ \nonterm{name}\ |\ \nonterm{application}\\
\nonterm{application}\ ::=&\ \nonterm{name}
\ \nonterm{expression}^+\\
\nonterm{declaration}\ ::=&\ \nonterm{name}\ \nonterm{pattern}^+
\ \term{:=}\ \nonterm{expression}\\
\nonterm{pattern}\ ::=&\ \nonterm{pattern-value}\ (\ \term{.}
\ \nonterm{pattern}\ )\ ?\\
\nonterm{pattern-value}\ ::=&\ \term{0}\ |\ \term{\_}\ |\ \term{(}
\ \nonterm{pattern}\ \term{)}\ |\ \nonterm{name}\\
\nonterm{name}\ ::=&\ [\term{a}\mathmono{-}\term{z}]
\ \left (\ [\term{-}\ \term{a}\mathmono{-}\term{z}]^*
\ [\term{a}\mathmono{-}\term{z}]\ \right )?
\end{align}

0-ary declarations are disallowed to avoid having to deal with constants in
general.

The term $\term{\_}$ in $\nonterm{pattern-value}$ is the conventional wildcard
operator; it indicates a value that won't used by the declaration, but allows
us to keep the same declaration signature. We hence define the \emph{signature}
of a declaration as follows:

\begin{definition}

A declaration signature in \D{} consists of the function name and the number of
parameters it has.

\end{definition}

We'll adopt the Erlang-like notation when talking about function signatures,
i.e. if we have a function \mono{less} having two arguments in it's signature,
we'll refer to it as \mono{less/2}.

% TODO this should be clear from the semantics.

% Multiple wildcards in the parameter list indicate possibly different value
% arguments, while multiple occurances of the same variable name in the parameter
% list are disallowed.
