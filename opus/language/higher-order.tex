\subsection{Built-in high-order
functions}\label{section:language-higher-order-built-ins}

Although \mono{D} is initially a first-order language, we will ignore that
limitation for a bit and define a few higher-order functions to provide some
syntactical sugar to the language. Beyond the discussion in this section, these
higher-order functions should be regarded as \mono{D} built-ins.

\subsubsection{Branching}

In the following definition, the variable names \mono{true} and \mono{false}
refer to expressions to be executed in either case.

\begin{verbatim}
if 0 _ false := false
if _._ _ true := true
\end{verbatim}

As you can see, we employ the C convention that any value other than $0$ is a
``truthy'' value, and the expression \mono{true} is returned.

Although the call-by-value nature of the language does not allow for
short-circuiting the if-statements defined in such a way, this shouldn't be any
impediment to further analysis.

\subsection{Sample programs}\label{section:d-samples}

As an illustration of the language syntax, the following program reverses a tree:

\begin{verbatim}
reverse 0 := 0
reverse left.right := (reverse right).(reverse left)
\end{verbatim}

The following program computes the Fibonacci number \mono{n}:

Assume that the argument is

\begin{verbatim}
fibonacci n = fibonacci-aux (normalize n) 0 0

fibonacci-aux 0 x y := 0
fibonacci-aux 0.0 x y := y
fibonacci-aux 0.n x y := fibonacci-aux n y (add x y)
\end{verbatim}

\emph{Note:} The return value is not normalized.


