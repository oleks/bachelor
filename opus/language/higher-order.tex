\section{Built-in high-order
functions}\label{section:language-higher-order-built-ins}

Although \mono{D} is initially a first-order language, we will ignore that
limitation for a bit and define a few higher-order functions to provide some
syntactical sugar to the language. Beyond the discussion in this section, these
higher-order functions should be regarded as \mono{D} built-ins.

\subsubsection{Branching}

In the following definition, the variable names \mono{true} and \mono{false}
refer to expressions to be executed in either case.

\begin{verbatim}
if 0 _ false := false
if _._ _ true := true
\end{verbatim}

As you can see, we employ the C convention that any value other than $0$ is a
``truthy'' value, and the expression \mono{true} is returned.

Although the call-by-value nature of the language does not allow for
short-circuiting the if-statements defined in such a way, this shouldn't be any
impediment to further analysis.

\subsection{Boolean operations}

\begin{verbatim}
and _._ _._ = 0.0
and _ _ = 0

or 0 0 = 0
or _ _ = 0.0
\end{verbatim}

