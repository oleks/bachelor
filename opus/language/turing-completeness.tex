\section{Turing-completeness of \D{}}

We show that \D{} is Turing-complete by showing that we can simulate a
universal Turing machine\footnote{The notation and type of universal Turing
machine used here was inspired by
\url{http://plato.stanford.edu/entries/turing-machine/}.}.

We can describe any Turing machine in terms of its transition table, which is a
list of 4-tuples $\left\langle \lambda_0, \sigma, \lambda_1, \omega
\right\rangle : \Lambda \times \Sigma \times \Lambda \times \Omega$, where
$\Lambda$ is the finite set of states of the machine (some non-existent),
$\Sigma$ is the alphabet of the machine, typically $\{0,1\}$, and $\Omega$ is
the action table of the machine. Given that $\Lambda$, $\Sigma$ and $\Omega$
have definitions that are subsets of $\mathbb{B}$, any such list of tuples can
be represented as a \D{} value. 

\begin{definition} A list of values $L=\left[ b_i \mid b_i\in\mathbb{B}, 0 < i
\leq |L| \right]$ in \D{} is represented as the value $b_1\cdot b_2\cdots
b_n$.\end{definition}

This superimposes an indexing on the values $b_1,b_2,\ldots,b_n$ using
normalized \D{} values. In particular, the element $b_i$ where $0<i\leq|L|$ has
the same location in the binary tree $b_l$ as the left child of the bottom-most
node of the value $i\in\mathbb{B}$ in normalized form.

\begin{definition} Let $M'$ denote a list representing a Turing machine
transition table, where each tuple has the form $\left\langle \sigma, \lambda,
\omega \right\rangle$. Let $M\in\mathbb{B}$ be the \D{} representation of $M'$.
The set of possible states is hence $\left\{ n(i)\mid i\in\mathbb{B},
0<i\leq|M| \right\}\subset\Lambda$, where $n$ is the normalization
function from \referToListing{normalize}.\end{definition}

$\Lambda$ is a superset of the set of possible states to allow transitions to
undefined states, the effect of which is a halting of the machine.

\begin{definition}\label{definition:turing-alphabet} The possible symbols, in
the set $\Sigma$, are denoted as follows:

\begin{enumerate}

\item \mono{0.0}, denoting the value $0$.

\item \mono{0.0.0}, denoting the value $1$.

\end{enumerate}

\end{definition}

\begin{definition} The possible actions, in the set $\Omega$, are denoted as follows:

\begin{enumerate}

\item \mono{0.(0.0)}, denoting the action ``move right on the tape''.

\item \mono{(0.0).0}, denoting the action ``move left on the tape''.

\item \mono{0}, denoting the action ``write $0$ at current position on the
tape''.

\item \mono{0.0}, denoting the action ``write $1$ at the current position on
the tape''.

\end{enumerate}

\end{definition}

Let us assume that we have a function \mono{find/2} at our disposal, that takes
in a normalized value $i\in\mathbb{B}$ and a list $l\in\mathbb{B}$, and returns
either the element $l_i$, if $i$ is a valid index in the list $l$, and $0$
otherwise. Then \referToListing{turing-completeness} is a possible
implementation of a universal Turing machine in \D{}.

\begin{lstlisting}[label=listing:turing-completeness,
  caption={A universal Turing machine in \D{}. \mono{m} stands for $M$, \mono{s}
  stands for $\lambda$, \mono{t} stands for tape, and \mono{l} and \mono{r}
  stand for left and right, respectively.}]
utm _ 0 _           := 0 (*@ \label{line:tc-halt-1} @*)
utm m s 0           := utm s 0.(0.0).0 m
utm m s 0.0         := utm s 0.(0.0).0 m
utm m s 0.r         := utm s 0.r.0 m
urm m s l.0         := urm s 0.l.0 m
utm m s l.(0.0).r   := utm-interpret m l.v.r (find s m)
utm m s l.(0.0.0).r := utm-interpret m l.v.r (find s m)
utm m s l.v.r       := utm m s (l.v).r (*@\label{line:tc-tape-fix} @*)

utm-interpret _ _ 0 := 0 (*@ \label{line:tc-halt-2} @*)

utm-interpret m l.(0.0).r (0.0).s.0 := utm m s l.(0.0).r
utm-interpret m l.(0.0).r (0.0).s.(0.0) := utm m s l.(0.0.0).r

utm-interpret m l.(0.0).r (0.0).s.(0.(0.0)) := utm m s (l.(0.0)).r
utm-interpret m l.(0.0).r (0.0).s.((0.0).0) := utm m s l.((0.0).r)

utm-interpret m l.(0.0.0).r (0.0.0).s.0 := utm m s l.(0.0).r
utm-interpret m l.(0.0.0).r (0.0.0).s.(0.0) := utm m s l.(0.0.0).r

utm-interpret m l.(0.0.0).r (0.0.0).s.(0.(0.0)) := utm m s (l.(0.0.0)).r
utm-interpret m l.(0.0.0).r (0.0.0).s.((0.0).0) := utm m s l.((0.0.0).r)

utm-interpret m t _.s._ := utm m s t (*@ \label{line:tc-state-fix} @*)
\end{lstlisting}

Line \ref{line:tc-tape-fix} is there merely to make sure that if the initial
tape is somehow invalid, then it doesn't break the universal Turing machine.
Line \ref{line:tc-state-fix}, is there, in part, due to a similar reason, to
ensure that if an invalid transition table is given, this does not break the
universal Turing machine.

The function \mono{utm/3} takes in a transition table \mono{m}, initial state
\mono{s}, and initial tape \mono{t}, while the function \mono{utm-interpret/3}
takes in a transition table \mono{m} the tape \mono{t} and some tuple of the
form $\left\langle \sigma, \lambda, \omega \right\rangle : \Sigma \times
\Lambda \times \Omega$. Lines \ref{line:tc-halt-1} and \ref{line:tc-halt-2} are
in place to ensure that the universal Turing machine halts on any invalid
state.

The code should otherwise be fairly self-explanatory given the definitions
above, with perhaps the exception of how we handle the tape, that in this case,
expands indefinitely in both directions. In particular, the end of the tape is
marked with a \mono{0}, which should explain the somewhat unusual
\referToDefinition{turing-alphabet}.
