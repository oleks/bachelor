\section{Built-in functions}\label{section:language-higher-order-built-ins}

In the following section we define a few built-in \D{} functions. Some of them
will be defined in terms of \D{} itself.

\subsection{Input}

To be able to write more interesting programs, we'll define the primitive
function \mono{input/0} that can yield any valid \D{} value. This is the only
non-deterministic, 0-ary function in \D{}.

\subsection{Boolean operations}

\begin{definition} We'll adopt the C-convention of letting any non-zero value
represent a true value, and any zero value to represent a false
value.\end{definition}

Given the definition above, we define the often useful functions \mono{and/2},
\mono{or/2} and \mono{not/1} in \referToListing{and}, \referToListing{or} and
\referToListing{not}, respectively. Note, that due to \D{} being a
call-by-value language, \mono{or/2} is \emph{not} shortcircuited as
conventionally is the case.

\begin{lstlisting}[label=listing:and,
  caption={The function \mono{and/2}.}]
and _._ _._ := 0.0
and _ _ = 0
\end{lstlisting}

\begin{lstlisting}[label=listing:or,
  caption={The function \mono{or/2}.}]
or 0 0 := 0
or _ _ := 0.0
\end{lstlisting}

\begin{lstlisting}[label=listing:not,
  caption={The function \mono{not/1}.}]
not 0 := 0.0
not _._ := 0
\end{lstlisting}

\subsection{Comparison}

There are many imaginable programs that rely on some value being less, more, or
equal to some other value. Since no such primitives are available we have to
define such comparisons ourselves. However, there is seemingly no
\emph{elegant} way of figuring out how the sizes of two arbitrary values in
\D{} differ, using \D{} itself. For this purpose we define the concept of a
\emph{normalized} value.

\begin{definition}\label{definition:normal-form} A binary tree in normalized
form is a binary tree that either is a leaf, or a node having a leaf as its
left child and a binary tree in standard representation as its right
child.\end{definition}

Visually, a binary tree in standard representation is just a tree that only
descends along the right-hand side. Refer to \referToFigure{normalize-example}
for an example of a value in \D{} and its normalized form.

\includeFigure{normalize-example}{A sample value $b\in\mathbb{B}$ to the left,
and its normalized form to the right.}

\referToDefinition{normal-form} allows us to define the function
\mono{normalize/1} that normalizes a value. We've done this
\referToListing{normalize}.

\begin{lstlisting}[label=listing:normalize,
  caption={The function \mono{normalize/1} turns any value $b\in\mathbb{B}$ into its normal form.}]
normalize a = normalize-aux a 0 0

normalize-aux 0     0     an := an
normalize-aux 0     bl.br an := normalize-aux bl br    an
normalize-aux 0.ar  b     an := normalize-aux ar b     0.an
normalize-aux al.0  b     an := normalize-aux al b     0.an
normalize-aux al.ar b     an := normalize-aux ar al.b  0.an
\end{lstlisting}

Comparing the sizes of two trees in this representation is just a matter of
walking down two normalized trees simultaneously, until one of them, or both,
bottoms out. If there is a tree that bottoms out strictly before another, that
is the lesser value by \referToDefinition{size}. This allows us to define the
functions \mono{less/2} and \mono{equal-size/2}\footnote{We add the
\mono{-size} suffix in order to reserve the name \mono{equal} for a function
that compares two values in \D{} by their actual tree structure rather than the
number of nodes.} which we do in \referToListing{less} and
\referToListing{equal-size}, respectively.

\begin{lstlisting}[label=listing:less,
  caption={The function \mono{less/2} yields true if the first argument 
    is less than the second and false otherwise.}]
less a b := normalized-less (normalize a) (normalize b)

normalized-less 0 b := b
normalized-less _ 0 := 0
normalized-less _.a _.b := normalized-less a b
\end{lstlisting}

\begin{lstlisting}[label=listing:equal-size,
  caption={The function \mono{equal-size/2} function.}]
equal-size a b := normalized-equal-size (normalize a) (normalize b)

normalized-equal-size 0 0 := 0.0
normalized-equal-size _ 0 := 0
normalized-equal-size 0 _ := 0
normalized-equal-size _.a _.b := normalized-equal-size a b
\end{lstlisting}

\subsection{Increase \& decrease}

As with comparison, there are many imaginable programs that increase or
decrease values. An increase in the number of nodes is trivial, as shown by the
function \mono{increase/1} in \referToListing{increase}. 

\begin{lstlisting}[label=listing:increase,
  caption={The function \mono{increase/1} increases a value by 1.}]
increase a := 0.a
\end{lstlisting}

A decrease of a value on the other hand, requires normalization of the value
and a right-wise walk down the tree until the bottom-most node is reached,
after which the node is removed.  What's more, \D{} has no overflow and no
negative values, so we must take care of what we do with the value $0$, which
hence cannot be decreased. We decide to let \mono{decrease 0} yield \mono{0}.
All this is summarized in \referToListing{decrease}.

\begin{lstlisting}[label=listing:decrease,
  caption={The function \mono{decrease/1} decreases a value 1,
    unless that value is 0, in which case nothing is done.}]
decrease 0 := 0
decrease a := normalized-decrease (normalize a)

normalized-decrease 0.0 := 0
normalized-decrease a.b := a.(normalized-decrease b)
\end{lstlisting}

