\newcommand{\D}{$\Delta$}
\chapter{The language \D{}}

The goal of this work is to describe a few automated termination analysis
techniques, and in particular, size-change termination. In order to allow for
the following chapters to retain a modest level of abstraction to the Turing
machine, such that the techniques are described for an environment that is
modestly applicable to solving moderate programming problems, a Turing complete
language \D{} is introduced.

\section{The goal}

The intent of the language is two-fold, \begin{inparaenum}[(1)]\item aid the
descriptions of automated termination analysis techniques in latter
chapters, and \item be relatively expressive\end{inparaenum}.

Expressiveness of a language is a rather subjective and domain-driven concept.
First and foremost, expressiveness depends on the initial intended domain of
the language. Of course, Turing complete languages are known to be universally
applicable, however, some languages are just more fine tuned to solving some
problems, while others are better tuned to solving other problems.

\D{} is a language with very few primitive operations but is expressive enough
to write the Fibonacci and Ackermann functions in an elegant way. To do this,
\D{} borrows some syntax and semantics from purely functional languages such as
ML or Haskell. Hence, programs in \D{} make heavy use of pattern matching and
recursion to achieve branching and looping, some of the constructs required for
a language to be Turing complete.

Unlike ML and Haskell, \D{} is a language that completely disregards the
concepts of abstract data structures and types. Hence, many data driven
programs will be hard to write in \D{}. Of course, this is not to say that data
flow analysis is irrelevant to termination analysis as such, on the contrary,
it is key to size-change termination. It is because of this prime importance of
data flow to termination analysis, that data value representation is kept to
its almost lowest possible denominator. This keeps the analysis clean of rather
irrelevant abstract data structure fiddling. What's more, any methods developed
for \D{} can be extended and used in a language with types and abstract data
structures, as long as it is computationally equivalent to \D{}.

Also, unlike most purely functional languages, \D{} is a first-order,
call-by-value language. This is done in part to adhere to the general flow of
\cite{size-change}, and in part to keep the analysis simple at first.
Higher-order constructs impose difficulties when deducing changes in size, and
evaluation strategies other than call-by-value impose a similar sort of
difficulties.

\section{Data}

\D{} is a simple language where the emphasis is on the sizes of data. Hence,
the way that data values are constructed does not have to be particularly
practical, but all values have to be well founded and easily comparable.

The language \D{} is untyped, and represents all data in terms of
\emph{unlabeled ordered binary trees}, henceforth referred to as simply,
\emph{binary trees}. Such a tree is recursively defined as follows:

\begin{definition}

A binary tree is a set that is either empty, heneceforth refferred to as leaf,
or contains a single unlabeled node with two binary trees as it's left and
right child, henceforth simply reffered to as node. We'll refer to the set of
all possible values in \D{} as $\mathbb{B}$.

\end{definition}

To operate on such trees we'll require a few primitives, namely a
representation of leafs, recursive construction and destruction of nodes, as
well as a way to tell nodes and leafs apart. Most of these will be derived in
the operational semantics of \D{}\footnote{See \referToSection{d-sos}.},
however, we do require the following basic definitions:

\begin{definition}

Let the atom $0$ represent a leaf binary tree.

\end{definition}

\begin{definition}

The function $\cdot
:\mathbb{B}\times\mathbb{B}\rightarrow\mathbb{B}$ constructs a node with the
two arguments as it's left and right child, respectively. 

\end{definition}


\section{Syntax}\label{section:d-syntax}

We describe the syntax of \D in terms of an extended Backus-Naur
form\footnote{The extension lends some constructs from regular expressions to
achieve a more concise dialect. The extension is described in detail in
\referToAppendix{ebnf}.}. This is a core syntax definition, and other, more
practical, syntactical features may be defined later on as needed.

\begin{align}
\nonterm{expression}\ ::=&\ \nonterm{value}\ (\ \term{.}\ \nonterm{expression}
\ )\ ?\\
\nonterm{value}\ ::=&\ \term{0}\ |\ \term{(}\ \nonterm{value}\ \term{)}\ |
\ \nonterm{application}\\
\nonterm{application}\ ::=&\ \nonterm{name}
\ \nonterm{expression}^*\\
\nonterm{function}\ ::=&\ \nonterm{name}\ \nonterm{pattern}^*
\ \term{:=}\ \nonterm{expression}\\
\nonterm{pattern}\ ::=&\ \nonterm{pattern-value}\ (\ \term{.}
\ \nonterm{pattern}\ )\ ?\\
\nonterm{pattern-value}\ ::=&\ \term{0}\ |\ \term{\_}\ |\ \term{(}
\ \nonterm{pattern}\ \term{)}\ |\ \nonterm{name}\\
\nonterm{name}\ ::=&\ [\term{a}\mathmono{-}\term{z}]
\ \left (\ [\term{-}\ \term{a}\mathmono{-}\term{z}]^*
\ [\term{a}\mathmono{-}\term{z}]\ \right )?
\end{align}

The term $\term{\_}$ in $\nonterm{pattern-value}$ is the conventional wildcard
operator; it indicates a value that won't used by the function declaration, but
allows us to keep the same function signature. We define the \emph{signature}
of a function as follows:

\begin{definition}

A function signature in D consists of the function name and the number of
parameters it has.

\end{definition}

% TODO this should be clear from the semantics.

% Multiple wildcards in the parameter list indicate possibly different value
% arguments, while multiple occurances of the same variable name in the parameter
% list are disallowed.

\section{Semantics}\label{section:d-sos}

% Revise the context of an expression within a function call, it should always be
% the context upon entering the function call! Or even better, the context when
% the function was defined!

% \textbf{Allow mutual recursion}

% \textbf{Perhaps pattern matching must be exhaustive in general.}

% \textbf{Every subsequent definition must be strictly less specific than the former.}

In the following section we describe the semantics of \D{} using a form of
structured operational semantics. The syntax used to define the reduction rules
is largely equivalent to the Aarhus report\cite{sos}, but differs
slightly\footnote{The syntax applied here is described in further detail in
\referToAppendix{sos}.}.

\subsection{The memory model}\label{section:d-semantics-memory}

\begin{definition}\label{definition:memory} Memory is considered in terms of a
binary relation $\sigma$ which for any given clause $c$ is the set $\{(n,b)\mid
n\in\mathbb{N} \wedge b\in\mathbb{B} \wedge n\in P_c\}$. For any given
$(n,b)\in\sigma_c$, we say that in the scope of $c$, the variable $n$, is bound
to $b$.\end{definition}

\begin{corollary}\label{corollary:init-empty-scope} Variables are bound when
arguments are matched to clause patterns, and if the argument matches a
pattern, hence before pattern matching the arguments for any given clause $c$,
$\sigma_c=\emptyset$.\end{corollary}

\referToDefinition{memory} and \referToCorollary{init-empty-scope} indicate
that \D{} is statically scoped.  Furthermore, \referToDefinition{memory} may
prove hampering if we were ever to extend \D{} with lambda calculus, but there
are initially no plans to do so.

\begin{definition} The \nonterm{expression} at the end of \nonterm{program} can
be considered as the main clause of a program, which we'll refer to as
$c_{main}$.\end{definition}

\begin{corollary} $\sigma_{c_{main}}=\emptyset$. \end{corollary}

% This renders $\sigma$ countably infinite since $\mathbb{V}$ is countably
% infinite.

\subsubsection{Functions and variables}

Due to \D{} being a first-order language, we should make sure to separate the
function and variable spaces. We'll represent these by $\phi$ and $\gamma$,
respectively.

Whenever we use $\sigma$, $\phi$ or $\gamma$ in set notation, we imply the sets
of the names of functions and variables, and not the stacks themselves
corresponding to those names.  Hence, $\sigma=\phi\cup\gamma$, and to keep \D{}
first-order we add the limitation that $\phi\cap\gamma=\emptyset$.

\subsubsection{Making \D{} higher order}

The only change that this would require is to let $\phi=\gamma=\sigma$.

\subsection{Function declarations}

Assuming that as a part of the semantic analysis all $\nonterm{declaration}$ with the same name are grouped into the set $\left\langle n F\right\rangle$

A declaration with a name $n$, a \emph{non-empty} pattern
list $P$ and an expression $e$ is stored in the function space $\phi$:

\begin{equation}\label{sem:declaration}
{\displaystyle
  \left\langle \phi(n)\mapsto \left\langle P, x, \phi\right\rangle\right\rangle
  \rightarrow
  \phi'
\over\displaystyle
  \left\langle n, P, x, \phi\right\rangle
  \rightarrow
  \phi'
}
\end{equation}

\subsection{Expression evaluation}

An expression $x$ is either the element $e$, or a construction of an element
$e_1$ with another expression $x_1$. That is, the binary infix operator $\cdot$
is right-associative, and has the following operational semantics:

\everymath{\displaystyle}

\begin{equation}
{\displaystyle
  \left\langle \proc{Single}, x,\sigma\right\rangle
  \rightarrow
  \left\langle v,\sigma\right\rangle
\vee
  \left\langle \proc{Chain}, x,\sigma\right\rangle
  \rightarrow
  \left\langle v,\sigma\right\rangle
\over\displaystyle
  \left\langle x,\sigma\right\rangle
  \rightarrow
  \left\langle v,\sigma\right\rangle
}
\end{equation}

\begin{equation}
{\displaystyle
  x\rightarrow e
\wedge
  \left\langle e,\sigma\right\rangle
  \rightarrow
  \left\langle v,\sigma\right\rangle
\over\displaystyle
  \left\langle \proc{Single}, x,\sigma\right\rangle
  \rightarrow
  \left\langle v,\sigma\right\rangle
}
\end{equation}

\begin{equation}
{\displaystyle
  x\Rightarrow e_1\cdot x_1
\wedge
  \left\langle e_1,\sigma\right\rangle
  \rightarrow
  \left\langle v_1,\sigma\right\rangle
\wedge
  \left\langle x_1,\sigma\right\rangle
  \rightarrow
  \left\langle v_2,\sigma\right\rangle
\over\displaystyle
  \left\langle \proc{Chain}, x, \sigma\right\rangle
  \rightarrow
  \left\langle v, \sigma\right\rangle
}
\quad(\text{where }v_1\cdot v_2=v)
\end{equation}

\subsection{Element evaluation}

According to the syntax specification, an element of an expression can either
be the atom $0$, or an application. We'd like to distinguish between variables
and functions, and we do that  

\begin{equation}
{\displaystyle
\left(
    e\Rightarrow 0
  \wedge
    v\equiv 0
\right)
\vee
{\displaystyle
    e\Rightarrow n
\over\displaystyle
    \beta(n)\Rightarrow v
}
\vee
{\displaystyle
    e\Rightarrow \left\langle n, X\right\rangle
\over\displaystyle
    \left\langle n,X,\sigma\right\rangle
    \Rightarrow
    \left\langle v,\sigma\right\rangle
}
\over\displaystyle
\left\langle e,\sigma\right\rangle
\Rightarrow
\left\langle v,\sigma\right\rangle
}
\end{equation}

\subsection{Function application}

\begin{equation}
{\displaystyle
{\displaystyle
{\displaystyle
  \left\langle n, \phi\right\rangle
  \Rightarrow
  \left\langle P, x, \phi\right\rangle
\over\displaystyle
  \left\langle P, X, \sigma\right\rangle
  \Rightarrow
  \sigma'
}
\over\displaystyle
  \left\langle x, \sigma'\right\rangle
  \Rightarrow
  \left\langle v,\sigma'\right\rangle
}
\over\displaystyle
    \left\langle n,X,\sigma\right\rangle
    \Rightarrow
    \left\langle v,\sigma\right\rangle
}
\end{equation}

\subsection{Pattern matching}

\begin{equation}
{\displaystyle
{\displaystyle
  \left\langle P_{head}, X_{head}, \sigma\right\rangle
  \Rightarrow
  \sigma''
\over\displaystyle
  \left\langle P_{tail}, X_{tail}, \sigma''\right\rangle
  \Rightarrow
  \sigma'
}
\over\displaystyle
  \left\langle P, X, \sigma\right\rangle
  \Rightarrow
  \sigma'
}
\end{equation}

\begin{equation}
{
  \left\langle\proc{I}, p,x,\sigma\right\rangle
  \Rightarrow
  \left\langle p',x',\sigma'\right\rangle
\vee
  \left\langle\proc{Z}, p,x,\sigma\right\rangle
  \Rightarrow
  \left\langle p',x',\sigma'\right\rangle
\vee
  \left\langle\proc{N}, p,x,\sigma\right\rangle
  \Rightarrow
  \left\langle p',x',\sigma'\right\rangle
\vee
  \left\langle\proc{P}, p,x,\sigma\right\rangle
  \Rightarrow
  \left\langle p',x',\sigma'\right\rangle
}\over{
  \left\langle p, x, \sigma\right\rangle
  \Rightarrow
  \left\langle p', x', \sigma'\right\rangle
}
\end{equation}

For the sake of an elegant notation, we'll override the function $\cdot$ for
patterns.

\begin{definition}

A pattern is an unlabeled of binary tree which is either empty or consists of
an unlabeled node with a $0$, $\_$, name, or a pattern as it's left and right
child. 

\end{definition}

\begin{definition}

Let the set of all possible patterns be denoted by $\mathbb{P}$.

\end{definition}

\begin{definition}

The function $\cdot
:\mathbb{P}\times\mathbb{P}\rightarrow\mathbb{P}$ constructs a pattern node with the
two arguments as it's left and right child, respectively. 

\end{definition}

\begin{equation}
{
    p\Rightarrow \_\cdot p'
  \wedge
    x\Rightarrow e\cdot x'
  \wedge
    \sigma\Rightarrow\sigma'
}\over{
  \left\langle\proc{Underscore}, p,x,\sigma\right\rangle
  \Rightarrow
  \left\langle p',x',\sigma'\right\rangle
}
\end{equation}

\begin{equation}
{
    p\Rightarrow 0\cdot p'
  \wedge
    x\Rightarrow e\cdot x'
  \wedge
    \sigma\Rightarrow\sigma'
}\over{
  \left\langle\proc{Zero}, p,x,\sigma\right\rangle
  \Rightarrow
  \left\langle p',x',\sigma'\right\rangle
}
\end{equation}


\begin{equation}
{\displaystyle
{\displaystyle
{\displaystyle
    p\Rightarrow n\cdot p'
  \wedge
    x\Rightarrow e\cdot x'
\over\displaystyle
    \left\langle e,\sigma\right\rangle
    \Rightarrow
    \left\langle v,\sigma\right\rangle
}\over\displaystyle
    \left\langle\sigma(n)\leftarrow v\right\rangle
    \Rightarrow
    \sigma'
}\over\displaystyle
  \left\langle\proc{Name}, p,x,\sigma\right\rangle
  \Rightarrow
  \left\langle p',x',\sigma'\right\rangle
}
\end{equation}

\begin{equation}
{\displaystyle
{\displaystyle
    p\Rightarrow p''\cdot p'  
  \wedge
    x\Rightarrow x''\cdot x'
\over\displaystyle
  \left\langle p'', x'', \sigma\right\rangle
  \Rightarrow
  \sigma'
}
\over\displaystyle
  \left\langle\proc{Pattern}, p,x,\sigma\right\rangle
  \Rightarrow
  \left\langle p',x',\sigma'\right\rangle
}
\end{equation}

\subsection{Deducing unary functions from multivariate functions}

When performing termination analysis of programs it may prove tedious to
consider a list of patterns rather than a single pattern, especially since a
multivariate function can always be encoded as a unary function with e.g. the
patterns \mono{a}, \mono{b}, \mono{c}, and d encoded as \mono{a.b.c.d}. The
same ``encoding'' would have to be applied to each call to the function as
well. Hence, we can limit ourselves to termination proofs of unary functions.

\section{Built-in functions}\label{section:language-higher-order-built-ins}

In the following section we define a few built-in \D{} functions. Some of them
will be defined in terms of \D{} itself.

\subsection{Input}

To be able to write more interesting programs, we'll define the primitive
function \mono{input/0} that can yield any valid \D{} value. This is the only
non-deterministic, 0-ary function in \D{}.

\subsection{Boolean operations}

\begin{definition} We'll adopt the C-convention of letting any non-zero value
represent a true value, and any zero value to represent a false
value.\end{definition}

Given the definition above, we define the often useful functions \mono{and/2},
\mono{or/2} and \mono{not/1} in \referToListing{and}, \referToListing{or} and
\referToListing{not}, respectively. Note, that due to \D{} being a
call-by-value language, \mono{or/2} is \emph{not} shortcircuited as
conventionally is the case.

\begin{lstlisting}[label=listing:and,
  caption={The function \mono{and/2}.}]
and _._ _._ := 0.0
and _ _ = 0
\end{lstlisting}

\begin{lstlisting}[label=listing:or,
  caption={The function \mono{or/2}.}]
or 0 0 := 0
or _ _ := 0.0
\end{lstlisting}

\begin{lstlisting}[label=listing:not,
  caption={The function \mono{not/1}.}]
not 0 := 0.0
not _._ := 0
\end{lstlisting}

\subsection{Comparison}

There are many imaginable programs that rely on some value being less, more, or
equal to some other value. Since no such primitives are available we have to
define such comparisons ourselves. However, there is seemingly no
\emph{elegant} way of figuring out how the sizes of two arbitrary values in
\D{} differ, using \D{} itself. For this purpose we define the concept of a
\emph{normalized} value.

\begin{definition}\label{definition:normal-form} A binary tree in normalized
form is a binary tree that either is a leaf, or a node having a leaf as it's
left child and a binary tree in standard representation as it's right
child.\end{definition}

Visually, a binary tree in standard representation is just a tree that only
descends along the right-hand side. Refer to \referToFigure{normalize-example}
for an example of a value in \D{} and its normalized form.

\includeFigure{normalize-example}{A sample value $b\in\mathbb{B}$ to the left,
and its normalized form to the right.}

\referToDefinition{normal-form} allows us to define the function
\mono{normalize/1} that normalizes a value. We've done this
\referToListing{normalize}.

\begin{lstlisting}[label=listing:normalize,
  caption={The function \mono{normalize/1} turns any value $b\in\mathbb{B}$ into its normal form.}]
normalize a = normalize-aux a 0 0

normalize-aux 0     0     an := an
normalize-aux 0     bl.br an := normalize-aux bl br    an
normalize-aux 0.ar  b     an := normalize-aux ar b     0.an
normalize-aux al.0  b     an := normalize-aux al b     0.an
normalize-aux al.ar b     an := normalize-aux ar al.b  0.an
\end{lstlisting}

Comparing the sizes of two trees in this representation is just a matter of
walking down two normalized trees simultaneously, until one of them, or both,
bottoms out. If there is a tree that bottoms out strictly before another, that
is the lesser value by \referToDefinition{size}. This allows us to define the
functions \mono{less/2} and \mono{equal-size/2}\footnote{We add the
\mono{-size} suffix in order to reserve the name \mono{equal} for a function
that compares two values in \D{} by their actual tree structure rather than the
number of nodes.} which we do in \referToListing{less} and
\referToListing{equal-size}, respectively.

\begin{lstlisting}[label=listing:less,
  caption={The function \mono{less/2} yields true if the first argument 
    is less than the second and false otherwise.}]
less a b := normalized-less (normalize a) (normalize b)

normalized-less 0 b := b
normalized-less _ 0 := 0
normalized-less _.a _.b := normalized-less a b
\end{lstlisting}

\begin{lstlisting}[label=listing:equal-size,
  caption={The function \mono{equal-size/2} function.}]
equal-size a b := normalized-equal-size (normalize a) (normalize b)

normalized-equal-size 0 0 := 0.0
normalized-equal-size _ 0 := 0
normalized-equal-size 0 _ := 0
normalized-equal-size _.a _.b := normalized-equal-size a b
\end{lstlisting}

\subsection{Increase \& decrease}

As with comparison, there are many imaginable programs that increase or
decrease values. An increase in the number of nodes is trivial, as shown by the
function \mono{increase/1} in \referToListing{increase}. 

\begin{lstlisting}[label=listing:increase,
  caption={The function \mono{increase/1} increases a value by 1.}]
increase a := 0.a
\end{lstlisting}

A decrease of a value on the other hand, requires normalization of the value
and a right-wise walk down the tree until the bottom-most node is reached,
after which the node is removed.  What's more, \D{} has no overflow and no
negative values, so we must take care of what we do with the value $0$, which
hence cannot be decreased. We decide to let \mono{decrease 0} yield \mono{0}.
All this is summarized in \referToListing{decrease}.

\begin{lstlisting}[label=listing:decrease,
  caption={The function \mono{decrease/1} decreases a value 1,
    unless that value is 0, in which case nothing is done.}]
decrease 0 := 0
decrease a := normalized-decrease (normalize a)

normalized-decrease 0.0 := 0
normalized-decrease a.b := a.(normalized-decrease b)
\end{lstlisting}


\section{Sample programs}\label{section:d-samples}

As an illustration of the language syntax, take a look at the programs in
\referToListing{reverse}, \referToListing{fibonacci} and
\referToListing{ackermann}.

\begin{lstlisting}[label=listing:reverse,
  caption={A program that reverses the order of the nodes of some supplied tree.}]
reverse 0 := 0
reverse left.right := (reverse right).(reverse left)

reverse input
\end{lstlisting}

\begin{lstlisting}[label=listing:fibonacci,
  caption={A program that computes the $n^{th}$ fibonacci when supplied with some $n$.}] 
fibonacci n = fibonacci-aux (normalize n) 0 0

fibonacci-aux 0 x y := 0
fibonacci-aux 0.0 x y := y
fibonacci-aux 0.n x y := fibonacci-aux n y (add x y)

fibonacci input
\end{lstlisting}

\begin{lstlisting}[label=listing:ackermann,
  caption={The Ackermann-P\'eter function.}]
ackermann 0 n := 0.n
ackermann a.b 0 := ackermann (decrease a.b) 1
ackermann a.b c.d := ackermann (decrease a.b) (ackermann a.b (decrease c.d))

ackermann input input
\end{lstlisting}

\section{Turing-completeness of \D{}}

We prove that \D{} is Turing complete by showing that any Turing machine can be
written in \D{}.

An arbitrary Turing machine can be described merely in terms of its transition
table, which in \D{} can be expressed in terms of multiple clauses of a single
function definition.

We can describe a Turing machine in terms of a list of 4-tuples $\left\langle
\lambda_0, \sigma, \lambda_1, \omega \right\rangle : \Lambda \times \Sigma
\times \Lambda \times \Omega$, where $\Lambda$ is the finite set of states of
the machine, $\Sigma$ is the alphabet of the machine, typically $\{0,1\}$, and
$\Omega$ is the action table of the machine.

