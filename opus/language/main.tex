\newcommand{\D}{$\Delta$}
\chapter{The language \D{}}

The goal of this work is to describe a few automated termination analysis
techniques, and in particular, size-change termination. In order to allow for
the following chapters to retain a modest level of abstraction to the Turing
machine, such that the techniques are described for an environment that is
modestly applicable to solving moderate programming problems, a Turing complete
language \D{} is introduced.

\section{The intent of the language}

The intent of the language is two-fold, \begin{inparaenum}[(1)]\item aid the
descriptions of automated termination analysis techniques in latter
chapters, and \item be relatively expressive\end{inparaenum}.

Expressiveness of a language is a rather subjective and domain-driven concept.
First and foremost, expressiveness depends on the initial intended domain of
the language. Of course, Turing complete languages are known to be universally
applicable, however, some languages are just more fine tuned to solving some
problems, while others are better tuned to solving other problems.

\D{} is a language with very few primitive operations but is expressive enough
to write the Fibonacci and Ackermann functions in an elegant way. To do this,
\D{} borrows some syntax and semantics from purely functional languages such as
ML or Haskell. Hence, programs in \D{} make heavy use of pattern matching and
recursion to achieve branching and looping, some of the constructs required for
a language to be Turing complete.

Unlike ML and Haskell, \D{} is a language that completely disregards the
concepts of abstract data structures and types. Hence, many data driven
programs will be hard to write in \D{}. Of course, this is not to say that data
flow analysis is irrelevant to termination analysis as such, on the contrary,
it is key to size-change termination. It is because of this prime importance of
data flow to termination analysis, that data value representation is kept to
its almost lowest possible denominator. This keeps the analysis clean of rather
irrelevant abstract data structure fiddling. What's more, any methods developed
for \D{} can be extended and used in a language with types and abstract data
structures, as long as it is computationally equivalent to \D{}.

Also, unlike most purely functional languages, \D{} is a first-order,
call-by-value language. This is done in part to adhere to the general flow of
\cite{size-change}, and in part to keep the analysis simple at first.
Higher-order constructs impose difficulties when deducing changes in size, and
evaluation strategies other than call-by-value impose a similar sort of
difficulties.

\begin{frame}

\frametitle{\D{}, values and shapes}

\begin{columns}

\column{0.3\textwidth}
\begin{center}
$$b\in\mathbb{B}$$
\includegraphics[scale=0.5]{figures/value}
\end{center}

\column{0.3\textwidth}

\begin{center}
{\fontsize{40}{20}$\succ$}
\end{center}

\column{0.3\textwidth}

\begin{center}
$$s\in\mathbb{S}$$
\includegraphics[scale=0.5]{figures/shape}
\end{center}

\end{columns}

\end{frame}

\begin{frame}

\frametitle{\D{}, shapes and shapes}

\begin{columns}

\column{0.3\textwidth}
\begin{center}
$$s_1\in\mathbb{S}$$
\includegraphics[scale=0.5]{figures/shape}
\end{center}

\column{0.3\textwidth}

\begin{center}
{\fontsize{40}{20}$\succ$}
\end{center}

\column{0.3\textwidth}

\begin{center}
$$s_2\in\mathbb{S}$$
\includegraphics[scale=0.5]{figures/shape-2}
\end{center}

\end{columns}

\end{frame}

\begin{frame}

\frametitle{Disjoint shapes}

\begin{center}

$$s_1\cap s_2=\emptyset\quad\text{iff}\quad B_1\cap B_2=\emptyset$$

where

$$s_1,s_2\in\mathbb{S} \wedge B_1=\{b\mid b\in\mathbb{B} \wedge b\succ
s_1\}\wedge B_2=\{b\mid b\in\mathbb{B} \wedge b\succ s_2\}$$

\end{center}

\end{frame}

% TODO: All patterns now have a name.

\begin{frame}

\begin{center}

Given a shape $s_i\in\mathbb{S}$, we define the {\bf sibling set} $S^d_i$, to
be the pairwise disjoint set of shapes disjoint with $s_i$.

\end{center}

\end{frame}

\begin{frame}[fragile]

\begin{textblock}{1}(13.8,-3.5)$(35)$\end{textblock}

\begin{lstlisting}
data Pattern
  = PNil
  | PVariable String
  | PNode Pattern Pattern

getSiblings :: Pattern -> [Pattern]

getSiblings PNil =
  [PNode (PVariable "_") (PVariable "_")]

getSiblings (PVariable _) = []

getSiblings (PNode leftP rightP) =
  let
    leftS = getSiblings leftP
    rightS = getSiblings rightP
    leftInit = map (\s -> PNode leftP s) rightS
    rightInit = map (\s -> PNode s rightP) leftS
  in
    [PNil] ++
      leftInit ++ rightInit ++
      interleaveSiblings name leftS rightS
\end{lstlisting}

\end{frame}

\begin{frame}

\begin{align*}
T(1)&=4\\
T(n)&=1+T(n-1)+T(n-1)+T(n-1)\cdot T(n-1)
\end{align*}

\end{frame}

\section{Syntax}\label{section:d-syntax}

We describe the syntax of \D{} in terms of an extended Backus-Naur
form\footnote{The extension lends some constructs from regular expressions to
achieve a more concise dialect. The extension is described in detail in
\referToAppendix{ebnf}.}. This is a core syntax definition, and other, more
practical, syntactical features may be defined later on as needed. The initial
non-terminal is $\nonterm{program}$.

\begin{align}
\nonterm{program}\ ::=&\ \nonterm{clause}^*\ \nonterm{expression}\\
\nonterm{expression}\ ::=&\ \nonterm{element}\ (\ \term{.}\ \nonterm{expression}
\ )\ ?\\
\nonterm{element}\ ::=&\ \term{0}\ |\ \term{(}\ \nonterm{element}\ \term{)}\ |
\ \nonterm{name}\ |\ \nonterm{application}\\
\nonterm{application}\ ::=&\ \nonterm{name}
\ \nonterm{expression}^+\\
\nonterm{clause}\ ::=&\ \nonterm{name}\ \nonterm{pattern}^+
\ \term{:=}\ \nonterm{expression}\\
\label{nonterm-pattern}
\nonterm{pattern}\ ::=&\ \nonterm{pattern-element}\ (\ \term{.}
\ \nonterm{pattern}\ )\ ?\\
\label{nonterm-pattern-element}
\nonterm{pattern-element}\ ::=&\ \term{0}\ |\ \term{\_}\ |\ \term{(}
\ \nonterm{pattern}\ \term{)}\ |\ \nonterm{name}\\
\nonterm{name}\ ::=&\ [\term{a}\mathmono{-}\term{z}]
\ \left (\ [\term{-}\ \term{a}\mathmono{-}\term{z}]^*
\ [\term{a}\mathmono{-}\term{z}]\ \right )?
\end{align}

\begin{definition} \referToTable{sos-definitions} defines shorthands for
various language constructs. We'll often refer to these in further discussions.
Additionally, we'll let the atoms $0$ and $\_$ represent
themselves.\end{definition} 

\makeTable[h!]
{sos-definitions}
{Shorthands for various language constructs for use in latter discussions. We
provide shorthands for an instance, a list, and the space of a construct. For
instance, $x$ is some particular expression, $X$ is some particular list of
expressions, and $\mathbb{X}$ is the set of all possible expressions.}
{|l|c|c|c|}
{\textbf{Description}&\textbf{Instance}&\textbf{Finite list}&\textbf{Space}}
{
Expression & $x$ & $X$ & $\mathbb{X}$\\
Element (of an expression) & $e$ & $E$ & $\mathbb{E}$\\
Function & $f$ & $F$ & $\mathbb{F}$\\
Clause & $c$ & $C$ & $\mathbb{C}$\\
Pattern & $p$ & $P$ & $\mathbb{P}$\\
Value & $b$ & $B$ & $\mathbb{B}$\\
Name & $v$ & $V$ & $\mathbb{V}$\\
Program & $r$ & $R$ & $\mathbb{R}$
}

%\begin{definition} The \nonterm{expression} at the end of \nonterm{program} can
%be considered as the main clause of a program, which we'll refer to as
%$c_{main}$.\end{definition}

\begin{definition} For any given $v\in\mathbb{V}$ and $P\subset\mathbb{P}$,
we say that $v\in P$ if $v$ occurs in some $p\in P$.\end{definition}

\begin{definition}\label{definition:clause-tuple} A clause $c\in\mathbb{C}$ is
a tuple $\left\langle v,P,x \right\rangle$, where $v\in\mathbb{V}$ is the name
of the clause, $P\subset\mathbb{P}$ is a non-empty list of patterns of the
clause, and $x\in\mathbb{X}$ is the expression of the clause. $P$ is ordered by
occurrence of the patterns in the program text.\end{definition}

\begin{definition} We say that a clause $c= \left\langle v,P,x \right\rangle$
``accepts'' an argument list $B$ iff $|P|=|B|$ and $\forall\ \{i\mid 0\geq i <
|P|\}\ b_i\in B \wedge p_i\in P \wedge b_i\succ p_i$.\end{definition}

\begin{definition}\label{definition:function-tuple} A function $f\in\mathbb{F}$
is a tuple $\left\langle v,C \right\rangle$, where $v \in \mathbb{V}$ is the
name of the function, and $C\subset\mathbb{C}$ is the non-empty list of clauses
of the function. It must hold for $C$ that $\forall\ c\in C\
\left(c=\left\langle v_c, P_c, x_c \right\rangle \wedge v_c=v\right)$ and
$\forall\ c_1,c_2\in C\ \left(c_1=\left\langle v_1, P_1, x_1 \right\rangle
\wedge c_2=\left\langle v_2, P_2, x_2 \right\rangle \wedge |P_1|=|P_2|\right)$.
$C$ is ordered by occurrence of the clauses in the program
text.\end{definition}

\begin{definition} A signature of some function $f=\left\langle v, C
\right\rangle$ is the tuple $\left\langle v,|P| \right\rangle$, s.t.
$\forall\ c\in C_f\ |P_c|=|P|$. We'll adopt the Erlang notation when talking
about function signatures, i.e. if we have a function \mono{less} that takes in
two parameters, we'll refer to it as \mono{less/2}.\end{definition}

We assume for it to be fairly simple to construct the set $F$ of a given
program $r$ given the set of clauses $C$ derived during syntactic analysis of
the program text.

\begin{definition} A program $r$ is a tuple $\left\langle F,x \right\rangle$,
where $F\subset\mathbb{F}$ is the list of functions defined in program $r$, and
$x$ is the expression of program $r$.\end{definition}

\begin{definition}\label{definition:function-call} A function call is a tuple
$\left\langle v, X \right\rangle$, where $v\in\mathbb{V}$ is the name of the
callee, and $X\subset\mathbb{X}$ is a non-empty list of arguments for the
function call, ordered by occurrence of the expressions in the program
text.\end{definition}

0-ary clauses are disallowed to avoid having to deal with constants in general.
The term $\term{\_}$ in $\nonterm{pattern-element}$ is the conventional
wildcard operator; it indicates a value that won't used in the clause
expression, but some value has to be there for an argument to match the
pattern. Furthermore, as will be clear from the semantics, multiple occurrences
of \term{\_} in a clause pattern list does not indicate that the same value has
to be in place for each \term{\_}. 

\begin{definition} When describing various values and patterns in definitions,
theorems, proofs, etc. we'll sometimes make use of $\_$ to denote parts of the
value or pattern that are irrelevant to the said definition, theorem, proof,
etc.\end{definition}

% TODO this should be clear from the semantics.

% Multiple wildcards in the parameter list indicate possibly different value
% arguments, while multiple occurances of the same variable name in the parameter
% list are disallowed.

\subsection{Patterns constitute shapes}

The nonterminal declarations for patterns, in particular \ref{nonterm-pattern}
and \ref{nonterm-pattern-element}, indicate that a pattern are equatable to
shapes.

\begin{definition}\label{definition:pattern-corresponds-shape} The pattern
\term{0} corresponds to the leaf shape. The patterns \term{\_} and
\nonterm{name} correspond to triangle shapes. Any pattern \mono{a.b}
corresponds to the shape $a\cdot b$ iff the pattern \mono{a} corresponds to the
shape $a$ and the pattern \mono{b} corresponds to the shape
$b$.\end{definition}

\begin{definition}\label{definition:pattern-is-shape} $\forall\ p\in\mathbb{P}\
\forall\ s\in\mathbb{S}\ p=s$ iff $p$ corresponds to $s$ as by
\referToDefinition{pattern-corresponds-shape}.\end{definition}

\begin{definition} We overload the binary relation $\succ$ with the set\\
$\left\{ \left\langle p_1,p_2 \right\rangle\mid p_1,p_2\in\mathbb{P},
s_1,s_2\in\mathbb{S} \wedge p_1=s_1 \wedge p_2=s_2
\right\}$.\end{definition}

\subsection{Unary functions from multivariate functions}

The patterns of a clause as well as the arguments of a function call get
special treatment in \D{} in that they according to
\referToDefinition{clause-tuple} and \referToDefinition{function-call} are
ordered by their occurrence in the program text. This order is important to
make sure that the appropriate argument is matched against the appropriate
pattern. 

While this is setup is practical for the programmer, it is of no use to us due
to \referToTheorem{multivariate-to-unary}. In latter discussions, this
particular theorem allows us to keep to unary functions, and regard the
extension to multivariate functions as a fairly simple matter.

\begin{theorem}\label{theorem:multivariate-to-unary} Any multivariate function
in \D{} can be represented with a unary function.\end{theorem}

\begin{proof}

Given a multivariate function $f= \left\langle v,C \right\rangle$:

\begin{enumerate}

\item For each clause $c\in C$, where $c=\left\langle v,P,x \right\rangle$,
replace the pattern list $P$ with $P'=\{p\}$. Construct $p$ by initially
letting $p=0$, and folding left-wise over $P$, performing $p=p\cdot p'$ for
each $p'\in P$. 

\item For each call $\left\langle v, X\right\rangle$ to function $f$, replace
$X$ with the set $X'=\{x\}$, where $x$ has been constructed in a manner
equivalent to the pattern $p$ above.

\end{enumerate}

It is easy to see that both the constructed patterns and expressions are indeed
valid patterns and expressions, and that $f$ hence becomes a unary
function.\end{proof}

As this transformation is relatively simple to perform, we redefine the generic
clause tuple to have but one pattern in place of a list. 

\begin{definition}\label{definition:unary-clause} We redefine the clause $c$ to
be the tuple $\left\langle v,p,x\right\rangle$, where $v\in\mathbb{V}$ is the
name of the clause, $p\in\mathbb{P}$ is the pattern of the clause, and
$x\in\mathbb{X}$ is the expression of the clause.\end{definition}

\begin{definition}\label{definition:unary-function-call} We redefine a function
call to be the tuple $\left\langle v,x\right\rangle$, where $v\in\mathbb{V}$ is
the name of the callee, and $x\in\mathbb{X}$ is the argument to the (always
unary) callee.\end{definition}

\section{Semantics}\label{section:d-sos}

Revise the context of an expression within a function call, it should always be
the context upon entering the function call! Or even better, the context when
the function was defined!


\textbf{Perhaps pattern matching must be exhaustive in general.}

\textbf{Every subsequent definition must be strictly less specific than the former.}




In the following section we describe the semantics of \D{} using structured
operational semantics. The syntax used to define the reduction rules is largely
equivalent to the Aarhus report\cite{sos}, but differs slightly\footnote{The
syntax applied here is described in further detail in \referToAppendix{sos}.}.
The most notable about the syntax used here is the following:

\begin{itemize}

\item Rules should be read in increasing order of equation number.

\item If some rule with a lower equation number makes use of an undefined
reduction rule, it is because the reduction rule is defined under some higher
equation number.

\item Rules can be defined in terms of themselves, i.e. they can be recursive,
even mutually recursive.

\end{itemize}

The syntax aside, \referToTable{sos-definitions} defines a few lower-letter
shorthands for various constructs. Additionally, we'll let the capital
equivalents of these letters represent a sets of the respective construct, as
well as let the atoms $0$ and $\_$ represent themselves in the reduction rules.
It is also worth noting that $\forall\ v\in V\ :\ v\in \mathbb{B}$.

\makeTable[h!]
{sos-definitions}
{Overview of some of the shorthands used in this text. The column \textbf{A}
refers to all possible instances of the given construct, i.e.  $\mathbb{B}$
reffers to all constructable values in \D{}. The column \textbf{P} refers to
all the instances of the given construct in a given program, i.e. $N$ reffers
to all the names in a given program. The column \textbf{I} reffers to specific
instances of the given constructs, i.e. $x$ reffers to a particular
expression.}
{|l|c|c|c|}
{\textbf{Description}&\textbf{I}&\textbf{P}&\textbf{A}}
{
Expression & $x$ & $X$ & $\mathbb{X}$\\
Element (of an expression) & $e$ & $E$ & $\mathbb{E}$\\
Pattern & $p$ & $P$ & $\mathbb{P}$\\
Value(binary tree) & $b$ & $B$ & $\mathbb{B}$\\
Name & $n$ & $N$ & $\mathbb{N}$
}

\subsection{The memory model}\label{section:d-semantics-memory}

Memory is considered in terms of a set of value stacks, $\sigma$. Every stack
has a unique identifier $n\in N$, that is, each variable in a given program
gets a value stack. As we enter a new scope, we bind a variable to a value,
that is, we push that value on top of the corresponding stack. We pop the value
off the corresponding stack as we leave the scope at the entry to which the
variable was bound.

An expression at a certain scope depth only has access to variables at the same
scope depth. This is to ensure static scope. We won't adhere to this problem
explicitly in the semantics, but instead ask you to simply keep it in mind.

\subsubsection{Functions and variables}

Due to \D{} being a first-order language, we should make sure to separate the
function and variable spaces. We'll represent these by $\phi$ and $\gamma$,
respectively.

Whenever we use $\sigma$, $\phi$ or $\gamma$ in set notation, we imply the sets
of the names of functions and variables, and not the stacks themselves
corresponding to those names.  Hence, $\sigma=\phi\cup\gamma$, and to keep \D{}
first-order we add the limitation that $\phi\cap\gamma=\emptyset$.

\subsubsection{Making \D{} higher order}

The only change that this would require is to let $\phi=\gamma=\sigma$.

\subsection{Declaration}

A declaration with a name $n$, a \emph{non-empty} pattern
list $[p]$ and an expression $e$ is stored in the function space $\phi$:

\begin{equation}\label{sem:declaration}
{\displaystyle
  \left\langle \phi(n)\leftarrow \left\langle P, x\right\rangle\right\rangle
  \Rightarrow
  \phi'
\over\displaystyle
  \left\langle n, P, x, \phi\right\rangle
  \Rightarrow
  \phi'
}
\end{equation}

\subsection{Expression evaluation}

An expression $x$ is either the element $e$, or a construction of an element
$e'$ with another expression $x'$. That is, the binary infix operator $\cdot$
is right-associative, and has the following operational semantics:

\everymath{\displaystyle}

\begin{equation}
{\displaystyle
  \left\langle \proc{Single}, x,\sigma\right\rangle
  \rightarrow
  \left\langle v,\sigma\right\rangle
\vee
  \left\langle \proc{Chain}, x,\sigma\right\rangle
  \rightarrow
  \left\langle v,\sigma\right\rangle
\over\displaystyle
  \left\langle x,\sigma\right\rangle
  \rightarrow
  \left\langle v,\sigma\right\rangle
}
\end{equation}

\begin{equation}
{\displaystyle
  x\rightarrow e
\wedge
  \left\langle e,\sigma\right\rangle
  \rightarrow
  \left\langle v,\sigma\right\rangle
\over\displaystyle
  \left\langle \proc{Single}, x,\sigma\right\rangle
  \rightarrow
  \left\langle v,\sigma\right\rangle
}
\end{equation}

\begin{equation}
{\displaystyle
  x\Rightarrow e_1\cdot x_1
\wedge
  \left\langle e_1,\sigma\right\rangle
  \rightarrow
  \left\langle v_1,\sigma\right\rangle
\wedge
  \left\langle x_1,\sigma\right\rangle
  \rightarrow
  \left\langle v_2,\sigma\right\rangle
\over\displaystyle
  \left\langle \proc{Chain}, x, \sigma\right\rangle
  \rightarrow
  \left\langle v, \sigma\right\rangle
}
\quad(\text{where }v_1\cdot v_2=v)
\end{equation}

\subsection{Element evaluation}

According to the syntax specification, an element of an expression can either
be the atom $0$, or an application. We'd like to distinguish between variables
and functions, and we do that  

\begin{equation}
{\displaystyle
\left(
    e\Rightarrow 0
  \wedge
    v\equiv 0
\right)
\vee
{\displaystyle
    e\Rightarrow n
\over\displaystyle
    \beta(n)\Rightarrow v
}
\vee
{\displaystyle
    e\Rightarrow \left\langle n, X\right\rangle
\over\displaystyle
    \left\langle n,X,\sigma\right\rangle
    \Rightarrow
    \left\langle v,\sigma\right\rangle
}
\over\displaystyle
\left\langle e,\sigma\right\rangle
\Rightarrow
\left\langle v,\sigma\right\rangle
}
\end{equation}

\subsection{Function application}

\begin{equation}
{\displaystyle
{\displaystyle
{\displaystyle
  \left\langle n, \phi\right\rangle
  \Rightarrow
  \left\langle P, x, \phi\right\rangle
\over\displaystyle
  \left\langle P, X, \sigma\right\rangle
  \Rightarrow
  \sigma'
}
\over\displaystyle
  \left\langle x, \sigma'\right\rangle
  \Rightarrow
  \left\langle v,\sigma'\right\rangle
}
\over\displaystyle
    \left\langle n,X,\sigma\right\rangle
    \Rightarrow
    \left\langle v,\sigma\right\rangle
}
\end{equation}

\subsection{Pattern matching}

\begin{equation}
{\displaystyle
{\displaystyle
  \left\langle P_{head}, X_{head}, \sigma\right\rangle
  \Rightarrow
  \sigma''
\over\displaystyle
  \left\langle P_{tail}, X_{tail}, \sigma''\right\rangle
  \Rightarrow
  \sigma'
}
\over\displaystyle
  \left\langle P, X, \sigma\right\rangle
  \Rightarrow
  \sigma'
}
\end{equation}

\begin{equation}
{
  \left\langle\proc{I}, p,x,\sigma\right\rangle
  \Rightarrow
  \left\langle p',x',\sigma'\right\rangle
\vee
  \left\langle\proc{Z}, p,x,\sigma\right\rangle
  \Rightarrow
  \left\langle p',x',\sigma'\right\rangle
\vee
  \left\langle\proc{N}, p,x,\sigma\right\rangle
  \Rightarrow
  \left\langle p',x',\sigma'\right\rangle
\vee
  \left\langle\proc{P}, p,x,\sigma\right\rangle
  \Rightarrow
  \left\langle p',x',\sigma'\right\rangle
}\over{
  \left\langle p, x, \sigma\right\rangle
  \Rightarrow
  \left\langle p', x', \sigma'\right\rangle
}
\end{equation}

For the sake of an elegant notation, we'll override the function $\cdot$ for
patterns.

\begin{definition}

A pattern is an unlabeled of binary tree which is either empty or consists of
an unlabeled node with a $0$, $\_$, name, or a pattern as it's left and right
child. 

\end{definition}

\begin{definition}

Let the set of all possible patterns be denoted by $\mathbb{P}$.

\end{definition}

\begin{definition}

The function $\cdot
:\mathbb{P}\times\mathbb{P}\rightarrow\mathbb{P}$ constructs a pattern node with the
two arguments as it's left and right child, respectively. 

\end{definition}

\begin{equation}
{
    p\Rightarrow \_\cdot p'
  \wedge
    x\Rightarrow e\cdot x'
  \wedge
    \sigma\Rightarrow\sigma'
}\over{
  \left\langle\proc{Underscore}, p,x,\sigma\right\rangle
  \Rightarrow
  \left\langle p',x',\sigma'\right\rangle
}
\end{equation}

\begin{equation}
{
    p\Rightarrow 0\cdot p'
  \wedge
    x\Rightarrow e\cdot x'
  \wedge
    \sigma\Rightarrow\sigma'
}\over{
  \left\langle\proc{Zero}, p,x,\sigma\right\rangle
  \Rightarrow
  \left\langle p',x',\sigma'\right\rangle
}
\end{equation}


\begin{equation}
{\displaystyle
{\displaystyle
{\displaystyle
    p\Rightarrow n\cdot p'
  \wedge
    x\Rightarrow e\cdot x'
\over\displaystyle
    \left\langle e,\sigma\right\rangle
    \Rightarrow
    \left\langle v,\sigma\right\rangle
}\over\displaystyle
    \left\langle\sigma(n)\leftarrow v\right\rangle
    \Rightarrow
    \sigma'
}\over\displaystyle
  \left\langle\proc{Name}, p,x,\sigma\right\rangle
  \Rightarrow
  \left\langle p',x',\sigma'\right\rangle
}
\end{equation}

\begin{equation}
{\displaystyle
{\displaystyle
    p\Rightarrow p''\cdot p'  
  \wedge
    x\Rightarrow x''\cdot x'
\over\displaystyle
  \left\langle p'', x'', \sigma\right\rangle
  \Rightarrow
  \sigma'
}
\over\displaystyle
  \left\langle\proc{Pattern}, p,x,\sigma\right\rangle
  \Rightarrow
  \left\langle p',x',\sigma'\right\rangle
}
\end{equation}

\input{language/input}
\section{Built-in high-order
functions}\label{section:language-higher-order-built-ins}

Although \mono{D} is initially a first-order language, we will ignore that
limitation for a bit and define a few higher-order functions to provide some
syntactical sugar to the language. Beyond the discussion in this section, these
higher-order functions should be regarded as \mono{D} built-ins.

\subsubsection{Branching}

In the following definition, the variable names \mono{true} and \mono{false}
refer to expressions to be executed in either case.

\begin{verbatim}
if 0 _ false := false
if _._ _ true := true
\end{verbatim}

As you can see, we employ the C convention that any value other than $0$ is a
``truthy'' value, and the expression \mono{true} is returned.

Although the call-by-value nature of the language does not allow for
short-circuiting the if-statements defined in such a way, this shouldn't be any
impediment to further analysis.

\subsection{Boolean operations}

\begin{verbatim}
and _._ _._ = 0.0
and _ _ = 0

or 0 0 = 0
or _ _ = 0.0
\end{verbatim}


\section{Sample programs}\label{section:d-samples}

As an illustration of the language syntax, take a look at the programs in
\referToListing{reverse}, \referToListing{fibonacci} and
\referToListing{ackermann}.

\begin{lstlisting}[label=listing:reverse,
  caption={A program that reverses the order of the nodes of some supplied tree.}]
reverse 0 := 0
reverse left.right := (reverse right).(reverse left)

reverse input
\end{lstlisting}

\begin{lstlisting}[label=listing:fibonacci,
  caption={A program that computes the $n^{th}$ fibonacci when supplied with some $n$.}] 
fibonacci n = fibonacci-aux (normalize n) 0 0

fibonacci-aux 0 x y := 0
fibonacci-aux 0.0 x y := y
fibonacci-aux 0.n x y := fibonacci-aux n y (add x y)

fibonacci input
\end{lstlisting}

\begin{lstlisting}[label=listing:ackermann,
  caption={The Ackermann-P\'eter function.}]
ackermann 0 n := 0.n
ackermann a.b 0 := ackermann (decrease a.b) 1
ackermann a.b c.d := ackermann (decrease a.b) (ackermann a.b (decrease c.d))

ackermann input input
\end{lstlisting}

