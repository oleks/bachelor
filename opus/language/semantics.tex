\section{Semantics}\label{section:d-sos}

In the following section we describe the semantics of \D{} using a form of
structured operational semantics. The syntax used to define the reduction rules
is largely equivalent to the Aarhus report\cite{sos}, but differs
slightly\footnote{The syntax applied here is described in further detail in
\referToAppendix{sos}.}.

\subsection{The memory model}\label{section:d-semantics-memory}

\begin{definition}\label{definition:memory} The memory is a binary relation
$\sigma$, which is the set $\{(v,b)\mid v\in\mathbb{V} \wedge
b\in\mathbb{B}\}$.\end{definition}

To keep \D{} first order we distinguish between the function space and variable
space.

\begin{definition} Let $\phi\subseteq\sigma$ represent the function space and
let $\beta\subseteq\sigma$ represent the variable space, what's more,
$\phi\cup\beta=\sigma$. When we refer to $\sigma$, $\phi$ or $\beta$ in set
notation, we refer merely to the names of the variables, hence to keep \D{}
first order, let $\phi\cap\beta=\emptyset$.\end{definition}

\subsubsection{Making \D{} higher order}

The only change that this would require is to let $\phi=\beta=\sigma$.

\subsection{Program evaluation}

Given a program $r=\left\langle F,x \right\rangle$, we apply the following
semantics:

\begin{equation}\label{sem:declaration}
{\displaystyle
    \left\langle F,\emptyset\right\rangle
    \rightarrow
    \phi_1
  \wedge
    \sigma=\phi_1
  \wedge
    \left\langle x,\sigma\right\rangle
    \rightarrow
    \left\langle b,\sigma\right\rangle
\over\displaystyle
  \left\langle F, x\right\rangle
  \rightarrow
  b
}
\end{equation}

\subsection{Function declarations}

Given a list of functions $F$, and some function space $\phi$, we apply the
following semantics:

\begin{equation}\label{sem:declaration}
{\displaystyle
  \left(
      F=\emptyset
    \wedge
      \phi=\phi_1
  \right)
  \vee
  \left(
      F_{head}=\left\langle v,C\right\rangle
    \wedge
      \phi_2=\phi[v]\mapsto \left\langle C,\phi\right\rangle
    \wedge
      \left\langle F_{tail},\phi_2\right\rangle
      \rightarrow
      \phi_1
  \right)
\over\displaystyle
  \left\langle F, \phi\right\rangle
  \rightarrow
  \phi_1
}
\end{equation}

The fact that the active function space is saved together with the list of
clauses for any given function should intentionally indicate that \D{} is
\emph{statically scoped}.

\subsection{Expression evaluation}

An expression $x$ is either the element $e$, or a construction of an element
$e_1$ with another expression $x_1$. That is, the binary infix operator $\cdot$
is right-associative, and has the following operational semantics:

\everymath{\displaystyle}

\begin{equation}
{\displaystyle
  \left\langle \proc{Single}, x,\sigma\right\rangle
  \rightarrow
  b
\vee
  \left\langle \proc{Chain}, x,\sigma\right\rangle
  \rightarrow
  b
\over\displaystyle
  \left\langle x,\sigma\right\rangle
  \rightarrow
  b
}
\end{equation}

\begin{equation}
{\displaystyle
  x = e
\wedge
  \left\langle e,\sigma\right\rangle
  \rightarrow
  b
\over\displaystyle
  \left\langle \proc{Single}, x,\sigma\right\rangle
  \rightarrow
  b
}
\end{equation}

\begin{equation}
{\displaystyle
  x = e_1\cdot x_1
\wedge
  \left\langle e_1,\sigma\right\rangle
  \rightarrow
  b_1
\wedge
  \left\langle x_1,\sigma\right\rangle
  \rightarrow
  b_2
\over\displaystyle
  \left\langle \proc{Chain}, x, \sigma\right\rangle
  \rightarrow
  b
}
\quad(\text{where }b_1\cdot b_2=b)
\end{equation}

\subsection{Element evaluation}

According to the syntax specification, an element of an expression can either
be the atom $0$, a variable, an expression (in parentheses), or an application. 

\begin{equation}
{\displaystyle
  \left\langle \proc{Zero}, e,\sigma\right\rangle
\vee 
  \left\langle \proc{Expression}, e,\sigma\right\rangle
\vee
  \left\langle \proc{Variable}, e,\sigma\right\rangle
\vee
  \left\langle \proc{Application}, e,\sigma\right\rangle
\over\displaystyle
  \left\langle e,\sigma\right\rangle
  \rightarrow
  \left\langle b,\sigma\right\rangle
}
\end{equation}

\begin{equation}
{\displaystyle
  e = 0
\wedge
  b = 0
\over\displaystyle
  \left\langle \proc{Zero}, e,\sigma\right\rangle
  \rightarrow
  b
}
\end{equation}

\begin{equation}
{\displaystyle
  e = x
\wedge
  \left\langle x,\sigma\right\rangle
  \rightarrow
  b
\over\displaystyle
  \left\langle \proc{Expression}, e,\sigma\right\rangle
  \rightarrow
  b
}
\end{equation}

\begin{equation}
{\displaystyle
  e = v
\wedge
  \beta[v]=b
\over\displaystyle
  \left\langle \proc{Variable}, e,\sigma\right\rangle
  \rightarrow
  b
}
\end{equation}

\begin{equation}
{\displaystyle
  e = \left\langle v, x\right\rangle
\wedge
  \phi[v]=\left\langle C,\phi_1 \right\rangle
\wedge
  \left\langle x,\sigma\right\rangle
  \rightarrow
  b_{arg}
\wedge
  \left\langle C,b_{arg},0,\phi_1\right\rangle
  \rightarrow
  b
\over\displaystyle
  \left\langle \proc{Application}, e,\sigma\right\rangle
  \rightarrow
  b
}
\end{equation}

\subsection{Clause matching}

%\begin{definition} We define the binary relation $\curlyvee$ to be the set
%$$\left\{(c_1,c_2)\mid c_1,c_2\in\mathbb{C} \wedge B_1 = \left\{ b \mid
%b\in\mathbb{B} \wedge b \succ c_1 \right\} \wedge B_2 = \left\{ b \mid
%b\in\mathbb{B} \wedge b \succ c_2 \right\} \wedge |B_1 \cup B_2| >
%|B_1|\right\}$$\end{definition}

%Clauses get some special get some special treatment in \D{}. In particular,
%according to \referToDefinition{clause-tuple}, the clauses of a function are
%ordered by their occurrence in the program text. This is intentional, as we
%would like to declare the following requirement:

%\begin{definition} Given a function $f= \left\langle v, C \right\rangle$, it
%must hold that $\forall\ i,j\in\{(i,j)\mid i,j\in\mathbb{Z}^+ \wedge 0 > i \leq
%|C| \wedge i > j \leq |C|\}\ c_i \curlyvee c_j$, where the binary relation
%$\curlyvee$ is overloaded with the set $\{(c_1,c_2)\mid c_1,c_2\in\mathbb{C}
%\wedge c_1= \left\langle v_1,p_1,x_1 \right\rangle \wedge c_2= \left\langle
%v_2,p_2,x_2 \right\rangle \wedge p_1\curlyvee p_2\}$.\end{definition}

We would like to ensure that pattern matching is exhaustive for any function
definition. This is to avoid programs that terminate due to inexhaustive
pattern matching.

\begin{definition} Given a function $f=\left\langle v, C\right\rangle$, it must
hold that $\forall\ b\in\mathbb{B}\ \exists\ \left\langle v,p,x \right\rangle
\in C\ b\succ p$.\end{definition}

This definition allows us to define the following semantics for clause
evaluation:

\begin{equation}
{\displaystyle
  \left(
      C=\emptyset
    \wedge
      b=b_{acc}
  \right)
  \vee
  \left(
      C_{head}=\left\langle v,p,x\right\rangle
    \wedge
      b_{arg}\succ p
    \wedge
      \left\langle x,\phi\right\rangle
      \rightarrow
      \left\langle b,\phi\right\rangle
  \right)
  \vee
  \left(
      \left\langle C_{tail},b_{arg},0,\phi\right\rangle
      \rightarrow
      b
  \right)
\over\displaystyle
  \left\langle C,b_{arg},b_{acc},\phi\right\rangle
  \rightarrow
  b
}
\end{equation}
