\chapter{Size-Change Termination}\label{section:size-change-termination}

The size-change termination analysis builds upon the idea of flow analysis of
programs. In general, flow analysis aims to answer the question, ``What can we
say about a given point in a program without regard to the execution path taken
to that point?''. A ``point'' in a computer program, is in this case a primitive
operation such as an assignment, a condition branch, etc.

The idea is then to construct a graph where such points are nodes, and the arcs
in between them represent a transfer of control between the primitive
operations, that would otherwise occur under the execution of the program.
Such a node may have variable in-degree and out-degree. For instance, a
condition branch would usually have two possible transfers of control depending
on the outcome of the condition. Hence, it serves useful to label arcs
depending on when they are taken.

Such graphs are referred to as \emph{control flow graphs}. With a control flow
graph at hand, various optimization algorithms can be devised to traverse the
graph and deduce certain properties, such as e.g.  reoccurring primitive
operations on otherwise static variables\cite{kildall}.

\section{Control flow graphs (or call graphs) in \D{}}

\subsection{Start and end nodes}

Conventionally, a control flow graph has a start and an end node. These nodes
do not explicitly represent control primitives, but rather the start and end of
a program. Clearly, a program cannot be started nor ended more than once, and
hence the start node, has out-degree $1$ and in-degree $0$, while the end node
has out-degree $0$, and (initially) variable in-degree since a program can be
ended in more than one way. For reasons that will become apparent in later on,
we chose to disregard the start and end nodes completely.

\subsection{Function clauses}

While node construction and destruction are primitive operations in \D{}, we'll
refrain ourselves from delving into such details in the control flow graphs of
our programs. This is because by the semantics of \D{}, node construction and
destruction always terminates. Instead, we'll let function clauses define
primitive program points. The expression of a given clause can make calls to
its enclosing, or some other function. Such calls are represented by transfer
of control, that is, edges between clauses.

\begin{definition}\label{definition:size-change-first-graph} Given a program
$r=\left\langle F,x \right\rangle$ we define a control flow graph $G =
\left\langle C, E \right\rangle$, where

$$C= \left\{ c \mid f= \left\langle v,C_f \right\rangle \in F \wedge c\in C_f
\right\},$$

and

$$E=\{ \left\langle c_s,c_t,x\right\rangle \mid c_s = \left\langle v_s, p_s,
x_s \right\rangle \in C \wedge c_t = \left\langle v_t, \_, \_ \right\rangle \in
C \wedge x\in\mathbb{X} \wedge \left\langle v_t, x \right\rangle \Subset x_s\}
.$$

\end{definition}

\begin{definition} Given a control transition graph $G= \left\langle C,E
\right\rangle$, we refer to a directed edge $e\in E$ as a call. We refer to $G$
as a call graph and given any call $\left\langle c_s,c_t,x \right\rangle \in
C$, we say that $c_s$ is the \emph{source} clause, $c_t$ is the \emph{target}
clause, and $x$ is the call argument.  \end{definition}

\begin{lemma}\label{lemma:first-call-graph-finite} Given a call graph $G=
\left\langle C,E \right\rangle$, $C$ and $E$ are both finite.\end{lemma}

\begin{proof} Follows from the semantics of \D{} and
\referToDefinition{size-change-first-graph}.\end{proof}

\begin{definition} Given a call graph $G = \left\langle C,E \right\rangle$ we
refer to the list $\left\langle c_1,c_2,\_ \right\rangle, \left\langle
c_2,c_3,\_ \right\rangle, \ldots,\left\langle c_{n-1}, c_n,\_ \right\rangle \in
E$ as a ``path''.\end{definition}

\referToDefinition{size-change-first-graph} might strike you as odd, since by
the semantics of \D{}, when we make a function call to some function $f$, we
will iteratively consider the list of clauses contained in the function,
looking for one which accepts the argument. While this is true, we will merely
concern ourselves with those control transitions, where a change in the values
of the program variables can occur. The transitions between the clauses of a
single definition, which occur when a function call argument is matched to a
clause, call them \emph{fail transitions}, do not change the value of the
argument but merely propagate it.  Hence, fail transitions are irrelevant to
our analysis, so long as we have an edge from the source clause to each
possible target clause, which is exactly what
\referToDefinition{size-change-first-graph} states.

We use a triplet to represent a call in order to ensure that calls with
different arguments get different edges. This is important to clauses with
expressions where multiple calls to the same function are made. In particular,
the different calls might modify the values in different ways.

\subsection{Order of evaluation}

\referToDefinition{size-change-first-graph} indicates that we disregard the
order of evaluation of the arguments. This too, is intentional.  In particular,
if all the call cycles terminate, then so will all the evaluations. We show
this with a few examples below and prove formally when we discuss
\referToTheorem{size-change}.

If calls are separated by node construction, the order in which those calls are
made is definitely insignificant. For instance, consider the expression
\mono{(f a).(g b)}, where \mono{f} and \mono{g} are some well-defined
functions, $\text{\mono{f}}\neq\text{\mono{g}}$, and \mono{a} and \mono{b} are
some bound variables. It makes no difference to the final result which of the
calls, \mono{f a} and \mono{g b}, is evaluated first. Indeed, they can be
evaluated in parallel, and we would still get the same result. This is easy to
see for any nested construction of results of function calls, as in e.g.
\mono{(f a).0.(g b)}.

On the other hand, the syntax and semantics of \D{} allow for function calls to
be nested as in e.g. the expression \mono{(f (g a) (h b))}, where \mono{h} is
also some well-defined function and is pairwise unequal to \mono{f} and
\mono{g}. The order of evaluation of \mono{(g a)} and \mono{(h b)} is
insignificant wrt. to each another, as with function calls separated by
construction. However, the order of evaluation of these two subexpressions wrt.
to the call to function \mono{f}, is significant to the result, and
\emph{might} be significant to termination analysis in general. However, we'll
initially regard this as insignificant for mere simplicity.

\subsection{An example}

We can now draw a control flow graph for the program define in
\referToListing{cfg-sample-1} as shown in \referToFigure{cfg-sample-1-pdf}.

\begin{lstlisting}[label=listing:cfg-sample-1,caption={A sample \D{} program, always returning \mono{0.0.0}.}]
f x y := x.y
g _ := 0
h _ := 0
i x y := (f ((h y).(g x)) (h y))
i input input
\end{lstlisting}

\includeFigure{cfg-sample-1-pdf}{A  control flow graph for the \D{} program in
\referToListing{cfg-sample-1}. The graph does not explicitly specify
back-propagation of control, if any.}

\subsection{Disregarding back-propagation}

It is worth noting that in \referToFigure{cfg-sample-1-pdf}, the clauses that
make no function calls have out-degree $0$. Technically, these functions do
transfer control, in particular, back to the callee. We may refer to this
process as \emph{back-propagation} of control. While considering
back-propagation is seemingly important to a concept that bases itself on the
changes in the sizes of the program values, we won't be concerned with any
exact values.

The thing with back-propagation is that forward-propagation after
back-propagation of a call cannot occur due to the way \D{} is defined. Hence,
what we are really concerned with is, ``how deep the rabbit hole goes'', before
we back-propagate, as back-propagation superimplies termination of the function
we're back-propagating out of.

\subsection{Call cycles}

\begin{definition}\label{definition:call-cycle} Given a call graph $G =
\left\langle C,E \right\rangle$, a call cycle is a path\\ $z= \left\langle
c_1,c_2,\_ \right\rangle, \left\langle c_2,c_3,\_ \right\rangle,
\ldots,\left\langle c_{n-1}, c_n,\_ \right\rangle \in E$ s.t.
$c_1=c_n$.\end{definition}

\begin{theorem}\label{theorem:acyclic-graph-terminates} Given program $r=
\left\langle F, x \right\rangle$ has a corresponding acyclic graph $G =
\left\langle C,E \right\rangle$, then the program terminates.\end{theorem}

\begin{proof} Assume for the sake of contradiction that the program does not
terminate. Then one of the acyclic paths in the graph must not terminate. Since
the set $E$ is finite by \referToLemma{first-call-graph-finite}, then any path
in $G$ must be finite.  Hence, one of the primitive operations, i.e.
construction, destruction, comparison, binding, function call, etc., must not
terminate. By the semantics of \D{} this is absurd.\end{proof}

\begin{definition}\label{definition:recursive-terminal} Given a call graph $G =
\left\langle C,E \right\rangle$, we refer to the set of calls that participate
in a call cycle as recursive clauses and all other clauses as terminal
clauses.\end{definition}

\begin{definition} If a program choses some edge $e$ over another edge $e'$, we
say that the program branches off to a new path starting with edge
$e$.\end{definition}

\begin{theorem} If a program branch takes a path $k$ consisting solely of
terminal clauses, the branch terminates.\end{theorem}

\begin{proof} If a program branches off to a path $k$ and $k$ consists solely
of terminal clauses, then the program may be regarded as having branched off
into a new program with a call graph consisting solely of the clauses visited
by $k$ and edges in $k$. Such a call graph is acyclic due to
\referToDefinition{recursive-terminal}, and by
\referToTheorem{acyclic-graph-terminates} is finite. The branch must therefore
terminate.\end{proof}

\begin{corollary}\label{corollary:program-branch-terminate} A program
terminates if all of its branches terminate.\end{corollary}

\referToCorollary{program-branch-terminate} allows us to disregard all the
clauses of a call graph that do not participate in call cycles.

\begin{definition}\label{definition:recursive-call-graph} Given a call graph
$G=\left\langle C, E \right\rangle$, and the set of call cycles $Z$ in $G$, we
define a call graph $G^r=\left\langle C^r, E^r\right\rangle$ to be the
``recursive call graph'', where $C^r=\left\{ c \mid c\in C \wedge c\Subset Z
\right\}$ and $E^r = \left\{ e \mid e\in E \wedge e\Subset Z \right\}$. WLOG,
all call graphs we refer to from this point on will be recursive call
graphs.\end{definition}

\subsection{Relation of call graphs to conventional static call graphs}

Conventionally, a static call graph is a graph over all the calls made by a
program at runtime. Such a graph has an infinite number of edges for any
nonterminating program in \D{}. Hence, deducing the halting property can be
rephrased as determining whether the static call graph has an infinite number
of edges.

\subsection{Visualization}

When describing size-change termination we'll often revert to examples. Here we
will make use of a few conventions wrt. listings and graphs s.t. a call graph
for any given program is easy to visualize.

\subsubsection{Multiple calls to the same function}

An edge $e\in E$ in a call graph $G = \left\langle C,E \right\rangle$ was
defined to be the tuple $\left\langle c_1, c_2, x \right\rangle : C \times C
\times \mathbb{X}$. While this ensures to distinguish calls to the same
function with different arguments from the same expression, it is not
particularly friendly to the eye to visually annotate each edge with an
expression.

Instead, we adopt the convention of disjunctively labelling all the function
calls of an expression as in \cite{size-change}. We'll do this both in listings
and in visualizations. Refer to \referToListing{cfg-sample-2} and
\referToFigure{cfg-sample-2-pdf} for an example.

\begin{lstlisting}[label=listing:cfg-sample-2,
  caption={A sample \D{} program, always returning \mono{(0.x).(0.y)},
  where \mono{x} and \mono{y} are arbitrary \D{} values supplied by the user.}]
f x y := x.y
g x := 0.x
i x y := (0: f (1: g x) (2: g y))
i input input
\end{lstlisting}

\includeFigure{cfg-sample-2-pdf}{A  control flow graph for the \D{} program in
\referToListing{cfg-sample-2}.}

\subsubsection{Multiple clauses}

As multiple clauses denote different nodes in the call graph, we would also
like to visually distinguish the nodes while keeping a relation to the original
listing. Hence, each clause of a function in the program listing will be
labeled with a label prefixed with the function name and postfixed with the
clause index starting at $0$. Refer to \referToListing{cfg-loop} and its
corresponding call graph in \referToFigure{cfg-loop-pdf} for an example.

\begin{lstlisting}[label=listing:cfg-loop,
  caption={A simple, down-counting loop in \D{}.}]
$f_0$: f 0 := 0
$f_1$: f x._ := f x
f input
\end{lstlisting}

\includeFigure[scale=1.5]{cfg-loop-pdf}{A  control flow graph for the program
defined in \referToListing{cfg-loop}.} 

As a more complex example, consider the call graph for the program
\mono{reverse} introduced in \referToSection{d-samples}. The program is
repeated in annotated form in \referToListing{cfg-reverse}, and its
corresponding call graph is shown in \referToFigure{cfg-reverse-pdf}.

\begin{lstlisting}[label=listing:cfg-reverse,
  caption={An annotated version of the program \mono{reverse} introduced in
  \referToSection{d-samples}.}]
$r_0$: reverse 0 := 0
$r_1$: reverse left.right := (0: reverse right).(1: reverse left)
reverse input
\end{lstlisting}

\includeFigure[scale=1.5]{cfg-reverse-pdf}{A  control flow graph for the \D{}
program in \referToListing{cfg-reverse}.}

\section{Size-change termination principle}

Consider the program in \referToListing{cfg-loop} and its corresponding call
graph in \referToFigure{cfg-loop-pdf}. Without any further information about
the calls, the program seemingly loops indefinitely. However,
there are some things that we can deduce about the control transitions.

\begin{theorem}\label{theorem:size-change} If every call cycle in a given
program reduces a value of a well-founded data-type on each iteration of the
cycle, then the value must eventually bottom out and the program must
terminate.\end{theorem}

\begin{proof} Assume for the sake of contradiction that a program that reduces
a value of a well-founded data type in each call cycle does not terminate.
Then, either the value reduces indefinitely, which is a contradiction to the
well-foundedness of its data type, or some noncyclic call sequence causes an
infinite loop, also an absurdity due to the definition of \D{}. \end{proof}

That is the \emph{size-change termination principle}\cite{size-change}. All
values in \D{} are inherently well-founded so what remains to be shown is how
we can deduce from a call cycle whether it reduces a value on each iteration.

\begin{definition}\label{definition:size-relation} For a given call
graph $G = \left\langle C,E \right\rangle$, let a size relation be the set

$$
\Phi = \left\{ \left\langle c_s, c_t,x, v_s, v_t, \rho \right\rangle \left| 
\begin{array}{ll}
&\left\langle c_s, c_t,x \right\rangle \in E\\
\wedge&c_t = \left\langle \_, p_t, \_ \right\rangle \in C \\
\wedge&v_s,v_t\in\mathbb{V}\\
\wedge&v_s\Subset x \wedge v_t \Subset p_t \\
\wedge&\rho\in\{\bot, <, \leq\}
\end{array}
\right.\right\}
$$

\end{definition}

This definition implies that for each call cycle there may be multiple
variables\footnote{We'll define what we mean by a variable more formally in
\referToDefinition{variable}.} to ensure reduction for. However, as there is
only a finite number of variables in any given pattern and any given
expression, we can define these as separate cycles.

\begin{definition}\label{definition:variable-call-graph} Given a recursive call
graph $G^r = \left\langle C^r, E^r \right\rangle$, let $G^u= \left\langle C^u,
E^u \right\rangle$ be a ``unary recursive call graph'', where

$$E^u = \left\{ \left\langle c_s, c_t, x, v_s, v_t \right\rangle \left|
\begin{array}{ll}
&\left\langle c_s, c_t,x \right\rangle \in E^r\\
\wedge&c_t = \left\langle \_, p_t, \_ \right\rangle \in C^r \\
\wedge&v_s,v_t\in\mathbb{V}\\
\wedge&v_s\Subset x \wedge v_t \Subset p_t
\end{array}
\right.\right\},$$

and $C^u= \left\{ c \mid c \in C^r \wedge c\Subset E^u
\right\}$.\end{definition}

\begin{theorem} A unary recursive call graph $G^u$ has a finite number of
edges.\end{theorem}

\begin{proof} Follows from \referToDefinition{variable-call-graph} and the
semantics of \D{}.\end{proof}

\referToDefinition{variable-call-graph} forces us to redefine the concept of a
call cycle, as a sequence of clauses may have multiple cycles in $G^u$.

\begin{definition}\label{definition:variable-call-cycle} Given a recursive call
graph $G^r = \left\langle C^r, E^r \right\rangle$, let $Z^r$ denote the set of
call cycles in $G^r$, and $Z^u$ the set of call cycles in $G^u$, then

$$Z^u = \bigcup_{z^r\in Z^r} \left\{ \left\langle c_s, c_t, x, v_s, v_t \right\rangle \left|
\begin{array}{ll}
&\left\langle c_s, c_t,x \right\rangle \in z^r\\
\wedge&c_t = \left\langle \_, p_t, \_ \right\rangle \in C^r \\
\wedge&v_s,v_t\in\mathbb{V}\\
\wedge&v_s\Subset x \wedge v_t \Subset p_t
\end{array}
\right.\right\}.$$

\end{definition}

\referToDefinition{variable-call-graph} allows us now to more formally define
what we've thus far meant as a variable that changes value from call to call.

\begin{definition}\label{definition:variable} Given a unary recursive call
graph $G^u$, every call cycle $z$ in $G^u$, changes exactly one variable, call
it ``cycle variable'', or $v_z$.\end{definition}

Hence, while a variable may have different names in different clauses, we've
defined an overall variable for every call cycle, and want to ensure that this
variable is reduced in every iteration of the call cycle.

\begin{theorem}\label{theorem:multivariable-patterns} We can WLOG to
termination analysis limit our attention to programs that bind at most one
variable in every clause.\end{theorem}

\begin{proof} Let a recursive call graph $G^r$ have a set of call cycles $Z^r$,
and a corresponding unary recursive call graph $G^u$ with a set of call cycles
$Z^u$. By \referToDefinition{variable-call-cycle} and
\referToDefinition{size-relation}, every call cycle in $Z^r$ reduces a value
iff every call cycle in $Z^u$ reduces a value. We can hence limit our attention
to deducing if each call cycle in $Z^u$ reduces a value.\end{proof}

\begin{definition}\label{definition:nice-call-graph} Let $G^1$ denote a
recursive call graph of a program where each clause binds at most one
variable.\end{definition}

Given \referToTheorem{multivariable-patterns} we can return to the old
definition of a call graph as per \referToDefinition{recursive-call-graph} and
call cycle as per \referToDefinition{call-cycle}. This demands a
simplification of the size relation $\Phi$.

\begin{definition}\label{definition:unary-size-relation} For a given $G^1 =
\left\langle C,E \right\rangle$ of a program, let a size relation be the set

$$\Phi^1 = \left\{ \left\langle c_s, c_t,x, \rho \right\rangle \mid
\left\langle c_s, c_t,x \right\rangle \in E \wedge \rho\in\{\bot, <, \leq\}
\right\}$$

\end{definition}

\begin{definition}\label{definition:increasing-decreasing-call} Given a call
graph $G$ with a call cycle $z$, a call $\left\langle c_s,c_t,x\right\rangle\in
z$ is,

\begin{enumerate}

\item A ``decreasing call'' iff $\left\langle c_s,c_t,x,< \right\rangle \in
\Phi^1$.

\item A ``nonincreasing call'' iff $\left\langle c_s,c_t,x,\leq \right\rangle
\in \Phi^1$.

\item An ``undteremined call'' iff $\left\langle c_s,c_t,x,\bot \right\rangle
\in \Phi^1$.

\end{enumerate}

\end{definition}

\begin{lemma}\label{lemma:cycle-reduce} A call cycle $z= \left\langle c_1,c_2
\right\rangle, \left\langle c_2, c_3 \right\rangle,\ldots, \left\langle
c_{n-1}, c_n \right\rangle$ reduces a value on each iteration iff
$$\left(\forall\ \left\langle c_i,c_j \right\rangle \in z\ \left(\left\langle
c_i,c_j,\_, < \right\rangle \in \Phi^1\right) \vee \left(\left\langle
c_i,c_j,\_, \leq \right\rangle \in \Phi^1 \right)\right)\wedge \left( \exists\
\left\langle c_i,c_j \right\rangle \in z\ \left\langle c_i,c_j,\_, <
\right\rangle \in \Phi^1 \right)$$.\end{lemma}

\begin{proof} If a value is not reduced in a cycle, it either stays the same or
is increased. If it is increased, then at least one call must've increased the
value, which is an absurdity. If it stays the same then none of the
participating control transitions have neither increased nor decreased the
value, also an absurdity.\end{proof}

\subsection{Deducing $\Phi^1$}

\begin{definition} Given a call graph $G^1 = \left\langle C^1,
E^1\right\rangle$ and a call $\left\langle c_s = \left\langle \_,p_s,\_
\right\rangle, c_t = \left\langle \_, p_t, \_ \right\rangle, x \right\rangle
\in E^1$, let $v_{p_s}$ denote the variable s.t. $v_{p_s}\in\mathbb{V}\wedge
v_{p_s}\Subset p_s$, and let $v_{p_t}$ denote the variable s.t.
$v_{p_t}\in\mathbb{V}\wedge v_{p_t}\Subset p_t$. \end{definition}

Since \D{} is a call-by-value language, for any given call $\left\langle
c_s,c_t,x \right\rangle \in E^1$, the source evaluates $x$, and generates some
$b\in\mathbb{B}$ as the actual argument for the call. The expression $x$ is by
definition a nested construction of either some concrete value, some variable
$v_{p_s}$, or a nested function call. Without further regard of nested function
calls, this implies that a size relation can be deduced between the $v_{p_s}$
and $b$.

We decide to ignore the nested function calls because this would imply a more
complex static analysis of the program. Specifically, we're unable to say
anything about the result of the nested function call from the scope of $c_s$
alone. Instead, we treat results from nested function calls simply as variables
with \emph{unknown} values. We also make sure to keep these variables separate
from the bound variables as there is no relationship to draw between these
auxiliary variables and the variable $v_t\in\mathbb{V}$ bound in $c_t$.

Due to nested functions calls being represented as variables with unknown
values, a precise size displacement between the bound variable $v_s$ and the
generated argument $b_s$ is not feasible. However, we can deduce a \emph{safe}
displacement estimate.

\begin{definition} A safe displacement estimate between the values
$b_1,b_2\in\mathbb{B}$ is a value $n\in\mathbb{N}$ s.t. $b_1+n\leq
b_2$.\end{definition}

\begin{definition}\label{definition:source-variable} Given a call
$\left\langle c_s,c_t,x_s\right\rangle$, we construct the expression $p_s$ where we
replace all $\left\langle \_,\_ \right\rangle\Subset p_s$ with an auxiliary
variable. We group those auxiliary variables in the set $V^s_{calls}$. There is
only one possible variable in use in $x_s$, we denote this variable
$v^s$.\end{definition} 

\begin{definition}\label{definition:clause-variable} Given a clause
$c_x=\left\langle \_,p_x,\_  \right\rangle \in \mathbb{C}$, we refer to the
only variable in $p_x$ as $v^x$, hence $v^x\in\mathbb{V} \wedge v^t\Subset
p_x$.\end{definition}

\begin{lemma} An expression $p_s\in\mathbb{X}$ is interchangeable with a
pattern. Hence, we say that $p_s\in\mathbb{P}.$\end{lemma}

\begin{proof} An expression $p_s$ contains no function calls. Function calls is
what syntactically distinguishes expressions from patterns by the semantics of
\D{}.\end{proof}

For instance, consider an expression such as \mono{(g c.(f c))}, where \mono{c}
is some variable bound in the source clause. Assume that we're considering the
call to the function \mono{g}. We may replace the expression \mono{c.(f c))}
with an expression like \mono{c.d} where $\textt{d}\in V^s_{calls}$. When this
argument evaluates to some value $b_s\in\mathbb{B}$, then we can deduce the set
of safe displacement estimates $\{b_s>\text{\mono{c}}\}$.

Consider now a target clause with a pattern like \mono{x.0}. The question
henceforth is how do we draw the relationship that $\textt{c}\equiv\textt{x}$.
In other words, that the call neither decreases nor increases the call cycle
value. We can perform a corresponding analysis on the pattern declaration and
deduce the set of conditions that will hold if pattern matching succeeds,
indeed, $\{b_s\geq \textt{x}\}$. The participation of \mono{x} in the same kind
of relations as \mono{c}, does not alone imply that $\textt{c}\equiv\textt{x}$,
since the property that $b_s\equiv\textt{c.d}$ is lost.

On the other hand, if we had to formally define the relation that had to be
built between the bound variable of a source clause and the value that a
function call argument evaluated to, this would be a relation between values
and some kind of ``abstract patterns'', i.e. shapes with annotated triangle
shapes, as e.g. $b_s\geq\textt{c.0}$.

To simplify the entire process, instead of deducing actual size relations
between the bound variable and actual argument values, we can simply turn the
argument into such an abstract pattern to begin with. The actual size relations
are hence kept and can be deduced at a later stage in the process.

%Indeed, the tuple $\left\langle p_s,v^s,V_{calls}^s \right\rangle : \mathbb{P}
%\times \mathbb{V} \times [\mathbb{V}]$ constitutes such an abstract pattern
%already.

\subsubsection{Pattern matching}

\begin{definition} Given a call graph $G^1= \left\langle C,E \right\rangle$ for
each $\left\langle c_s=\_, c_t=\left\langle \_,p_t,\_\right\rangle,
x_s\right\rangle \in E$, by \referToDefinition{source-variable} and
\referToDefinition{clause-variable}, we can construct the tuples $\left\langle
p^s,v^s,V_{calls}^s \right\rangle$ and $\left\langle p^t,v^t \right\rangle$. We
hence assume for these tuples to be readily available for any given call
$\left\langle c_s, c_t, x_s \right\rangle$.\end{definition}

%\begin{definition} Let the function $\phi\ :\ \mathbb{V} \times \mathbb{V}
%\rightarrow \{<,\leq,\bot\}$ denote the function $\lambda N^t, N^s .
%\Phi\left(C^t,C^s,N^t,N^s\right)$.\end{definition}

%In the following section we will discuss the rules involved in deducing the
%function $\phi$, that is, the function $\Phi$ for some given source and target
%of a success transition.

%For this purpose we will regard the tuples $(P^s,N_{vars}^s)$ and
%$(P^t,N_{vars}^t)$, of a given success transition, where $P^s$ is the list of
%abstract patterns derived from the function arguments in the source, and $P^t$
%is the list of corresponding actual patterns in the target. Furthermore, let
%$N_{vars}^s$ and $N_{vars}^t$ be unary functions of the type
%$\mathbb{P}\rightarrow\mathbb{N}^*$, accepting a pattern and yielding the
%variable names that are contained both in the input pattern and the sets
%$N_{vars}^s$ and $N_{vars}^t$, respectively.

%In the following analysis we will look at but one instance of the lists $P^s$
%and $P^t$, namely the abstract pattern $p^s$ from the source and its
%corresponding actual pattern in the declaration, $p^t$. In total, however, this
%process has to be repeated for each such pair given the sets $P^s$ and $P^t$,
%iteratively extending the definition of the relation $\phi$ to all variables
%bound in the sets $N_{vars}^s$ and $N_{vars}^t$.

We initially define $\phi$ to yield the value $\bot$ for all arguments. We will
continuously modify this definition as we process $p^s$ and $p^t$. We denote
this within the semantics in a manner similar to the state $\sigma$ in the
semantics\footnote{See \referToSection{d-sos}.}. However, $\phi$ is now a
binary ``memory'', requiring both a target name and a source name (in that
order). For simplicity, we will borrow some suger coding from the matlab
notation which allows us to provide a collection in place of a single element
and let the runtime apply the given function to each element in the collection.
For instance, we might write that $\phi\left(N_{vars}^t(p^t), n^s\right)\mapsto
<$, meaning that all the target variables used in $p^t$ are strictly less than
the source variable $n^s$.

We now define a summoning rule, dividing the rules up into sub-rules:

\begin{equation}
{
    \left\langle\proc{A},p^t,p^s,\phi\right\rangle
    \rightarrow
    \phi'
  \vee
    \left\langle\proc{B},p^t,p^s,\phi\right\rangle
    \rightarrow
    \phi'
  \vee
    \left\langle\proc{C},p^t,p^s,\phi\right\rangle
    \rightarrow
    \phi'
  \vee
    \left\langle\proc{D},p^t,p^s,\phi\right\rangle
    \rightarrow
    \phi'
  \vee
    \left\langle\proc{E},p^t,p^s,\phi\right\rangle
    \rightarrow
    \phi'
}\over{
  \left\langle p^t,p^s,\phi\right\rangle
  \rightarrow
  \phi'
}
\end{equation}

One of the simpler cases is when the abstract pattern $p^s$ is simply \mono{0},
or some name $n^s$, and $n^s\in N_{calls}^s$. Since no variables bound in the
source participate in $p^s$, then no relations need to be drawn to any of the
target variables that might appear in the corresponding $p^t$. Hence, $\phi$
need not be modified.

\begin{equation}\label{eq:sct-pattern-source-fail}
{
\left(
    p^s\rightarrow 0
  \vee
\left(
    p^s\rightarrow n^s
  \wedge
    n^s\notin N_{vars}^s
\right)
\right)
  \wedge
    \phi\rightarrow\phi'
}\over{
  \left\langle\proc{A},p^t,p^s,\phi\right\rangle
  \rightarrow
  \phi'
}
\end{equation}

This has a symmetrical case. Indeed when $p^t$ is neither a destruction, nor
any name $n^t$, that is, it is \mono{\_} or \mono{0}. This pattern contains no
variables, and hence  no relations need to be drawn from any of the variables
that might appear in the corresponding $p^s$. Hence, $\phi$ need not be
modified in such a case either.

\begin{equation}
{
\left(
    p^t\rightarrow 0
\vee
    p^t\rightarrow \_
\right)
  \wedge
    \phi\rightarrow\phi'
}\over{
  \left\langle\proc{B},p^t,p^s,\phi\right\rangle
  \rightarrow
  \phi'
}
\end{equation}

If $p^t$ is the name pattern $n^t$, the matters get a bit more complicated:

\begin{enumerate}

\item If $p^s$ is some node, then all the variables that occur in $p^s$, i.e.
$N_{vars}^s(p^s)$, will all be strictly less than $n^t$ by the semantics of
\D{}. However, we are not concerned with this relation, as we would like to
know when a value is decreased from source to target, and not, as in this case,
increased.

\item If $p^s$ is also some name pattern $n^s$, and  $n^s\in N^s_{vars}$, then
the values of these corresponding variables will be \emph{equivalent}. However,
we're not concerned with exact equivalence, and simply mark this relationship
with the weaker, but still sound relation, $\leq$:

\begin{equation}
{
    p^t\rightarrow n^t
  \wedge
    p^s\rightarrow n^s
  \wedge
    n^s\in N_{vars}^s
  \wedge
    \left\langle\phi\left(n^t, n^s\right)\mapsto \leq\right\rangle\rightarrow\phi'
}\over{
  \left\langle\proc{C},p^t,p^s,\phi\right\rangle
  \rightarrow
  \phi'
}
\end{equation}

\end{enumerate}

If $p^t$ is a destruction and $p^s$ is the variable name $n^s$, then we can safely say that
all the variables that occur in $p^t$, i.e. $N_{vars}^t(p^t)$, are all strictly less
than the variable in $n^s$:

\begin{equation}
{
    p^t\rightarrow p^t_1\cdot p^t_2
  \wedge
    p^s\rightarrow n^s
  \wedge
    n^s\in N_{vars}^s
  \wedge
    \left\langle\phi\left(N_{vars}^t(p^t), n^s\right)\mapsto <\right\rangle\rightarrow\phi'
}\over{
  \left\langle\proc{D},p^t,p^s,\phi\right\rangle
  \rightarrow
  \phi'
}
\end{equation}

If both $p^t$ and $p^s$ are a destructions, then the following recursive rule applies:

\begin{equation}
{
    p^t\rightarrow p^t_1\cdot p^t_2
  \wedge
    p^s\rightarrow p^s_1\cdot p^s_2
  \wedge
    \left\langle p^t_1, p^s_1, \phi\right\rangle
    \rightarrow
    \phi''
  \wedge
    \left\langle p^t_2, p^s_2, \phi''\right\rangle
    \rightarrow
    \phi'
}\over{
  \left\langle\proc{E},p^t,p^s,\phi\right\rangle
  \rightarrow
  \phi'
}
\end{equation}

\section{Graph annotation}

Hence, we can deduce from \referToListing{cfg-loop}, that when $f_1$ makes a
call to $f_0$ it does so with a value strictly less then its own argument,
i.e. the transition $f_1\rightarrow f_0$ strictly decreases a value. Visually
we will mark this with a $\downarrow$. The Lemmas \refer{lemma:d-pattern-leq}
and \refer{lemma:d-pattern-less} can be used to deduce the same sort of
relationship for the transitions $r_1\xrightarrow{0,1} r_0$ for
\referToListing{cfg-reverse}. These observations are summarised in
\referToFigure{sct-first}.

\begin{figure}[htbp!]
\centering

\subfigure[The program in \referToListing{cfg-loop}.]{
\begin{tikzpicture}[>=latex',line join=bevel,]

\pgfsetlinewidth{1bp}
\pgfsetcolor{black}

% Edge: F0 -> F1
\draw [->,dotted] (133.06bp,73.287bp) .. controls (136.46bp,68.306bp) and (137.79bp,63.325bp)  .. (133.09bp,48.766bp);

\draw (90bp,60bp) node {$\downarrow$};

% Edge: F1 -> F0
\draw [->] (100.91bp,48.766bp) .. controls (97.522bp,53.747bp) and (96.204bp,58.727bp)  .. (100.94bp,73.287bp);

% Node: F0
\begin{scope}
  \definecolor{strokecol}{rgb}{0.0,0.0,0.0};
  \pgfsetstrokecolor{strokecol}
  \draw (117bp,88bp) ellipse (27bp and 18bp);
  \draw (117bp,88bp) node {$f_0$};
\end{scope}

% Node: F1
\begin{scope}
  \definecolor{strokecol}{rgb}{0.0,0.0,0.0};
  \pgfsetstrokecolor{strokecol}
  \draw (117bp,34bp) ellipse (27bp and 18bp);
  \draw (117bp,34bp) node {$f_1$};
\end{scope}

\end{tikzpicture}
}

\subfigure[The program in \referToListing{cfg-reverse}.]{
\begin{tikzpicture}[>=latex',line join=bevel,]
\pgfsetlinewidth{1bp}
\pgfsetcolor{black}

% Edge: R0 -> R1
\draw [->,dotted] (49.969bp,34.268bp) .. controls (56.882bp,36.851bp) and (64.635bp,39.282bp)  .. (72bp,40.58bp) .. controls (79.767bp,41.948bp) and (88bp,40.866bp)  .. (105.24bp,35.467bp);

% Edge: R1 -> R0
\draw [->] (99.957bp,22.51bp) .. controls (94.056bp,22.123bp) and (87.816bp,21.778bp)  .. (82bp,21.58bp) .. controls (76.252bp,21.384bp) and (70.164bp,21.454bp)  .. (53.809bp,22.204bp);

\definecolor{strokecol}{rgb}{0.0,0.0,0.0};
\pgfsetstrokecolor{strokecol}

\draw (76bp,29.08bp) node {$0$};
\draw (83bp,29.08bp) node {$\downarrow$};

% Edge: R1 -> R0
\draw [->] (108.98bp,11.053bp) .. controls (98.589bp,4.4007bp) and (84.865bp,-1.5798bp)  .. (72bp,1.5798bp) .. controls (66.559bp,2.9161bp) and (61.044bp,5.0566bp)  .. (46.908bp,12.145bp);

\draw (76bp,9.0798bp) node {$1$};
\draw (83bp,9.0798bp) node {$\downarrow$};

% Node: R0
\begin{scope}
  \definecolor{strokecol}{rgb}{0.0,0.0,0.0};
  \pgfsetstrokecolor{strokecol}
  \draw (27bp,25bp) ellipse (27bp and 18bp);
  \draw (27bp,24.58bp) node {$r_0$};
\end{scope}

% Node: R1
\begin{scope}
  \definecolor{strokecol}{rgb}{0.0,0.0,0.0};
  \pgfsetstrokecolor{strokecol}
  \draw (127bp,25bp) ellipse (27bp and 18bp);
  \draw (127bp,24.58bp) node {$r_1$};
\end{scope}
%
\end{tikzpicture}
}

\caption[]{Call graphs with annotated edges for various programs.}

\label{figure:sct-first}

\end{figure}



\section{The algorithm}

We define the size-change termination algorithm as follows:

\begin{definition}\label{definition:size-change-algorithm} Given a program $r$
and its corresponding size-change graph $G$, yield ``halts'' if all the call
cycles in $G$ are monotonically decreasing, and ``unknown''
otherwise.\end{definition}
