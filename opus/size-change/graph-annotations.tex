\section{Graph annotation}

Given the discussion in \referToSection{size-change-principle}, we can now
deduce the calls from the recursive clause of the program in
\referToListing{cfg-reverse} are both decreasing, yielding the conclusion that
the program \mono{reverese} terminates.

Graphically, we won't define any special syntax for functions that have clauses
where more than one variable is bound, instead, we will merely define a syntax
for the set of size relations $\{<,\leq,\bot\}$. \referToFigure{sct-reverse-pdf}
shows an example.

\begin{definition} Given a call graph $G^1 = \left\langle C^1, E^1
\right\rangle$, and its deduced size relation $\Phi^1$, let the graph $G^s =
\left\langle C^1, \Phi^1 \right\rangle$ be known as a size-change graph. Note,
that by the rules defined in \referToSection{size-change-final} and
\referToDefinition{size-relation}, $$\left(\forall\ \left\langle c_s, c_t, x
\right\rangle \in E^1\ \exists\ \left\langle c_s, c_t, x, \_ \right\rangle \in
\Phi^1\right) \wedge \left(\exists\ \left\langle c_s, c_t, x, \rho
\right\rangle \in \Phi^1\longrightarrow \nexists \left\langle c_s, c_t, x,
\rho' \right\rangle \in \Phi^1\ \rho\neq\rho' \right).$$\end{definition}

\includeFigure[scale=1.5]{sct-reverse-pdf}{A size-change graph for the program
in \referToListing{cfg-reverse}.}
