\section{Size-change termination
principle}\label{section:size-change-principle}

Consider the program in \referToListing{cfg-loop} and its corresponding call
graph in \referToFigure{cfg-loop-pdf}. Without any further information about
the calls, the program seemingly loops indefinitely. However,
there are some things that we can deduce about the control transitions.

\begin{theorem}\label{theorem:size-change} If every call cycle in a given
program reduces a value of a well-founded data-type on each iteration of the
cycle, then the value must eventually bottom out and the program must
terminate.\end{theorem}

\begin{proof} Assume for the sake of contradiction that a program that reduces
a value of a well-founded data type in each call cycle does not terminate.
Then, either the value reduces indefinitely, which is a contradiction to the
well-foundedness of its data type, or some noncyclic call sequence causes an
infinite loop, also an absurdity due to the definition of \D{}. \end{proof}

That is the \emph{size-change termination principle}\cite{size-change}. All
values in \D{} are inherently well-founded so what remains to be shown is how
we can deduce from a call cycle whether it reduces a value on each iteration.

\begin{definition}\label{definition:size-relation} For a given call
graph $G = \left\langle C,E \right\rangle$, let a size relation be the set

$$
\Phi = \left\{ \left\langle c_s, c_t,x, v_s, v_t, \rho \right\rangle \left| 
\begin{array}{ll}
&\left\langle c_s, c_t,x \right\rangle \in E\\
\wedge&c_t = \left\langle \_, p_t, \_ \right\rangle \in C \\
\wedge&v_s,v_t\in\mathbb{V}\\
\wedge&v_s\Subset x \wedge v_t \Subset p_t \\
\wedge&\rho\in\{\bot, <, \leq\}
\end{array}
\right.\right\}
$$

\end{definition}

This definition implies that for each call cycle there may be multiple
variables\footnote{We'll define what we mean by a variable more formally in
\referToDefinition{variable}.} to ensure reduction for. However, as there is
only a finite number of variables in any given pattern and any given
expression, we can define these as separate cycles.

\begin{definition}\label{definition:variable-call-graph} Given a recursive call
graph $G^r = \left\langle C^r, E^r \right\rangle$, let $G^u= \left\langle C^u,
E^u \right\rangle$ be a ``unary recursive call graph'', where

$$E^u = \left\{ \left\langle c_s, c_t, x, v_s, v_t \right\rangle \left|
\begin{array}{ll}
&\left\langle c_s, c_t,x \right\rangle \in E^r\\
\wedge&c_t = \left\langle \_, p_t, \_ \right\rangle \in C^r \\
\wedge&v_s,v_t\in\mathbb{V}\\
\wedge&v_s\Subset x \wedge v_t \Subset p_t
\end{array}
\right.\right\},$$

and $C^u= \left\{ c \mid c \in C^r \wedge c\Subset E^u
\right\}$.\end{definition}

\begin{theorem} A unary recursive call graph $G^u$ has a finite number of
edges.\end{theorem}

\begin{proof} Follows from \referToDefinition{variable-call-graph} and the
semantics of \D{}.\end{proof}

\referToDefinition{variable-call-graph} forces us to redefine the concept of a
call cycle, as a sequence of clauses may have multiple cycles in $G^u$.

\begin{definition}\label{definition:variable-call-cycle} Given a recursive call
graph $G^r = \left\langle C^r, E^r \right\rangle$, let $Z^r$ denote the set of
call cycles in $G^r$, and $Z^u$ the set of call cycles in $G^u$, then

$$Z^u = \bigcup_{z^r\in Z^r} \left\{ \left\langle c_s, c_t, x, v_s, v_t \right\rangle \left|
\begin{array}{ll}
&\left\langle c_s, c_t,x \right\rangle \in z^r\\
\wedge&c_t = \left\langle \_, p_t, \_ \right\rangle \in C^r \\
\wedge&v_s,v_t\in\mathbb{V}\\
\wedge&v_s\Subset x \wedge v_t \Subset p_t
\end{array}
\right.\right\}.$$

\end{definition}

\referToDefinition{variable-call-graph} allows us now to more formally define
what we've thus far meant as a variable that changes value from call to call.

\begin{definition}\label{definition:variable} Given a unary recursive call
graph $G^u$, every call cycle $z$ in $G^u$, changes exactly one variable, call
it ``cycle variable'', or $v_z$.\end{definition}

Hence, while a variable may have different names in different clauses, we've
defined an overall variable for every call cycle, and want to ensure that this
variable is reduced in every iteration of the call cycle.

\begin{theorem}\label{theorem:multivariable-patterns} We can WLOG to
termination analysis limit our attention to programs that bind at most one
variable in every clause.\end{theorem}

\begin{proof} Let a recursive call graph $G^r$ have a set of call cycles $Z^r$,
and a corresponding unary recursive call graph $G^u$ with a set of call cycles
$Z^u$. By \referToDefinition{variable-call-cycle} and
\referToDefinition{size-relation}, every call cycle in $Z^r$ reduces a value
iff every call cycle in $Z^u$ reduces a value. We can hence limit our attention
to deducing if each call cycle in $Z^u$ reduces a value.\end{proof}

\begin{definition}\label{definition:nice-call-graph} Let $G^1$ denote a
recursive call graph of a program where each clause binds at most one
variable.\end{definition}

Given \referToTheorem{multivariable-patterns} we can return to the old
definition of a call graph as per \referToDefinition{recursive-call-graph} and
call cycle as per \referToDefinition{call-cycle}. This demands a
simplification of the size relation $\Phi$.

\begin{definition}\label{definition:unary-size-relation} For a given $G^1 =
\left\langle C,E \right\rangle$ of a program, let a size relation be the set

$$\Phi^1 = \left\{ \left\langle c_s, c_t,x, \rho \right\rangle \mid
\left\langle c_s, c_t,x \right\rangle \in E \wedge \rho\in\{\bot, <, \leq\}
\right\}.$$

\end{definition}

\begin{definition}\label{definition:increasing-decreasing-call} Given a call
graph $G$ with a call cycle $z$, a call $\left\langle c_s,c_t,x\right\rangle\in
z$ is,

\begin{enumerate}

\item A ``decreasing call'' iff $\left\langle c_s,c_t,x,< \right\rangle \in
\Phi^1$.

\item A ``nonincreasing call'' iff $\left\langle c_s,c_t,x,\leq \right\rangle
\in \Phi^1$.

\item An ``undteremined call'' iff $\left\langle c_s,c_t,x,\bot \right\rangle
\in \Phi^1$.

\end{enumerate}

\end{definition}

\begin{lemma}\label{lemma:cycle-reduce} A call cycle $z= \left\langle c_1,c_2
\right\rangle, \left\langle c_2, c_3 \right\rangle,\ldots, \left\langle
c_{n-1}, c_n \right\rangle$ reduces a value on each iteration iff
$$\left(\forall\ \left\langle c_i,c_j \right\rangle \in z\ \left(\left\langle
c_i,c_j,\_, < \right\rangle \in \Phi^1\right) \vee \left(\left\langle
c_i,c_j,\_, \leq \right\rangle \in \Phi^1 \right)\right)\wedge \left( \exists\
\left\langle c_i,c_j \right\rangle \in z\ \left\langle c_i,c_j,\_, <
\right\rangle \in \Phi^1 \right).$$\end{lemma}

\begin{proof} If a value is not reduced in a cycle, it either stays the same or
is increased. If it is increased, then at least one call must've increased the
value, which is an absurdity. If it stays the same then none of the
participating control transitions have neither increased nor decreased the
value, also an absurdity.\end{proof}

\subsection{Deducing $\Phi^1$}

\begin{definition} Given a call graph $G^1 = \left\langle C^1,
E^1\right\rangle$ and a call $\left\langle c_s = \left\langle \_,p_s,\_
\right\rangle, c_t = \left\langle \_, p_t, \_ \right\rangle, x \right\rangle
\in E^1$, let $v_{p_s}$ denote the variable s.t. $v_{p_s}\in\mathbb{V}\wedge
v_{p_s}\Subset p_s$, and let $v_{p_t}$ denote the variable s.t.
$v_{p_t}\in\mathbb{V}\wedge v_{p_t}\Subset p_t$. If some $p\in\mathbb{P}$
contains no variables, then $\forall\ v\in\mathbb{V}\ v\neq
v_p$.\end{definition}

\begin{definition} Given a call graph $G^1 = \left\langle C^1,
E^1\right\rangle$ and a call $\left\langle \_, \_, x \right\rangle \in E^1$,
let $b_x\in\mathbb{B}$ denote the actual value $x$ evaluates
to.\end{definition}

This is a three-step process, for any given call $\left\langle c_s, c_t, x
\right\rangle \in E^1$, \begin{inparaenum}[(1)]\item \label{nice-1} deduce a
size relation between $v_{p_s}$ and $b_x$, \item \label{nice-2} deduce a size
relation between $b_x$ and $v_{p_t}$, \item given (\ref{nice-1}) and
(\ref{nice-2}), deduce a size relation between $v_{p_s}$ and
$v_{p_t}$\end{inparaenum}, allowing us to update the relevant $\Phi^1$.

In the discussions below we assume to be considering a single call $e\in E^1$,
hence the variables $c_s$, $c_t$, $x$, $b_x$, $p_s$, $p_t$, $v_{p_s}$,
$v_{p_t}$ and $V_x$ are available without further details.

\subsubsection{Source to argument}

Since \D{} is a call-by-value language, the source evaluates $x$, and generates some
$b_x\in\mathbb{B}$ as the actual argument for the call. The expression $x$ is
by definition a nested construction of either some concrete value, some
variable $v_{p_s}$, or a nested function call. Without further regard of nested
function calls, this implies that a size relation can be deduced between the
$v_{p_s}$ and $b_x$.

We decide to ignore the nested function calls because this would imply a more
complex static analysis of the program. Specifically, we're unable to say
anything about the result of the nested function call from the scope of $c_s$
alone. Instead, we treat results from nested function calls simply as variables
with \emph{unknown} values. We also make sure to keep these variables separate
from the bound variables as there is no relationship to draw between these
auxiliary variables and the variable $v_{p_t}$.

Due to nested functions calls being represented as variables with unknown
values, a precise size displacement between the bound variable $v_{p_s}$ and
the generated argument $b$ is not feasible. However, we can deduce a
\emph{safe} displacement estimate.

\begin{definition} A safe displacement estimate between the values
$b_1,b_2\in\mathbb{B}$ is a value $n\in\mathbb{N}$ s.t. $b_1+n\leq
b_2$.\end{definition}

\begin{definition}\label{definition:source-variable} Given a call $\left\langle
c_s,c_t,x\right\rangle$, we construct the expression $p_x$ where we replace all
$\left\langle \_,\_ \right\rangle\Subset x$ with an auxiliary variable $v\neq
v_{p_s}$. We group those auxiliary variables in the set $V_x$, that is,\\ $V_x =
\left\{ v \mid v\in\mathbb{V} \wedge v\Subset p_x \wedge v \neq v_{p_s}
\right\}$.\end{definition} 

\begin{lemma} Given a call $\left\langle c_s,c_t,x\right\rangle$, the
expression $p_x$ is interchangeable with a pattern. Hence, we say that
$p_x\in\mathbb{P}.$\end{lemma}

\begin{proof} The expression $p_x$ contains no function calls. Function calls
is what syntactically distinguishes expressions from patterns by the semantics
of \D{}.\end{proof}

For instance, consider an expression $x$ such as \mono{(g c.(f c))}, where
$v_{p_s}=\textt{c}$. Assume that we're considering the call to the function
\mono{g}. By \referToDefinition{source-variable}, we replace the expression
\mono{c.(f c))} with an expression like \mono{c.d}, and let $V_x = \left\{
\textt{d} \right\}$. When this argument evaluates to some value
$b_x\in\mathbb{B}$, then we can deduce the set of safe displacement estimates
$\{b_x>\text{\mono{c}}\}$.

Consider now a target clause with a pattern like \mono{e.0}. The question
henceforth is how do we draw the relationship that $\textt{c}=\textt{e}$. In
other words, that the call neither decreases nor increases the call cycle
value. We can perform a corresponding analysis on the pattern declaration and
deduce the set of conditions that will hold if pattern matching succeeds,
indeed, $\{b_x\geq \textt{e}\}$. The participation of \mono{c} in the same kind
of relations as \mono{e}, does not alone imply that $\textt{c}=\textt{e}$,
since the property that $b_x=\textt{c.d}$, where $\textt{d}\geq 0$, is lost.

On the other hand, if we had to formally define the relation that had to be
drawn between $v_{p_s}$ and $b_x$ for any given $\left\langle c_s,c_t,x
\right\rangle$, this would be a relation between $b_s$ and some sort of
``abstract patterns'', as e.g. $b_s\geq\textt{c.0}$ where $\textt{c}\geq 0$.

To simplify the entire process, instead of deducing actual size relation
between $v_{p_s}$ and $b_x$, we can simply turn $x$ into such an abstract
pattern to begin with (\referToDefinition{source-variable}). The actual size
relations are hence kept and can be deduced at a later stage in the process.

%Indeed, the tuple $\left\langle p_s,v^s,V_{calls}^s \right\rangle : \mathbb{P}
%\times \mathbb{V} \times [\mathbb{V}]$ constitutes such an abstract pattern
%already.

\subsubsection{Source to target}\label{section:size-change-final}

%\begin{definition} Given a call graph $G^1= \left\langle C^1,E^1 \right\rangle$
%for each $\left\langle c_s=\_, c_t=\left\langle \_,p_t,\_\right\rangle,
%x_s\right\rangle \in E$, by \referToDefinition{source-variable} and
%\referToDefinition{clause-variable}, we can construct the tuples $\left\langle
%p^s,v^s,V_{calls}^s \right\rangle$ and $\left\langle p^t,v^t \right\rangle$. We
%hence assume for these tuples to be readily available for any given call
%$\left\langle c_s, c_t, x_s \right\rangle$.\end{definition}

%\begin{definition} Let the function $\phi\ :\ \mathbb{V} \times \mathbb{V}
%\rightarrow \{<,\leq,\bot\}$ denote the function $\lambda N^t, N^s .
%\Phi\left(C^t,C^s,N^t,N^s\right)$.\end{definition}

%In the following section we will discuss the rules involved in deducing the
%function $\phi$, that is, the function $\Phi$ for some given source and target
%of a success transition.

%For this purpose we will regard the tuples $(P^s,N_{vars}^s)$ and
%$(P^t,N_{vars}^t)$, of a given success transition, where $P^s$ is the list of
%abstract patterns derived from the function arguments in the source, and $P^t$
%is the list of corresponding actual patterns in the target. Furthermore, let
%$N_{vars}^s$ and $N_{vars}^t$ be unary functions of the type
%$\mathbb{P}\rightarrow\mathbb{N}^*$, accepting a pattern and yielding the
%variable names that are contained both in the input pattern and the sets
%$N_{vars}^s$ and $N_{vars}^t$, respectively.

%In the following analysis we will look at but one instance of the lists $P^s$
%and $P^t$, namely the abstract pattern $p^s$ from the source and its
%corresponding actual pattern in the declaration, $p^t$. In total, however, this
%process has to be repeated for each such pair given the sets $P^s$ and $P^t$,
%iteratively extending the definition of the relation $\phi$ to all variables
%bound in the sets $N_{vars}^s$ and $N_{vars}^t$.

In the section below we will focus on defining a set of rules that allow us to
deduce a relation between $v_{p_t}$ and $v_{p_t}$, directly given some $p_x$
which has been constructed as by \referToDefinition{source-variable}.

\begin{definition} Initially, let $\left\langle c_s,c_t,x, \bot\right\rangle
\in \Phi^1$.\end{definition}

In the discussion below, we use $\Phi^1$ in a manner similar to the state
$\sigma$ in the definition of the semantics of \D{} in \referToSection{d-sos}.
However, $\Phi^1$ is now a binary memory element, requiring both a source name
and a target name (in that order).

In particular, we would like a set of rules for modifying $\Phi^1$, given some
argument expression $p_x$ and some target pattern $p_t$. We start by defining a
summoning rule, dividing the rules up into sub-rules:

\begin{equation}
{\displaystyle
\begin{array}{ll}
&   \left\langle\proc{A},p_x,p_t,\Phi^1\right\rangle
    \rightarrow
    \Phi^1_1\\
  \vee&
    \left\langle\proc{B},p_x,p_t,\Phi^1\right\rangle
    \rightarrow
    \Phi^1_1\\
  \vee&
    \left\langle\proc{C},p_x,p_t,\Phi^1\right\rangle
    \rightarrow
    \Phi^1_1\\
  \vee&
    \left\langle\proc{D},p_x,p_t,\Phi^1\right\rangle
    \rightarrow
    \Phi^1_1\\
  \vee&
    \left\langle\proc{E},p_x,p_t,\Phi^1\right\rangle
    \rightarrow
    \Phi^1_1
\end{array}
\over\displaystyle
  \left\langle p_x,p_t,\Phi^1\right\rangle
  \rightarrow
  \Phi^1_1
}
\end{equation}

One of the simpler cases is when the abstract pattern $p_x$ is simply \mono{0},
or $v\in V_x$. Since $v_{p_s}$ does not participate in $p_x$, then no relation
needs to be drawn between $v_{p_s}$ and $v_{p_t}$ because no such relation
exists.

\begin{equation}\label{eq:sct-pattern-source-fail}
{
\left(
    p_x = 0
  \vee
\left(
    p_x = v
  \wedge
    v\neq v_{p_s}
\right)
\right)
  \wedge
    \Phi^1\rightarrow\Phi^1_1
}\over{
  \left\langle\proc{A},p_x,p_t,\Phi^1\right\rangle
  \rightarrow
  \Phi^1_1
}
\end{equation}

This has a symmetrical case. Indeed when $p_t$ is neither a destruction, nor
any name $v_{p_t}$, that is, it is either \mono{\_} or \mono{0}. In this case,
no relation needs to be drawn between $v_{p_t}$ and $v_{p_s}$ as such a
relationship is absent.

\begin{equation}
{
\left(
    p_t = 0
\vee
    p_t = \_
\right)
  \wedge
    \Phi^1\rightarrow\Phi^1_1
}\over{
  \left\langle\proc{B},p_x,p_t,\Phi\right\rangle
  \rightarrow
  \Phi^1_1
}
\end{equation}

If $p_t$ is the name pattern $v_{p_t}$, the matters get a bit more complicated:

\begin{enumerate}

\item If $p_x$ is some node, then all the variables that occur in $p_x$, will
be strictly less than $v_{p_t}$ by the semantics of \D{}. However, we are not
concerned with this relation, as this indeed would be an increasing call rather
than a decreasing or nonincreasing call.

\item If $p_x$ is $v_{p_s}$, then the values of these corresponding variables
will be \emph{equivalent}. However, we're not concerned with exact equivalence,
and simply mark this relationship with the weaker, but still sound relation,
$\leq$:

\begin{equation}
{
    p_t = v_{p_t}
  \wedge
    p_x = v_{p_s}
  \wedge
    \left\langle\Phi^1\left(c_s, c_t, x\right)\mapsto \leq\right\rangle\rightarrow\Phi^1_1
}\over{
  \left\langle\proc{C},p_x,p_t,\Phi^1\right\rangle
  \rightarrow
  \Phi^1_1
}
\end{equation}

\end{enumerate}

If $p_t$ is a destruction and $p_x = v_{p_x}$, then we can safely say that all
the variables that occur in $p_t$, are all strictly less than the variable in
$v_{p_x}$:

\begin{equation}
{
    p_t = p_{t_1}\cdot p_{t_2}
  \wedge
    p_x = v_{p_s}
  \wedge
    v_{p_t}\Subset{p_t}
  \wedge
    \left\langle\Phi^1\left(c_s, c_t, x\right)\mapsto <\right\rangle\rightarrow\phi^1_1
}\over{
  \left\langle\proc{D},p_x,p_t,\Phi^1\right\rangle
  \rightarrow
  \Phi^1_1
}
\end{equation}

If both $p^t$ and $p^s$ are a destructions, then the following recursive rule applies:

\begin{equation}
{
    p_t = p_{t_1}\cdot p_{t_2}
  \wedge
    p_s = p_{x_1}\cdot p_{x_2}
  \wedge
    \left\langle p_{t_1}, p_{x_1}, \Phi^1\right\rangle
    \rightarrow
    \Phi^1_2
  \wedge
    \left\langle p_{t_2}, p_{x_2}, \Phi^1_2\right\rangle
    \rightarrow
    \Phi^1_1
}\over{
  \left\langle\proc{E},p_x,p_t,\Phi^1\right\rangle
  \rightarrow
  \Phi^1_1
}
\end{equation}
