\section{Size-change termination principle}

Consider the program in \referToListing{cfg-loop} and its corresponding call
graph in \referToFigure{cfg-loop-pdf}. Without any further information about
the calls, the program seemingly loops indefinitely. However,
there are some things that we can deduce about the control transitions.

\begin{theorem}\label{theorem:size-change} If every call cycle in a given
program reduces a value of a well-founded data-type on each iteration of the
cycle, then the value must eventually bottom out and the program must
terminate.\end{theorem}

\begin{proof} Assume for the sake of contradiction that a program that reduces
a value of a well-founded data type in each call cycle does not terminate.
Then, either the value reduces indefinitely, which is a contradiction to the
well-foundedness of its data type, or some noncyclic call sequence causes an
infinite loop, also an absurdity due to the definition of \D{}. \end{proof}

That is the \emph{size-change termination principle}\cite{size-change}. All
values in \D{} are inherently well-founded so what remains to be shown is how
we can deduce from a call cycle whether it reduces a value on each iteration.

\begin{definition}\label{definition:size-relation} For a given call
graph $G = \left\langle C,E \right\rangle$, let a size relation be the set

$$
\Phi = \left\{ \left\langle c_s, c_t,x, v_s, v_t, \rho \right\rangle \left| 
\begin{array}{ll}
&\left\langle c_s, c_t,x \right\rangle \in E\\
\wedge&c_t = \left\langle \_, p_t, \_ \right\rangle \in C \\
\wedge&v_s,v_t\in\mathbb{V}\\
\wedge&v_s\Subset x \wedge v_t \Subset p_t \\
\wedge&\rho\in\{\bot, <, \leq\}
\end{array}
\right.\right\}
$$

\end{definition}

This definition implies that for each call cycle there may be multiple
variables\footnote{We'll define what we mean by a variable more formally in
\referToDefinition{variable}.} to ensure reduction for. However, as there is
only a finite number of variables in any given pattern and any given
expression, we can define these as separate cycles.

\begin{definition}\label{definition:variable-call-graph} Given a recursive call
graph $G^r = \left\langle C^r, E^r \right\rangle$, let $G^u= \left\langle C^u,
E^u \right\rangle$ be a ``unary recursive call graph'', where

$$E^u = \left\{ \left\langle c_s, c_t, x, v_s, v_t \right\rangle \left|
\begin{array}{ll}
&\left\langle c_s, c_t,x \right\rangle \in E^r\\
\wedge&c_t = \left\langle \_, p_t, \_ \right\rangle \in C^r \\
\wedge&v_s,v_t\in\mathbb{V}\\
\wedge&v_s\Subset x \wedge v_t \Subset p_t
\end{array}
\right.\right\},$$

and $C^u= \left\{ c \mid c \in C^r \wedge c\Subset E^u
\right\}$.\end{definition}

\begin{theorem} A unary recursive call graph $G^u$ has a finite number of
edges.\end{theorem}

\begin{proof} Follows from \referToDefinition{variable-call-graph} and the
semantics of \D{}.\end{proof}

\referToDefinition{variable-call-graph} forces us to redefine the concept of a
call cycle, as a sequence of clauses may have multiple cycles in $G^u$.

\begin{definition}\label{definition:variable-call-cycle} Given a recursive call
graph $G^r = \left\langle C^r, E^r \right\rangle$, let $Z^r$ denote the set of
call cycles in $G^r$, and $Z^u$ the set of call cycles in $G^u$, then

$$Z^u = \bigcup_{z^r\in Z^r} \left\{ \left\langle c_s, c_t, x, v_s, v_t \right\rangle \left|
\begin{array}{ll}
&\left\langle c_s, c_t,x \right\rangle \in z^r\\
\wedge&c_t = \left\langle \_, p_t, \_ \right\rangle \in C^r \\
\wedge&v_s,v_t\in\mathbb{V}\\
\wedge&v_s\Subset x \wedge v_t \Subset p_t
\end{array}
\right.\right\}.$$

\end{definition}

\referToDefinition{variable-call-graph} allows us now to more formally define
what we've thus far meant as a variable that changes value from call to call.

\begin{definition}\label{definition:variable} Given a unary recursive call
graph $G^u$, every call cycle $z$ in $G^u$, changes exactly one variable, call
it ``cycle variable'', or $v_z$.\end{definition}

Hence, while a variable may have different names in different clauses, we've
defined an overall variable for every call cycle, and want to ensure that this
variable is reduced in every iteration of the call cycle.

\begin{theorem}\label{theorem:multivariable-patterns} We can WLOG to
termination analysis limit our attention to programs that bind at most one
variable in every clause.\end{theorem}

\begin{proof} Let a recursive call graph $G^r$ have a set of call cycles $Z^r$,
and a corresponding unary recursive call graph $G^u$ with a set of call cycles
$Z^u$. By \referToDefinition{variable-call-cycle} and
\referToDefinition{size-relation}, every call cycle in $Z^r$ reduces a value
iff every call cycle in $Z^u$ reduces a value. We can hence limit our attention
to deducing if each call cycle in $Z^u$ reduces a value.\end{proof}

\begin{definition}\label{definition:nice-call-graph} Let $G^1$ denote a
recursive call graph of a program where each clause binds at most one
variable.\end{definition}

Given \referToTheorem{multivariable-patterns} we can return to the old
definition of a call graph as per \referToDefinition{recursive-call-graph} and
call cycle as per \referToDefinition{call-cycle}. This demands a
simplification of the size relation $\Phi$.

\begin{definition}\label{definition:unary-size-relation} For a given $G^1 =
\left\langle C,E \right\rangle$ of a program, let a size relation be the set

$$\Phi^1 = \left\{ \left\langle c_s, c_t,x, \rho \right\rangle \mid
\left\langle c_s, c_t,x \right\rangle \in E \wedge \rho\in\{\bot, <, \leq\}
\right\}$$

\end{definition}

\begin{definition}\label{definition:increasing-decreasing-call} Given a call
graph $G$ with a call cycle $z$, a call $\left\langle c_s,c_t,x\right\rangle\in
z$ is,

\begin{enumerate}

\item A ``decreasing call'' iff $\left\langle c_s,c_t,x,< \right\rangle \in
\Phi^1$.

\item A ``nonincreasing call'' iff $\left\langle c_s,c_t,x,\leq \right\rangle
\in \Phi^1$.

\item An ``undteremined call'' iff $\left\langle c_s,c_t,x,\bot \right\rangle
\in \Phi^1$.

\end{enumerate}

\end{definition}

\begin{lemma}\label{lemma:cycle-reduce} A call cycle $z= \left\langle c_1,c_2
\right\rangle, \left\langle c_2, c_3 \right\rangle,\ldots, \left\langle
c_{n-1}, c_n \right\rangle$ reduces a value on each iteration iff
$$\left(\forall\ \left\langle c_i,c_j \right\rangle \in z\ \left(\left\langle
c_i,c_j,\_, < \right\rangle \in \Phi^1\right) \vee \left(\left\langle
c_i,c_j,\_, \leq \right\rangle \in \Phi^1 \right)\right)\wedge \left( \exists\
\left\langle c_i,c_j \right\rangle \in z\ \left\langle c_i,c_j,\_, <
\right\rangle \in \Phi^1 \right)$$.\end{lemma}

\begin{proof} If a value is not reduced in a cycle, it either stays the same or
is increased. If it is increased, then at least one call must've increased the
value, which is an absurdity. If it stays the same then none of the
participating control transitions have neither increased nor decreased the
value, also an absurdity.\end{proof}

\subsection{Deducing $\Phi^1$}

Since \D{} is a call-by-value language, when a function call $\left\langle
v_t,x_s \right\rangle$ is encountered, the source evaluates $x_s$ and generates
some $b_s\in\mathbb{B}$ as the actual argument for the function call. The
expression $x_s$ is by definition a nested construction of either some concrete
value, some variable $v_s\in\mathbb{V}$ bound in the source, or a nested
function call. Without further regard of nested function calls, this implies
that a size relation can be deduced between the $v_s$ and $b_s$.

We decide to ignore the nested function calls because this would imply a more
complex static analysis of the program. Specifically, we're unable to say
anything about the result of the nested function call from the scope of the
source clause alone. Instead, we treat results from nested function calls
simply as variables with \emph{unknown} values. We also make sure to keep these
variables separate from the bound variables as there is no relationship to draw
between these ``variables'' and the variable that might be bound in the
target\footnote{While this information may be useful for dead-code elimination
and other forms of static analysis, this is of little importance to size-change
termination.}.

Due to nested functions calls being represented as variables with unknown
values, a precise size displacement between the bound variable $v_s$ and the
generated argument $b_s$ is not feasible. However, we can deduce a \emph{safe}
displacement estimate.

\begin{definition} A safe displacement estimate between the values
$b_1,b_2\in\mathbb{B}$ is a value $n\in\mathbb{N}$ s.t. $b_1+n\leq
b_2$.\end{definition}

\begin{definition} Given a function $\left\langle v_t,x_s\right\rangle$, we
construct the expression $p_s$ where we replace all $\left\langle v,x
\right\rangle\Subset p_s$ with an auxiliary variable. We group those auxiliary
variables in the set $V^s_{calls}$.\end{definition} 

\begin{lemma} An expression $p_s\in\mathbb{X}$ is interchangeable with a
pattern. Hence, we say that $p_s\in\mathbb{P}.$\end{lemma}

\begin{proof} An expression $p_s$ contains no function calls. Function calls is
what syntactically distinguishes expressions from patterns by the semantics of
\D{}.\end{proof}

For instance, consider an expression such as \mono{(g c.(f c))}, where \mono{c}
is some variable bound in the source clause. Assume that we're considering the
call to the function \mono{g}. We may replace the expression \mono{c.(f c))}
with an expression like \mono{c.d} where $\textt{d}\in V^s_{calls}$. When this
argument evaluates to some value $b_s\in\mathbb{B}$, then we can deduce the set
of safe displacement estimates $\{b_s>\text{\mono{c}}\}$.

Consider now a target clause with a pattern like \mono{x.0}. The question
henceforth is how do we draw the relationship that $\textt{c}\equiv\textt{x}$.
In other words, that the call neither decreases nor increases the call cycle
value. We can perform a corresponding analysis on the pattern declaration and
deduce the set of conditions that will hold if pattern matching succeeds,
indeed, $\{b_s\geq \textt{x}\}$. The participation of \mono{x} in the same kind
of relations as \mono{c}, does not alone imply that $\textt{c}\equiv\textt{x}$,
since the property that $b_s\equiv\textt{c.d}$ is lost.

On the other hand, if we had to formally define the relation that had to be
built between the bound variable of a source clause and the value that a
function call argument evaluated to, this would be a relation between values
and some kind of ``abstract patterns'', i.e. shapes with annotated triangle
shapes, as e.g. $b_s\geq\textt{c.0}$.

To simplify the entire process, instead of deducing actual size relations
between the bound variable and actual argument values, we can simply turn the
argument into such an abstract pattern to begin with. The actual size relations
are hence kept and can be deduced at a later stage in the process.

Indeed, the tuple $\left\langle p_s,v^s,V_{calls}^s \right\rangle : \mathbb{P}
\times \mathbb{V} \times [\mathbb{V}]$ constitutes such an abstract pattern
already.

\begin{definition} Given a clause $c_t=\left\langle \_,p_t,\_  \right\rangle
\in \mathbb{C}$, we refer to the only variable in $p_t$ as $v^t$, hence
$v^t\in\mathbb{V} \wedge v^t\Subset p_t$.\end{definition}

\subsection{Pattern matching}

Our task is to deduce a size relation between the variables in the sets
$N_{vars}^s$ and $N_{vars}^t$ given the tuples $(p^s,N_{vars}^s,N_{calls}^s)$
and $(p^t,N_{vars}^t)$.


Let the function $\phi\ :\ \mathbb{N} \times \mathbb{N} \rightarrow
\{<,\leq,\bot\}$ denote the function $\lambda N^t, N^s .
\Phi\left(C^t,C^s,N^t,N^s\right)$. In the following section we will discuss the
rules involved in deducing the function $\phi$, that is, the function $\Phi$
for some given source and target of a success transition.

For this purpose we will regard the tuples $(P^s,N_{vars}^s)$ and
$(P^t,N_{vars}^t)$, of a given success transition, where $P^s$ is the list of
abstract patterns derived from the function arguments in the source, and $P^t$
is the list of corresponding actual patterns in the target. Furthermore, let
$N_{vars}^s$ and $N_{vars}^t$ be unary functions of the type
$\mathbb{P}\rightarrow\mathbb{N}^*$, accepting a pattern and yielding the
variable names that are contained both in the input pattern and the sets
$N_{vars}^s$ and $N_{vars}^t$, respectively.

In the following analysis we will look at but one instance of the lists $P^s$
and $P^t$, namely the abstract pattern $p^s$ from the source and its
corresponding actual pattern in the declaration, $p^t$. In total, however, this
process has to be repeated for each such pair given the sets $P^s$ and $P^t$,
iteratively extending the definition of the relation $\phi$ to all variables
bound in the sets $N_{vars}^s$ and $N_{vars}^t$.

We initially define $\phi$ to yield the value $\bot$ for all arguments. We will
continuously modify this definition as we process $p^s$ and $p^t$. We denote
this within the semantics in a manner similar to the state $\sigma$ in the
semantics\footnote{See \referToSection{d-sos}.}. However, $\phi$ is now a
binary ``memory'', requiring both a target name and a source name (in that
order). For simplicity, we will borrow some suger coding from the matlab
notation which allows us to provide a collection in place of a single element
and let the runtime apply the given function to each element in the collection.
For instance, we might write that $\phi\left(N_{vars}^t(p^t), n^s\right)\mapsto
<$, meaning that all the target variables used in $p^t$ are strictly less than
the source variable $n^s$.

We now define a summoning rule, dividing the rules up into sub-rules:

\begin{equation}
{
    \left\langle\proc{A},p^t,p^s,\phi\right\rangle
    \rightarrow
    \phi'
  \vee
    \left\langle\proc{B},p^t,p^s,\phi\right\rangle
    \rightarrow
    \phi'
  \vee
    \left\langle\proc{C},p^t,p^s,\phi\right\rangle
    \rightarrow
    \phi'
  \vee
    \left\langle\proc{D},p^t,p^s,\phi\right\rangle
    \rightarrow
    \phi'
  \vee
    \left\langle\proc{E},p^t,p^s,\phi\right\rangle
    \rightarrow
    \phi'
}\over{
  \left\langle p^t,p^s,\phi\right\rangle
  \rightarrow
  \phi'
}
\end{equation}

One of the simpler cases is when the abstract pattern $p^s$ is simply \mono{0},
or some name $n^s$, and $n^s\in N_{calls}^s$. Since no variables bound in the
source participate in $p^s$, then no relations need to be drawn to any of the
target variables that might appear in the corresponding $p^t$. Hence, $\phi$
need not be modified.

\begin{equation}\label{eq:sct-pattern-source-fail}
{
\left(
    p^s\rightarrow 0
  \vee
\left(
    p^s\rightarrow n^s
  \wedge
    n^s\notin N_{vars}^s
\right)
\right)
  \wedge
    \phi\rightarrow\phi'
}\over{
  \left\langle\proc{A},p^t,p^s,\phi\right\rangle
  \rightarrow
  \phi'
}
\end{equation}

This has a symmetrical case. Indeed when $p^t$ is neither a destruction, nor
any name $n^t$, that is, it is \mono{\_} or \mono{0}. This pattern contains no
variables, and hence  no relations need to be drawn from any of the variables
that might appear in the corresponding $p^s$. Hence, $\phi$ need not be
modified in such a case either.

\begin{equation}
{
\left(
    p^t\rightarrow 0
\vee
    p^t\rightarrow \_
\right)
  \wedge
    \phi\rightarrow\phi'
}\over{
  \left\langle\proc{B},p^t,p^s,\phi\right\rangle
  \rightarrow
  \phi'
}
\end{equation}

If $p^t$ is the name pattern $n^t$, the matters get a bit more complicated:

\begin{enumerate}

\item If $p^s$ is some node, then all the variables that occur in $p^s$, i.e.
$N_{vars}^s(p^s)$, will all be strictly less than $n^t$ by the semantics of
\D{}. However, we are not concerned with this relation, as we would like to
know when a value is decreased from source to target, and not, as in this case,
increased.

\item If $p^s$ is also some name pattern $n^s$, and  $n^s\in N^s_{vars}$, then
the values of these corresponding variables will be \emph{equivalent}. However,
we're not concerned with exact equivalence, and simply mark this relationship
with the weaker, but still sound relation, $\leq$:

\begin{equation}
{
    p^t\rightarrow n^t
  \wedge
    p^s\rightarrow n^s
  \wedge
    n^s\in N_{vars}^s
  \wedge
    \left\langle\phi\left(n^t, n^s\right)\mapsto \leq\right\rangle\rightarrow\phi'
}\over{
  \left\langle\proc{C},p^t,p^s,\phi\right\rangle
  \rightarrow
  \phi'
}
\end{equation}

\end{enumerate}

If $p^t$ is a destruction and $p^s$ is the variable name $n^s$, then we can safely say that
all the variables that occur in $p^t$, i.e. $N_{vars}^t(p^t)$, are all strictly less
than the variable in $n^s$:

\begin{equation}
{
    p^t\rightarrow p^t_1\cdot p^t_2
  \wedge
    p^s\rightarrow n^s
  \wedge
    n^s\in N_{vars}^s
  \wedge
    \left\langle\phi\left(N_{vars}^t(p^t), n^s\right)\mapsto <\right\rangle\rightarrow\phi'
}\over{
  \left\langle\proc{D},p^t,p^s,\phi\right\rangle
  \rightarrow
  \phi'
}
\end{equation}

If both $p^t$ and $p^s$ are a destructions, then the following recursive rule applies:

\begin{equation}
{
    p^t\rightarrow p^t_1\cdot p^t_2
  \wedge
    p^s\rightarrow p^s_1\cdot p^s_2
  \wedge
    \left\langle p^t_1, p^s_1, \phi\right\rangle
    \rightarrow
    \phi''
  \wedge
    \left\langle p^t_2, p^s_2, \phi''\right\rangle
    \rightarrow
    \phi'
}\over{
  \left\langle\proc{E},p^t,p^s,\phi\right\rangle
  \rightarrow
  \phi'
}
\end{equation}
