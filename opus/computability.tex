\chapter{Computability and the halting problem}

\section{Computable problems and effective procedures}

A computable problem is a problem that can be solved by an effective procedure.

A problem can be solved by an effective procedure iff the effective procedure
is well-defined for the entire problem domain\footnote{Invalid inputs are, in
this instance, irrelevant.}, and passing a value from the domain as input to
the procedure \emph{eventually} yields a correct result (to the problem) as
output of the procedure. That is, an effective procedure can solve a problem if
it computes an injective partial function that associates the problem domain
with the range of solutions to the problem.

An effective procedure is discrete, in the sense that computing the said
function cannot take an infinite amount of time. To do this, an effective
procedure makes use of a finite sequence of steps that themselves are discrete.
This has a few inevitable consequences for the input and output values, namely
that they themselves must be discrete and that there must be a discrete number
of them\footnote{A finite sequence of discrete values can be trivially encoded
as a single discrete value.}. This is because an infinite value clearly cannot
be processed nor produced by a finite sequence of discrete steps.

An effective procedure is also deterministic, in the sense that passing the
same input value always yields the same output value. This means that all of
the steps of the procedure that are relevant to its output\footnote{All other
steps can be omitted without loss of generality.} are themselves deterministic.
This is easy to see, by a sort of cut-and-paste proof. In particular, if a
procedure made use of a stochastic process to yield a deterministic result, the
stochastic process would have to yield the same output for the same input in
order for the global determinism property to be withheld, which is clearly
absurd.

%In effect, a procedure can be said to comprise of a finite sequence of other
%procedures, which themselves may comprise of other procedures, however, all
%procedures eventually bottom out, in that a finite sequence of composite
%procedures can always be replaced by a finite sequence of basic procedures that
%are implemented in underlying hardware.

%The procedure input is a continuous machine state, the output is a discrete
%change of the continuous state. And yet, the values that a procedure may use to
%produce its result are all discrete.

%Instructions that the computer can execute may have no effect, but such
%instructions are seldom of any particular interest. More amusing instructions
%alter some sort of state that the computer withkeeps across instructions.

%An instruction
%commands a machine to perform a discrete and deterministic alternation of the
%its own state. For the simplest of purposes, the machine state can comprise of
%the index of the currently executing instruction, known as the program counter,
%the current state of the machine memory (tape), and the finite sequence of
%instructions that the machine is using as its program.

%The machine state comprises of the state of the tape, the current position on
%tape and the program.

%The instruction sequence, is by its sequential nature enumerable. Initial
%program execution would be to iterate through the sequence in a particular
%direction, e.g. top-down. In this process we implicitly make use of a ``program
%counter'', which is a value that points to the index of the next instruction to
%execute. When the program counter moves past the end of the instruction
%sequence, the procedure halts.

%In this model, a ``loop'' is constructable by having a special \emph{jump}
%instruction that changes the program counter to some value less than the
%current program counter.

%A ``loop'' can be constructed by a sequence of such instructions, due its
%sequential nature being enumerable, and with the existance of a special
%\emph{jump} instruction, that changes the 

%Clearly, a procedure that eventually halts and returns the correct output for
%every input in the problem domain, solves the problem.


%Procedures take in an input value and produce an output value. Multiple values
%can be represented as a single value via. a pairing function.

%\begin{itemize}

%\item effective procedure

%\item effectively decidable

%\item effectively enumerable

%\end{itemize}

\section{Enumerability}

\begin{definition} A set $S$ is enumerable if there exists a bijective function
$f:S\longrightarrow \mathbb{N}$.\end{definition}

\section{Turing machine}\label{section:turing-machine}

Alan Turing was one of the first to introduce a powerful, yet simplistic
computational model, in the form of what is now known as a \emph{Turing
machine}\footnote{Although various definitions exist, this one was inspired by
\url{http://plato.stanford.edu/entries/turing-machine/}.}.

A Turing machine is a machine that has unlimited and unrestricted memory in the
form of a one dimensional tape that can be extended infinitely in both
directions, known as left and right. The tape itself is a series of symbols of
a finite alphabet.  Furthermore, the Turing machine has a read/write head that
at any given point in time is located over a single symbol on the tape. The
head can hence read in the symbol and either overwrite it, move one symbol to
the left, or move one symbol to the right, both in one computational step.

The actual action to perform is defined in terms of a transition table which is
a finite list of of the 4-tuples $\left\langle \lambda_0, \sigma, \lambda_1,
\omega \right\rangle : \Lambda \times \Sigma \times \Lambda \times \Omega$,
where $\Lambda$ is the set of states of the machine (some non-existent),
$\Sigma$ is the alphabet of the machine, and hence the tape, typically
$\{0,1\}$, and $\Omega$ is the action table of the machine. The action table is
typically the set $\{ \leftarrow, \rightarrow, 0, 1 \}$, denoting a left-wise
and right-wise move, as well as write $0$ and $1$, respectively. At any given
point in time a Turing machine is in a state, say $\lambda_0$. Such a state has
one and only one transition, namely $\left\langle \lambda_0, \sigma, \lambda_1,
\omega \right\rangle$. If the symbol on the tape is equal to $\sigma$, the
action $\omega$ is performed and regardless of the symbol, the machine moves
into state $\lambda_1$.

Last but not least, the Turing machine has an initial state, and often an
accepting and rejecting state. We'll keep the definition simple and let any
state that is undefined in the transition table yield a halting of the machine.
The value which the read/write head is located over, can hence represent an
accepting or rejecting state as needed.

\section{Computational equivalence}

\begin{definition} We say that two languages are computationally equivalent if
they both can compute the same class of functions.\end{definition}

\section{The halting problem}

\begin{theorem} There does not exist a Turing machine $H(M,x)$ that given an
arbitrary Turing machine $M$ and an arbitrary input $x$ determines whether $M$
halts for $x$.\end{theorem}

\begin{proof}

Assume for the sake of contradiction that there exists a Turing machine
$H(M,x)$, that given a Turing machine $M$ decides whether $M$ halts on input
$x$, that is

$$H(M,x)=\left\{
\begin{array}{ll}
true&M\ \text{halts on}\ x\\
false&M\ \text{does not halt on}\ x.
\end{array}
\right.$$

This means that we can construct another Turing machine $F$, which calls
$H(M,M)$ and depending on the outcome, yields the opposite result, i.e. 

$$F(M)=\left\{
\begin{array}{ll}
true&\text{if}\ H(M,M)\leadsto false\\
false&\text{if}\ H(M,M)\leadsto true.
\end{array}
\right.$$

Consider now running the Turing machine $F$ with itself as input, either result
is absurd, and hence neither the Turing machine $F$ nor $H$ can
exist.\end{proof}

\section{Introduction to size-change termination}

In the chapters that follow we will describe the size-change termination
principle as in \cite{size-change}, and extend it slightly. The general idea
behind the technique is to construct a control flow graph for a given program
and consider the cycles in that graph.

If every control flow cycle reduces a value of a well-founded data type on
every iteration, i.e. monotonically decreases the value, then, rather
intuitively, the program must terminate. In particular, a well-founded data
type is a data type where any set of values has a minimal element. The
technique hence relies on the fact such a value eventually bottoms out and the
cycle, unable to decrease the value any further, must inevitably terminate.
