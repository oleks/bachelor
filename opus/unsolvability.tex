\chapter{On the general unsolvability of the halting problem}

A computable problem is a problem that can be solved by an effective procedure.
A problem can be solved by an effective procedure iff the effective procedure
is well-defined for the entire problem domain, and iff passing a value from the
domain as input to the procedure yields a correct result (to the problem) as
output of the procedure. Invalid inputs are, in this instance, irrelevant.

That is, an effective procedure computes a partial function which represents a
relationship between the input and output values s.t. it represents a solution
to the said problem.

Procedures themselves are algorithms, comprised of text, or more specifically,
a finite sequence of discrete and deterministic instructions, i.e. without
continous and stochastic processes.


Procedures take in an input value and produce an output value. Multiple values
can be represented as a single value via. a pairing function.

\section{Computation environment}

Finitely many different instructions

A procedure is a finite length sequence of instructions, i.e. a countable set
that can be enumerated.

Every instruction is discrete, as in there are no continuous processes.

Every instruction is deterministic, as in there are no stochastic processes.

\section{Effectiveness}

Effective procedure

Effectively enumerable

Effectively decidable

Recursively enumerable -- countable sets

Co-recursively enumerable


% 155
% 22
% 23
% 170
% 98
% 141
% 54
% 163
% 115
