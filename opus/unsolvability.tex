\chapter{On the general unsolvability of the halting problem}

A computable problem is a problem that can be solved by an effective procedure.
A problem can be solved by an effective procedure iff the effective procedure
is well-defined for the entire problem domain, and iff passing a value from the
domain as input to the procedure eventually yields a correct result (to the
problem) as output of the procedure. Invalid inputs are, in this instance,
irrelevant. That is, an effective procedure computes a partial function which
associates the problem input domain to it's output range.

Procedures themselves are algorithms, comprised of text, or more specifically,
a finite sequence of finite, discrete and deterministic instructions, that is,
without infinite, continous or stochastic processes. They operate on finite
values, and the finiteness of values, individual instructions and the
instruction sequence means that the effective procedure eventually halts or
goes into an infinite loop for any given input. 

The instruction sequence, is by it's sequential nature enumerable. Initial
program execution would be to iterate through the sequence in a particular
direction, e.g. top-down. In this process we implicitly make use of a ``program
counter'', which is a value that points to the index of the next instruction to
execute. When the program counter moves past the end of the instruction
sequence, the procedure halts.

In this model, a ``loop'' is constructable by having a special \emph{jump}
instruction that changes the program counter to some value less than the
current program counter.

A ``loop'' can be constructed by a sequence of such instructions, due it's sequential nature being enumerable, and with the existance of a special \emph{jump} instruction, that changes the 

Clearly, a procedure that eventually halts and returns the correct output for
every input in the problem domain, solves the problem.


Procedures take in an input value and produce an output value. Multiple values
can be represented as a single value via. a pairing function.



\section{Computation environment}

Finitely many different instructions

A procedure is a finite length sequence of instructions, i.e. a countable set
that can be enumerated.

Every instruction is discrete, as in there are no continuous processes.

Every instruction is deterministic, as in there are no stochastic processes.

\section{Effectiveness}

Effective procedure

Effectively enumerable

Effectively decidable

Recursively enumerable -- countable sets

Co-recursively enumerable


% 155
% 22
% 23
% 170
% 98
% 141
% 54
% 163
% 115
