\chapter{Preface}

\section{Motivation}

The halting problem is undecidable in general, however this property is often
abused to deduce that for all programs. The intent of this project is to
explore some context in which the halting property \emph{is} decidable, and to
analyze how useful this indeed is.

\section{Expectations of the reader}

The reader is expected to have a background in computer science on a graduate
level or higher. In particular, it is expected that the reader is familiar with
basic concepts of compilers, computability and complexity, which are subject to
basic undergraduate courses at the state of writing. Furthermore the reader is
expected to be well familiar with the basic concepts of mathematics tought in
undergraduate computer science courses.

For those still in doubt, it is expected that the following terms can be used
without definiiton:

\begin{itemize}

\item Algorithm

\item Recursion

\item Induction

\item Big $O$ Notation

\item Regular Expressions (\mono{preg} syntax)

\item Backus-Naur Form

\item Halting Problem

\item Turing Machine

\end{itemize}


