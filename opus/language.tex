\chapter{Language}

\section{The language \mono{D}}

The language \mono{D} is defined in the simplest possible terms.

That is,
extraneous syntactical sugar is left out of the language definition and
initially defined by functions. If it should prove practical to extend the
syntax for various purposes later on in the text, this will be done.

\newcommand{\mathmono}[1]{\ensuremath{\text{\mono{#1}}}}
\newcommand{\nonterm}[1]{\ensuremath{\text{\mono{<#1>}}}}
\newcommand{\term}[1]{\ensuremath{\text{\mono{`#1'}}}}

Programs are defined as follows:

\begin{align}
\nonterm{program}\ &::=\ \nonterm{type}^*\ \nonterm{function}^*
\ \nonterm{expression}\\
\end{align}

That is, a program is a possibly empty set of type declarations followed by a
possibly empty set of function declarations followed by a single expression.

The non-terminal \mono{<type>} designates an infinite algebraic data-type
defined as follows:

\begin{align}
\nonterm{type}\ &::=\ \term{type}\ \nonterm{type-name}
\ \term{:=}\ \nonterm{constructor}\ (\term{|}
\ \nonterm{constructor})^*\\
\nonterm{type-name}\ &::=\ \nonterm{literal}\\
\nonterm{literal}\ &::=\ \nonterm{char}
\ \left (\ \left (\term{-}\ |\ \nonterm{char}\right )^*
\ \nonterm{char}\ \right )?\\
\nonterm{char}\ &::=\ [\term{a}\mathmono{-}\term{z}]\\
\nonterm{constructor}\ &::=\ \nonterm{constructor-name}
\ [\ \term{of}\ \nonterm{type-name}\ ]\\
\nonterm{constructor-name}\ &::=\ \nonterm{literal}\ \text{s.t.}
\ \nonterm{constructor-name}\notin \nonterm{type-name}
\end{align}



Functions are defined as follows:

\begin{align}
\nonterm{function}\ &::=\ \nonterm{function-name}
\ \nonterm{argument}^*\ \term{:=}\ \nonterm{expression}\\
\nonterm{function-name}\ &::=\ \nonterm{literal}\ \text{s.t.} 
\nonterm{argument}\ &::=\ \nonterm{literal}\ \text{, s.t.}\nonterm{argument}\notin{\nonterm{type-name}\cup\nonterm{function-name}}
\end{align}

The infinite algebraic datatype.


In addition to these simple features, we define a set of functions $B$
containing the function $if-then-else$..

If this definition should prove impractical in further affairs, such as lack of
basic types of basic functions, these will be introduced as needed instead of
being included in the core grammar.
