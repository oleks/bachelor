\chapter{Extended-BNF}\label{appendix:ebnf}

This report makes use of an extended version of the Backus-Naur form (BNF).
This appendix is provided to cover the extensions employed in the report. This
is done instead of using some universal extension since no universally
acknowledged extension seemingly exists, like there is a universally
acknowledged Backus-Naur form, namely the one used in the ALGOL 60 Reference
Manual\cite{algol-bnf}.

\section{What's in common with the original BNF}

The following parts are in-common with the original Backus-Naur form:

\makeTable{bnf}{Constructs in common with the original BNF}
{|c|p{0.5\textwidth}|}
{\textbf{Construct} & \textbf{Description}}
{
$<\ldots>$ & A metalinguistic variable, aka. a nonterminal.\\
$::=$ & Definition symbol\\
$|$ & Alternation symbol
}

In the original BNF, everything else represents itself, aka. a terminal.

\section{Constructs borrowed from regular expressions.}

This extension has it in common with many other extensions that it encapsulates
terminals in single-quotes.

This allows us to give characters such as $($, $)$, $]$, $]$, $*$, $+$, and
${}^*$ special meaning, namely:

\makeTable{ebnf}{Constructs borrowed from regular expressions.}
{|c|l|}
{\textbf{Construct}&\textbf{Meaning}}
{
$(\ldots)$ & Entity group\\
$[\ldots]$ & Character group\\
$\text{-}$ & Character range\\
$*$ & $0\text{-}\infty$ repetition\\
$+$ & $1\text{-}\infty$ repetition\\
$?$ & $0\text{-}1$ repetition
}

An entity group is a shorthand for an auxiliary nonterminal declaration. This
means, for instance, that using the alternation symbol within it would mean an
alternation of entity sequences within the entity group rather than the entire
declaration that contains the entity group.

A character group may only contain single character terminals and an
alternation of the terminals is implied from their mere sequence. It is
identical to an auxiliary single character nonterminal declaration. A character
range binary operator can be used to shorten a given character group, e.g.
$[\term{a}\mathmono{-}\term{z}]$ implies the list of characters from $\term{a}$
to $\term{z}$ in the ASCII table.  Moreover, a character range is the only
operator allowed in a character group.

Applying the repetition operators to either the closing brace of an entity
group or the closing bracket of a character group has the same effect as
applying the repetition operator to their respective hypothetical auxiliary
declarations.

\section{Nonterminals as sets}
