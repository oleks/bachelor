\section{The structured operational semantics used in this work}\label{appendix:sos}

The following section describes the syntax used in this text to describe the
operational semantics of the language \D{}. The syntax is inspired by
\cite{sos}, but differs slightly.

\subsection{Some general properties}

\begin{itemize}

\item Rules should be read in increasing order of equation number.

\item If some rule with a lower equation number makes use of an undefined
reduction rule, it is because the reduction rule is defined under some higher
equation number.

\item Rules can be defined in terms of themselves, i.e. they can be recursive,
even mutually recursive.

\end{itemize}

\subsection{Atoms}\label{appendix:sos:atoms}

To keep the rules clear and concise we'll make use of atoms to subdivide a rule
into subrules and distinguish those rules from the rest. If you're familiar
with Prolog, this shouldn't be particularly new to you.

For instance, a chained expression $x$ may have the following semantics: 

\begin{equation}\label{appendix:equation:expression}
{\displaystyle
  \left\langle \proc{Single}, x,\sigma\right\rangle
  \rightarrow
  \left\langle v,\sigma\right\rangle
\vee
  \left\langle \proc{Chain}, x,\sigma\right\rangle
  \rightarrow
  \left\langle v,\sigma\right\rangle
\over\displaystyle
  \left\langle x,\sigma\right\rangle
  \rightarrow
  \left\langle v,\sigma\right\rangle
}
\end{equation}

This means that either the rule corresponding to the single element expression
($\left\langle \proc{Single}, x,\sigma\right\rangle\rightarrow\left\langle
v,\sigma\right\rangle$) validates, or the rule corresponding to the element
followed by another expression ($\left\langle \proc{Chain},
x,\sigma\right\rangle\rightarrow\left\langle v,\sigma\right\rangle$) does.

Atoms are used in both propositions and conclusions of rules. For instance,
\ref{appendix:equation:single} defines one of the subrules to the above rule.

\subsection{The proposition operators}

\subsubsection{The $=$ operator}

The notation used in \cite{sos} does not make use of atoms\footnote{See
\referToAppendix{sos:atoms}.}, but instead leaves the reader stranded guessing
which rule to apply next. This is derivable from the language syntax, so
usually this is isn't a problem. For instance, if an expression is either an
if-statement or a while-loop we wouldn't find a summoning rule for expressions,
but rather ``orphan rules'' like the following:

\begin{equation*}
{\displaystyle
  \cdots
\over\displaystyle
  \left\langle\If\ e\ \kw{then}\ c_1\ \kw{else}\ c_2, \sigma\right\rangle
  \longrightarrow
  \cdots
}
\end{equation*}

\begin{equation*}
{\displaystyle
  \cdots
\over\displaystyle
  \left\langle\kw{while}\ e\ \kw{do}\ c, \sigma\right\rangle
  \longrightarrow
  \cdots
}
\end{equation*}

In the notation used in this text we define a summoning rule first, such as
(\ref{appendix:equation:expression}), and use atoms to subdivide that rule into
subrules. The subrules are then defined further down, such as
(\ref{appendix:equation:single}). However, we still need a way to distinguish
between things like if-statements and for-loops, or in the case of the running
example elements and expressions.

Hence, the first part of the proposition of a subrule will often begin with a
``rule'' that uses the $=$ operator. For instance, $x = e$ means that the
expression $x$ that we're considering really is just a single element, or $x =
e\cdot x'$ means that the expression $x$ that we're considering really is a
construction of an element $e$ and some other expression $x'$.

\subsubsection{The $\rightarrow$ operator}

\cite{sos} uses the operator $\longrightarrow$ to indicate a transition. Since
we will blend this operator with other binary operators like $\wedge$ and
$\vee$, and wish for the transition to have higher precedence\footnote{See
\referToAppendix{sos:precedence}.}, it is visually more appropriate to use the
$\rightarrow$ operator, since that keeps the vertical space between the
operators roughly the same as between the operators $\wedge$ and $\vee$.

\subsubsection{The $\wedge$ operator}

The $\wedge$ operator is used as a conventional \emph{and} operator to combine
multiple rules that must hold in a proposition. The left-to-right evaluation
order is superimposed on the binary operator such that the ending values of the
left hand rule can be used in the right hand rule. For instance, in the
following rule, the value $e$ resulting from validating the left side of the
$\wedge$ operator is carried over to the right side of the operator and used in
another rule.

\begin{equation}\label{appendix:equation:single}
{\displaystyle
  x\Rightarrow e
\wedge
  \left\langle e,\sigma\right\rangle
  \rightarrow
  \left\langle v,\sigma\right\rangle
\over\displaystyle
  \left\langle \proc{Single}, x,\sigma\right\rangle
  \rightarrow
  \left\langle v,\sigma\right\rangle
}
\end{equation}

\subsubsection{The $\vee$ operator}

The $\vee$ operator is used as a conventional short-circuited \emph{or}
operator. That is, a left-to-right evaluation order is also superimposed but
evaluation stops as soon as one of the operands holds.

\subsubsection{Operator precedence}\label{appendix:sos:precedence}

To avoid ambiguity, and having to revert to using parentheses we'll define the
precedences of the possible operators in the prepositions of rules. Operators
with higher precedence are bind tighter:

\begin{enumerate}

\item $\vee$

\item $\wedge$

\item $\rightarrow$

\end{enumerate}
