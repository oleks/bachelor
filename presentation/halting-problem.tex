\begin{frame}

$$H(M,x)=\left\{
\begin{array}{ll}
true&M\ \text{halts on}\ x,\\
false&M\ \text{does not halt on}\ x.
\end{array}
\right.$$

\end{frame}

\begin{frame}

$$H(M,x)=\left\{
\begin{array}{ll}
true&M\ \text{halts on}\ x,\\
false&M\ \text{does not halt on}\ x.
\end{array}
\right.$$

$$F(M)=\left\{
\begin{array}{ll}
true& H(M,M) \leadsto false,\\
false& H(M,M) \leadsto true.
\end{array}
\right.$$

$$\text{Consider}\ F(F).$$

\end{frame}

\begin{frame}

$$H(M,x)=\left\{
\begin{array}{ll}
true&M\ \text{halts on}\ x,\\
false&M\ \text{does not halt on}\ x,\\
unknown&M\ \text{may or may not halt on}\ x.
\end{array}
\right.$$

% The definition used in the report.

\end{frame}

\begin{frame}

$$H(M,x)=\left\{
\begin{array}{ll}
true&M\ \text{halts on}\ x,\\
unknown&M\ \text{may or may not halt on}\ x.
\end{array}
\right.$$

% To be honest, given the size-change termination principle, this definition is
% even better.

\end{frame}
