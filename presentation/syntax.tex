\begin{frame}

\frametitle{The syntax of \D{}}

\begin{align*}
\nonterm{program}\ \textt{::=}&\ \nonterm{clause}^{\color{red}\textt{+}}
\ \nonterm{expression}
& \\
\nonterm{expression}\ \textt{::=}&\ \nonterm{element}\ \textt{(}\ \term{.}
\ \nonterm{expression}\ \textt{)}\ \textt{?}
& \\
\nonterm{element}\ \textt{::=}&\ \term{0}\ \textt{|}\ \term{(}
\ \nonterm{element}\ \term{)}\ \textt{|}\ \nonterm{name}\ \textt{|}
\ \nonterm{application}
& \\
\nonterm{application}\ \textt{::=}&\ \nonterm{name}
\ \nonterm{expression}^{\color{yellow}\textt{*}}
& \\
\nonterm{clause}\ \textt{::=}&\ \nonterm{name}
\ \nonterm{pattern}^{\color{yellow}\textt{*}}\ \term{:=}
\ \nonterm{expression}
& \\
\nonterm{pattern}\ \textt{::=}&\ \nonterm{pattern-element}\ \textt{(}
\ \term{.}\ \nonterm{pattern}\ \textt{)}\ \textt{?}
& \\
\nonterm{pattern-element}\ \textt{::=}&\ \term{0}\ \textt{|}\ \term{\_}
\ \textt{|}\ \term{(}\ \nonterm{pattern}\ \term{)}\ \textt{|}\ \nonterm{name}
& \\
\nonterm{name}\ \textt{::=}&\ \textt{[}\term{a}\mathmono{-}\term{z}\textt{]}
\ \textt{(}\ \textt{[}\term{-}\ \term{a}\mathmono{-}\term{z}\textt{]}^\textt{*}
\ \textt{[}\term{a}\mathmono{-}\term{z}\textt{]}\ \textt{)}\ \textt{?}
\end{align*}

(14)

% <program> should probably be <clause>+ <expression>, since most interesting
% problems have at least one clause.

% Why did we avoid constants? No reason. Constants are 0-ary functions that
% trivially terminate.

\end{frame}
